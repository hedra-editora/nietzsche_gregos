\titulo{A filosofia na era trágica dos gregos}
\autor{Nietzsche}
\organizador{Org. e trad.}{Fernando R.~de Moraes Barros}
\isbn{978-85-7715-026-7}
\preco{16}
\pag{128}
\release{\textsc{A filosofia na era trágica dos gregos}, publicado pela 
primeira vez na íntegra em edição brasileira, constitui um dos comentários 
mais importantes da filosofia pré{}-socrática. Escrito póstumo e inacabado, Nietzsche defende nele 
a tese de que os pensadores anteriores a Platão foram os únicos que ousaram compreender a dimensão 
trágica das forças que regem a vida dos homens. Ao contrário da filosofia posterior, não reduziram 
metafisicamente a realidade à dimensão do certo e do errado. Se costumamos atribuir a Sócrates o início 
da filosofia, Nietzsche sugere aqui que talvez ela tenha justamente terminado com ele.

Um dos textos mais desafiadores do jovem \mbox{Nietzsche,} que, 
ao dar vida aos pensadores pré-socráticos, ``personagens'' de uma trama intricada e cheia 
de peripécias especulativas, combina erudição e identificação simpática, saber documentado e 
intuições viscerais. Fiando-se na sabedoria dos antigos, esta obra fundamental permite compreender o filosofar 
pré-socrático como a arte de não sucumbir à circunstância trágica e caótica de nossa existência. 

\noindent\textbf{Fernando R.~de Moraes Barros} é doutor em filosofia pela Universidade 
de São Paulo (\textsc{usp}) e professor da Universidade Estadual de Santa Cruz 
(\textsc{uesc}). É autor de \textit{A maldição transvalorada -- o problema da civilização 
em} O anticristo \textit{de Nietzsche} (Discurso/Unijuí, 2002) 
e \textit{O pensamento musical de Nietzsche} (Perspectiva, 2007).}















