\hyphenation{Nie-tzsche nie-tzsche-a-na nie-tzsche-a-no}

\chapter[Introdução, por Fernando R. de Moraes Barros]{Introdução}
\markboth{Introdução}{}

\textsc{É com certo} desconforto que \textit{A filosofia na era trágica dos
 gregos}, de 1873, deixa"-se agrupar em torno dos lugares"-comuns
 habitualmente reservados ao pensamento do jovem Nietzsche. Aqui, não toma a
 palavra o menino prodígio da filologia clássica alemã a fim de traduzir e
 interpretar, com destemida frieza analítica, textos e manuscritos da
 Antiguidade; e tampouco o romântico pensador iconoclasta, que conta dirigir
 provocações ao austero cânon da tradição histórica, localizando o despertar
 da cultura ática a partir do convulsivo exercício de impulsos dionisíacos.
 Aliás, a própria pergunta pela origem da filosofia, se não se acha
 desatendida, é ao menos desestabilizada em termos de sua convencional
 decifração. Não por acaso, dir"-se"-á alto e bom som: ``As perguntas sobre
 os inícios da filosofia são absolutamente indiferentes''.\footnote {A
 tradução foi feita a partir da seguinte edição: Nietzsche, Friedrich. ``Die
 Philosophie im tragischen Zeitalter der Griechen''. In:\textit{Sämtliche Werke}. 
 \textit{Kritische Studienausgabe}. Edição organizada por Giorgio Colli e 
 Mazzino Montinari. Berlim/Nova York, Walter de Gruyter,
 1999, vol.~\textsc{i}, p.~806.} (p.~\pageref{perguntassobreosinicios)} 

\section{a origem da filosofia como questão} 

É claro que a questão acerca do ponto de partida do filosofar é de importância nevrálgica para os passos
 argumentativos contidos no texto de Nietzsche e, por certo, está presente de
 ponta a ponta. Todavia, o fundamental é que, mesmo quando tenha de ser
 colocada, a pergunta pelo nascimento da filosofia se torne atuante apenas
 como as técnicas de cinzelagem se acham presentes na bela obra de um
 escultor, isto é, como um meio e não como um fim em si mesmo. Isto porque,
 para o autor de
\textit{A filosofia na era trágica dos gregos}, o problema da origem, longe de
 comportar a objetividade da datação lexicológica, abriga um sentimento de
 existência mais recuado, revelando uma procedência que nos escapa quando
 permanecemos aferrados unicamente à letra compilada pelo doxógrafo, a saber,
 ``uma atmosfera pessoal, uma coloração de que se pode lançar mão a fim obter
 a imagem do filósofo'' (p.~\pageref{atmosferapessoal}). 

\section{A maneira de viver e o conceito}

É certo que o autor de tais linhas não é ainda o célebre genealogista que,
anos mais tarde, terminará por introduzir a questão do ``valor dos valores'',
tomando sobre si a hercúlea missão de descerrar a atividade subterrânea
exercida pelos instintos à base de nossas apreciações valorativas. Movido,
contudo, por um visceral interesse genealógico, o jovem escritor julga
``possível inferir o solo a partir da planta'' (p.~\pageref{possivelinferir}) 
e, com isso, remete os primeiros ``sistemas'' filosóficos,
não ao saber exclusivamente documentado, mas ao âmbito que designa ``\textit
{a maneira} de viver'' (p.~\pageref{amaneirade}). Albergando distintos
ângulos de visão ou diferentes maneiras de viver, o solo a partir do qual
nasceu a filosofia não é algo unívoco e implica, antes do mais, a pergunta
pela ``pessoa'' por detrás de cada empreendimento filosófico. Tanto é assim
que, de saída, somos advertidos:

\begin{quote} 
Eu conto a história de tais filósofos de um modo simplificado: 
espero destacar apenas o ponto de cada sistema que é um pedaço
 de \textit{personalidade} e pertence àquele aspecto incontestável 
 e indiscutível, a ser preservado pela história (p.~\pageref{simplificado}).
\end{quote}

Penosa é a tarefa de reencontrar tais marcas pessoais para, a partir daí,
remontar às condições teórico"-especulativas de que partiram os filósofos da
Antiguidade. Pois, se, ``com Platão, inicia"-se algo inteiramente novo''
(p.~\pageref{complatao}), justamente para nós, rebentos laicizados do código
platônico de representação e herdeiros do uso apofântico da linguagem, é
quase impossível referir"-{}se a tais pensadores sem transportar, ao mesmo
tempo e às escondidas, os pesados depósitos semânticos formados pela moderna
maneira de pensar. Se, como indica Nietzsche, ao homem antigo era
``incrivelmente árduo apreender o conceito como conceito'' (p.~\pageref{incrivelmentearduo}), 
ao moderno tipo cultural de homem cumpre,
inversamente, sublimar o que há de mais pessoal sob a forma de intrincadas
significações abstratas. Reféns de um conhecimento haurido da erudição
fragmentada e pulverizadora, somos então levados a aceitar uma atávica e
inafugentável inacessibilidade:

\begin{quote} 
A tarefa a ser levada a cabo por um filósofo no interior de
 uma efetiva cultura, formada segundo um estilo unitário, não se deixa
 adivinhar com perfeita clareza a partir de nossas condições e vivências,
 porque simplesmente não dispomos de tal cultura (p.~\pageref{tarefaaserlevada}).
\end{quote} 

\section{o grande homem} 

Isto não significa, porém, que a chamada era trágica dos filósofos gregos 
esteja, para nós, morta e enterrada como realidade
 histórico"-cultural. Sem resignação, seria ainda possível visitá"-la e
 recuperá"-la ``mediante comparação'' (p.~\pageref{comparacao}). Mais até.
 Por meio de uma curiosa espécie de intuição congenial, Nietzsche reputa
 exequível a tarefa de trazer à tona ``aquilo que nenhum conhecimento
 posterior poderá nos roubar: o grande homem'' (p.~\pageref{ograndehomem}). 
 Modelos"-vivos, os primeiros filósofos se prestariam, pois,
 a uma reprodução que parte da identificação simpática com o ``grande
 homem''. Acreditando que espíritos afins já de longe se reconhecem, o autor
 de \textit{A filosofia na era trágica dos gregos} permite"-se, de sua parte,
 reproduzir tais homens com a ponta de sua pena.

\paragraph{Tales de Mileto} E o primeiro ``grande homem'' a ser esculpido é
 Tales. A filosofia teria aqui seu início por ter sido ele o primeiro a
 pensar a totalidade sem subterfúgios e fabulações, enunciando algo a
 propósito da origem do mundo mediante uma suposição cuja legitimidade não
 depende, em rigor, de um ideário miticamente orientado: ``O valor do
 pensamento de Tales [\ldots] está, antes, no fato de ele não ter sido
 almejado mítica e alegoricamente'' (p.\,\pageref{pensamentodetales}). É bem
 verdade que, em sua contabilidade, Nietzsche não deixa de acusar um débito
 com o símile ``tudo é água'' quanto ao modo místico de agir e sentir, já
 que, a seu ver, aquilo que conduziu Tales a tal hipótese global ``foi um
 dogma metafísico que se origina numa intuição mística [\ldots] `tudo é
 um'\,'' (p.\,\pageref{foiumdogma}). Contudo, e apesar disso, o filósofo
 milésio não teria renunciado à ousadia de fiar"-se na natureza, na medida em
 que, ao menos, fia"-se na água; e, se sua fantasia apanha, aqui e acolá,
 ``certezas que estão a voar'' (p.\,\pageref{certezasque}), a reflexão
 filosófica termina por garantir propósito e disposição à sua imaginação
 esvoaçante, trazendo à baila suas ``medidas e seus moldes'' 
 (p.\,\pageref{medidaseseusmoldes}) -- o que, no caso da fantasmagoria místico"-religiosa,
 está longe de ser algo evidente.

\paragraph{Anaximandro de Mileto} Mas, se a ideia de totalidade permanece,
 aqui, submetida às injunções mais imediatas de uma expressão física, a
 unidade daquilo que existe assume, com Anaximandro, uma coloração metafísica
 radicalmente diferente, permitindo"-lhe auferir, de resto, a alcunha de
 ``primeiro escritor filosófico da Antiguidade'' (p.\,\pageref{primeiroescritor}). 
 Que se retome, à guisa de introdução, sua sentença
 lapidar -- tal como Nietzsche a transcreve: ``Lá onde as coisas têm seu
 surgimento, para lá também devem ir ao fundo, segundo a necessidade; pois
 têm de pagar expiação e ser justiçadas por suas injustiças, conforme a ordem
 do tempo'' (p.\,\pageref{laondeascoisas}). Máxima sombria de um pessimista
 extremo ou divisa oracular a oscilar enigmaticamente sobre a filosofia, como
 levá"-la em consideração? 

Ora, o autor de \textit{A filosofia na era trágica dos gregos} bem sabe que,
em linhas gerais, não há uma resposta inequívoca à pergunta pelo
\textit{apeiron}, o ilimitado, já que ela implicaria, como condição de
 compreensibilidade, um isto ou aquilo apto a qualificá"-la, ou seja, algo
 delimitado. Tratar"-se"-ia, então, no fundo, de uma espécie de nada
 indecifrável, ou, quiçá, de uma ideia que impõe limites a outras ideias, mas
 cuja realidade objetiva não pode ser factualmente conhecida. É nesse
 sentido, aliás, que Nietzsche dirá:

\begin{quote} 
Essa última unidade naquele `indeterminado', ventre materno
 de todas as coisas, só pode, com efeito, ser descrito negativamente pelo
 homem, isto é, como algo a que não pode ser concedido qualquer predicado
 advindo do mundo existente do vir"-a"-ser (p.\,\pageref{essaultimaunidade}).
\end{quote} 

 Irremediável, essa impenetrabilidade parece imantar a caracterização de
 Anaximandro a uma férrea e condenatória apreciação da existência, pois, se
 não é dado ao homem entrever o ilimitado, tudo se passa como se dele
 tampouco fosse existencialmente independente. Ao contrário, inclusive. Foi
 por meio do \textit{apeiron} que, segundo Nietzsche -- que aqui se vale de
 espessas lentes schopenhauerianas --, o pensador grego logrou ponderar sobre
 o drama da morte e do sofrimento, partindo do aceite de que nenhuma
 efetividade pode redimir o homem da condição injusta de sua finitude. E,
 desta feita, será preciso reconhecer ``todo vir"-a"-ser como uma emancipação
 do ser eterno digna de punição, isto é, como algo injusto que deve ser
 expiado com o declínio'' (p.\,\pageref{sereterno}). Condenada por uma espécie
 de maldição lançada pela ordenação moral do universo, a vida humana seria
 como que arrastada pela dinâmica vampírica do pessimismo em si; e o
 indivíduo, complacente com o eterno ciclo culpabilizante em que se acha
 enredado, agora tem de bastar"-se com o seguinte estado de coisas: 

\begin{quote} 
o que vale vossa existência? E, se ela de nada vale, então
 com qual finalidade estais aí? Por vossa culpa, pelo que observo,
 permaneceis nesta existência. Tereis que expiá"-la com a morte 
 (p.\,\pageref{oquevale}).
\end{quote} 

\paragraph{Heráclito de Éfeso} É nítido que, como anátema proferido contra a
 existência, o aludido ``delito ontológico'' só pode ensejar um corrosivo
 movimento de desvalorização deste mundo em nome de outro, imutável,
 indelineável e incólume aos reveses humanos. Mas a tensão do arco
 argumentativo reflete, aqui, um acúmulo de forças que visa ao arremesso
 verdadeiramente perfurante da flecha a ser disparada texto adentro; como se
 o mais acintoso descrédito do existir também não pudesse deixar de sugerir,
 como contra"-ideal, um antipódico hino à glorificação do vir"-a"-ser. Assim
 é que, servindo"-se de tal ocasião, Nietzsche volta os holofotes em direção
 a uma personagem que lhe é muito cara: ``No meio dessa noite mística na qual
 se achava encoberto o problema do vir"-a"-ser de Anaximandro, adentrou
 Heráclito de Éfeso e iluminou"-a por meio de um relâmpago divino''
 (p.\,\pageref{noitemistica}).

Com Heráclito, ganha terreno o pensamento da instabilidade, sendo que dois
seriam os passos dados para alcançá"-lo: primeiramente, o filósofo efésio
teria denegado a ideia de dois mundos distintos entre si e, num segundo
momento, mais arrojado e temerário, teria suprimido o ser em geral.
Duplamente corajosa, a negação teria como consequência uma superação positiva
do ser eterno e igual a si mesmo, que, doravante, passa a depender do outro
de si para adquirir autovaloração, isto é, do próprio vir"-a"-ser. O
resultado seria uma visão de conjunto inspirada pela infixidez pura e
plena: 

\begin{quote} 
Pois, esse único mundo que lhe sobrou -- escudado ao seu
 redor por leis eternas não escritas, fluindo de cima a baixo conforme a
 brônzea batida do ritmo -- não mostra, em nenhum lugar, uma persistência,
 uma indestrutibilidade, um lugar seguro na correnteza (p.\,\pageref{esseunicomundo}).
\end{quote} 

Mas, não só inexiste tal boia salva"-vidas, apta a manter"-nos imóveis e
seguros à flor da água, como sequer haveria um único e idêntico rio fora de
nós. Assustadora, a representação formada a partir da impermanência de tudo o
que existe faz do curso polimorfo da natureza uma figura incômoda e, quando
não, insuportável. Seu influxo, dirá Nietzsche, ``se aproxima ao máximo da
sensação de quem, num abalo sísmico, perde a confiança na terra bem
firmada'' (p.\,\pageref{abalosismico}). Mas, afora o temor, tal hipótese
global de interpretação comporta, no mínimo, uma dificuldade; pois, ainda que
infinitamente novo, o vir"-a"-ser não pressupõe, aqui, uma força
ilimitadamente crescente. Donde vem, então, o rio sempre renovado do
vir"-a"-ser? 

À resposta a essa pergunta somos levados por Heráclito por meio da ideia de
luta, haja vista que o próprio vir"-a"-ser é concebido sob a forma de um
conflito universalmente conflagrado, como o desmembramento de toda força em
duas atividades opostas que, esforçando"-se uma em direção à outra, lutam, em
paradoxal beligerância, por sua reunificação. Sem trégua ou descanso, a luta
concederia, por assim dizer, a chancela da eternidade ao vir"-a"-ser: 

\begin{quote} 
Todo vir"-a"-ser surge da guerra dos opostos: as qualidades
 determinadas, que se nos aparecem como sendo duradouras, exprimem tão"-só a
 prevalência momentânea de um dos lutadores, mas, com isso, a guerra não
 chega a seu termo, senão que a luta segue pela eternidade 
 (p.\,\pageref{sereterno}).
\end{quote} 

Não se trata, porém, de substancializar o vir"-a"-ser no interior de uma
efetividade impermeável a qualquer transcendência; tampouco basta desviar,
por um atalho, o rumo da metafísica dogmática, que, mediante inferências
inquestionáveis, deriva o condicionado a partir do incondicionado. Cumpre
ainda evitar a tentação de imprimir à efetividade, sob o pretexto de afirmar
a eterna regularidade do vir"-a"-ser, o tom uniforme da unidade
dogmaticamente alicerçada; se não há ser imóvel, então tampouco pode haver
unidades estabilizadoras sob aquilo que vem a ser; o que também indica,
noutra chave, que seria um tanto pueril dirigir insultos ao pessimismo
anaximândrico em prol de um júbilo que, em rigor, apenas mantém ao revés as
extremidades implantadas pela doutrina de dois mundos. E, afinal de contas,
se nada pode redimir o homem de sua condição, melhor seria dizer que
Anaximandro e Heráclito eram, cada um a seu modo, pensadores trágicos,
vertendo, cada qual, seu quinhão de lágrimas. Tragicidade, aliás, que não
escapa ao próprio Nietzsche:

\begin{quote} 
Acreditamos na tradição segundo a qual ele
 [Anaximandro] caminhava aqui e acolá [\ldots] demonstrando um orgulho
 verdadeiramente trágico em seus gestos e hábitos de vida. [\ldots] erguia a
 mão e apoiava o pé como se esta existência fosse uma tragédia na qual ele,
 como herói, tivesse nascido para nela tomar parte (p.\,\pageref{acreditamosnatradicao}).
\end{quote} 

\paragraph{Parmênides de Eleia} O desafio consiste, em verdade, em proceder ao
 questionamento do ser eterno a partir de seu próprio domínio. O que também
 significa, num certo sentido, que o terreno mais condizente com a acalentada
 tarefa crítica não é propriamente o jônico, senão o eleata. Daí, a outra
 importante personagem trazida à baila por Nietzsche, única a ombrear, como
 contra"-ideal, com o caudaloso e instável rio de Heráclito:

\begin{quote} 
com seu contemporâneo \textit{Parmênides} uma
 contra"-imagem coloca"-se ao seu lado, fazendo igualmente as vezes de um
 tipo de profeta da verdade, mas como que formado de gelo, e não de fogo,
 emanando luz fria e perfurante ao seu redor (p.\,\pageref{comseucontemporaneo}).
\end{quote} 

Rubricando as qualidades que se lhe apresentavam a partir dos habituais polos
dicotômicos à base do pensar dualista, Parmênides opera, em geral, com
divisões e comparações. Segundo Nietzsche, seu método era o seguinte: 

\begin{quote} 
ele tomava um par de opostos como, por exemplo, leve e
 pesado, fino e espesso, ativo e passivo, e mantinha"-os à luz daquela
 oposição exemplar entre luz e escuridão; aquilo que correspondia à luz era o
 positivo, aquilo que correspondia à escuridão, a propriedade negativa
 (p.\,\pageref{eletomavaumpar}).
\end{quote} 

Afeito ao modo lógico"-abstrato de raciocinar, tal método teria conduzido o
filósofo eleata à substituição do par positivo"-negativo pelos termos
``existente'' e ``não"-existente'', afastando"-o não apenas das impressões
sensíveis mais singulares, mas também do âmbito que designa a própria
efetividade, já que esta última, além da falta de qualidades positivas, passa
a expressar igualmente a ausência mesma de ser. Taxando as propriedades ora
de positivas ora de negativas, Parmênides estaria, no fundo, tachando o
vir"-a"-ser de irreal e enganoso. Desde então, de acordo com Nietzsche, o
interesse do eleata pelos fenômenos ``atrofia"-se, engendrando para si um
ódio pelo fato de não poder livrar"-se deste eterno embuste dos sentidos''
(p.\,\pageref{odio}).

É bem verdade que, vez por outra, a descrição nietzschiana de Parmênides soa
algo caricatural e reproduz, para o seu desfavor, certos prejuízos e
generalizações. Não por acaso, Giorgio Colli irá comentar a esse respeito: 

\begin{quote} 
poder"-se"-ia desconfiar, inclusive, que a caracterização
 do	\label{colli} eleata -- frieza, abstração esvaída em sangue, negação da
 vida, tautologia do conhecimento -- contradiz precisamente a
 verdade.\footnote{ Cf. Nietzsche, Friedrich. ``Nachwort''. In:
\textit{Sämtliche Werke}. \textit{\mbox{Kritische} Studienausgabe}. Edição
 organizada por Giorgio Colli e Mazzino Montinari. Berlim/Nova York, Walter
 de Gruyter, 1999, vol.\,\textsc{i}, p.~917.} 
\end{quote} 

Aquilo que nos importa, porém, é a consequência filosófica a que somos, aqui,
impelidos. Como o avesso da multiplicidade, o ser ideado pelo pensador eleata
logra uma identidade absoluta e, em oposição à volubilidade do mundo
sensível, identifica"-se com a própria atividade reflexiva, fundando a vida
consciente numa unidade simples e substancial, a contrapelo do vir"-a"-ser e
da natureza. Mais até. Com Parmênides, torna"-se possível o argumento
ontológico, de sorte que, agora, poder"-se"-ia inferir a existência do ser a
partir do mero conceito ``ser''. Nesse sentido, Nietzsche escreve: 

\begin{quote} 
O tema da ontologia prenuncia"-se na filosofia de
 Parmênides. Em nenhuma parte a experiência lhe ofereceu um ser tal como
 havia ideado, mas, porque foi capaz de pensá"-lo, ele então concluiu que tal
 ser tinha de existir (p.\,\pageref{temadaontologia}).
\end{quote} 

\paragraph{Zenão de Eleia} Mas, se, em tal cenário, ser e pensar se recobrem,
 então a pergunta pela possibilidade do vir"-a"-ser também irá adquirir uma
 outra formulação. Faz"-se a experimentação de mobilizar a inteira força do
 raciocínio lógico contra os dados sensíveis elementares, retirando da
 própria transitoriedade, por assim dizer, os argumentos que a tornariam
 inconcebível ao pensamento. A ideia mesma de infinitude concorre, aqui, para
 pôr em cheque e soterrar as condições de realização do movimento. Algo que
 vem à luz, por exemplo, por meio de outro eleata, fiel depositário do ser
 parmenidiano e célebre pelo seguinte paradoxo:

\begin{quote} 
Zenão valeu"-se, sobretudo, de um método de demonstração
 indireto: disse, por exemplo, que ``não pode haver nenhum movimento de um
 lugar para outro, pois, se tal movimento existisse, então haveria uma
 infinitude acabada; isto é, porém, uma impossibilidade''. Aquiles não pode,
 na corrida, alcançar a tartaruga que detém uma pequena vantagem à sua
 frente (p.\,\pageref{zenaovaleuse}).
\end{quote} 

Valendo"-se da ideia de série infinita, Zenão irá pôr em causa a possibilidade
de percorrer, num tempo limitado, a extensão de uma reta ilimitadamente
fracionada. Inexplicável ao pensamento, tal situação levaria à insólita
conclusão de que nem o movimento nem a transitoriedade seriam logicamente
pertinentes.

\paragraph{Anaxágoras de Clazômenas} É esse o horizonte hermenêutico contra o
 qual o adversário do imobilismo deverá defrontar"-se. Pouco importa, a essa
 altura, que o ser eternamente imóvel de Xenófanes se ache eivado de
 misticismo religioso. Talvez seja, afinal, a ``concepção de um velho homem
 que, por fim, sossegou"-se, daquele que, depois da mobilidade de seus
 descaminhos [\ldots] encontra, diante da alma, o que há de mais elevado
 [\ldots] na permanência de todas as coisas'' (p.\,\pageref{velhohomem}). 
 Imanente, a crítica terá que se dirigir aos conceitos
 utilizados pelo próprio Zenão, detendo"-se na ardilosa sobreposição da noção
 de divisibilidade infinita à ideia de grandeza determinada. Trata"-se de
 mostrar que, no transcurso de seu trajeto rumo à tartaruga, Aquiles não
 estaria, em realidade, subdividindo uma reta em secções cada vez menores,
 mas apenas esticando os pontos de uma linha cuja estrutura é caudatária da
 concepção de tempo entendido como
\textit{continuum}. O entimema do paradoxo estaria no fato de que, aqui, o
 movimento mesmo é confundido com a representação da sucessão temporal ínsita
 ao nosso sentido interno. Não por acaso, entra em cena, logo a seguir,
 o \textit{nous} de Anaxágoras, pois, como ação de apreender algo pelo
 intelecto, ele faz as vezes de uma sucessão que cuida de si mesma:
 ``Então -- disse ele a si mesmo -- há algo que traz consigo a origem e o
 início do movimento'' (p.\,\pageref{aorigemeoinicio}).

Elevado, porém, a princípio ativo do universo, o \textit{nous} anaxagórico tem
que se haver com um problema de fundo, que vai além da concatenação
sequencial de certas representações, a saber, o movimento material. Por isso,
``em segundo lugar, ele [Anaxágoras] ainda constata que tal representação não
move apenas a si própria, senão que também algo completamente diferente, a
saber, o corpo'' (p.\,\pageref{asipropria}). As teorias mais antigas elegem,
como se viu, um elemento primordial como causa do vir"-a"-ser -- seja ele
água, ar, fogo ou o indeterminado. Anaxágoras, de seu lado, procura afirmar
que a mudança nunca poderia ser explicada, a contento, a partir de um único
ser existente. E há motivos para tanto. Imputar a mudança a tal ser
equivaleria a separar o movimento de um ``sujeito'' do movimento, como se o
vir"-a"-ser fosse a mística transformação de uma coisa naquilo que lhe é
formalmente diferente. Se se trata de manter o mínimo de inerência nas
relações, será preciso lembrar: 

\begin{quote} 
as coisas que são necessariamente estranhas entre si não
 podem exercer qualquer tipo de efeito uma em relação à outra, e, portanto,
 tampouco podem mover ou ser movidas (p.\,\pageref{coisasquesao}).
\end{quote} 

Dizer que é o movimento do espírito que move a matéria implica aceitar, mas
sem perder de vista o que lhes é intrínseco, que ambos jamais poderiam
relacionar"-se sem alguma causalidade comum. Mas se não há, de resto, nada
que subsista entre os dois, então, como dirá Nietzsche, ``o contato é tão
ininteligível quanto a atração mágica'' (p.\,\pageref{ocontatoe}).

Aos poucos, torna"-se patente que a indicação a processos que condicionam o
nascer e o perecer passa pela elucidação da causa da mudança e do
deslocamento; mas pouco valeria, aqui, recorrer ao conceito exclusivamente
mecânico de atividade, pois, se conceber o movimento como consequência de
algo acarreta remontar à sua causa inicial, no caso dos movimentos mecânicos,
o primeiro elemento da série não poderia, em todo caso, assentar"-se noutro
movimento mecânico, já que, como dirá Nietzsche, ``isto significaria o mesmo
que recorrer ao conceito absurdo de \textit{causa sui}'' (p.\,\pageref{causasui}). 
Torna"-se então imperioso descerrar um princípio movente que não
seja eternamente imóvel e que, apenas de quando em quando, tome sobre si a
tarefa de mover a matéria. Daí, a saída entrevista por Anaxágoras: ``Nesse
dilema, Anaxágoras julgou encontrar uma salvação e um socorro extraordinário
naquele \textit{nous} que se move a si mesmo e que de nada depende; cuja
essência é suficientemente obscura e velada para poder iludir quanto ao fato
de que sua suposição também envolve aquela \textit{causa sui} proibida''
(p.\,\pageref{nessedilema}).

O fato de ser caracterizado com os atributos do pensamento não faz do
\textit{nous}, porém,\textit{ }o útero primordial do universo e tampouco o
 núcleo primitivo do movimento em si, incorrendo ainda em injúria aquele que
 dele fala ``como de um deus \textit{ex machina}'' (p.\,\pageref{exmachina}). 
 Ocorre que a ele se antepõe, não como posse demiúrgica do que
 vem a ser, mas como possibilidade do que existe, um amálgama caótico e
 aglutinante. No fundo, o vir"-a"-ser seria uma espécie de decomposição
 premeditada, pressupondo complexas configurações nas quais se desenrola toda
 sorte de trocas e combinações: 

\begin{quote} uma mistura que ele[Anaxágoras] imaginou como um completo
 penetrar"-se em si mesmo até a mais ínfima parte, depois que todas aquelas
 existências elementares estivessem trituradas como numa argamassa e
 dissolvidas em átomos de pó, de sorte que, em meio a tal caos, elas pudessem
 ser remexidas entre si tal como numa batedeira (p.\,\pageref{mistura}).
\end{quote} 

Evitando derivar o múltiplo daquilo que é uno e indivisível, Anaxágoras
retrata a existência primordial do mundo como se fosse uma mescla heteróclita
de elementos infinitamente pequenos, onde cada qual é um simples, mas de tal
modo que a totalidade ``também se assemelha às suas partes'' (p.\,\pageref{suaspartes}). 
Assim, a multiplicidade se deixaria compreender a partir da
singularidade e o todo, por seu turno, dar"-se"-ia a conhecer igualmente nas
partes em que se revela. Todavia, sustentar que o cosmos adveio do ``caos''
não equivale, aqui, a conceber o começo do mundo como um conjunto de
qualidades que poderiam engendrar e movimentar a matéria, mas que, de fato e
de direito, não tinham por destinação engendrá"-la e movimentá"-la. O que
também significa que ``caos'' e ``caos'' são duas palavras distintas segundo
o seu emprego, pois tomar o termo como pretexto para designar uma
criação \textit{ex nihilo}, é olvidar que Anaxágoras, como diz Nietzsche, não
coloca ``a pergunta pela razão de algo existir (\textit{causa finalis}), mas,
em todos os casos e antes de tudo, a pergunta pelo meio através do qual algo
existe(\textit{causa efficiens})'' (p.\,\pageref{causafinalis}).

Por um caminho diferente, mas não adverso, voltamos, com isso, a Heráclito.
Também aqui não se pode dizer que há algo a ser atingido no percurso linear e
constante de um incremento das forças que atravessam o mundo; à diferença de
um \textit{telos} necessário e previsível sob a efetividade: 

\begin{quote} o \textit{nous} não possui nenhuma obrigação e, portanto,
 tampouco algum propósito que estivesse forçado a cumprir; se uma vez deu
 início àquele movimento e estabeleceu para si uma meta -- a resposta é
 difícil, sendo que o próprio Heráclito poderia respondê"-la --, tudo isto
 não passou de um
\textit{jogo} (p.\,\pageref{naopossuinenhuma}).
\end{quote} 

\section{O vir"-a"-ser}

 E, de volta ao jogo heraclitiano, cumpre indagar à
 guisa de conclusão e curiosidade: o que aprender com \textit{A filosofia na
 era trágica dos gregos}?

Em primeiro lugar, que o vir"-a"-ser não é nenhum autômato que, deixando"-se
condicionar pelo programa imposto pelos seus construtores, se compraz na
atividade maquinal que lhe foi atribuída; em segundo lugar, que a unidade não
é, necessariamente, sinônimo de simplicidade; ao contrário, como organização,
ela implica um misto composto cuja unidade talvez não seja assegurada sequer
nominalmente. E Nietzsche, desconfiado dos acosmismos que fazem a
multiplicidade desaparecer sob o pesado manto do uno, não deixa de prestar
testemunhos de sua inegável predileção pelo vir"-a"-ser. Ao enfatizar, não só
as antigas hipóteses de interpretação do homem e do universo, mas também as
vidas singulares que sob elas respiram, o filósofo alemão não pretende,
porém, cultuar personalidades ou erigir ídolos. E tampouco poderia ser
diferente. Afinal de contas: ``Outros povos possuem santos, enquanto que os
gregos, por sua vez, têm sábios'' (p.\,\pageref{santos}).
