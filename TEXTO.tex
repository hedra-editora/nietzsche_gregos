\SVN $Id: TEXTO.tex 5377 2010-01-07 17:18:18Z jorge $	

\chapter{Primeiro prefácio}

\textsc{No que tange} a homens muito recuados no tempo, basta"-nos saber
seus objetivos para enaltecê"-los ou reprová"-los como um todo. Quanto
aos que nos são mais próximos, julgamos de acordo com os meios pelos
quais eles cumprem seus objetivos: não raro, condenamos seus objetivos,
mas terminamos por amá"-los em virtude dos meios e do tipo de seu
querer. Os sistemas filosóficos são integralmente verdadeiros apenas
aos olhos daqueles que os fundaram: a todos os filósofos ulteriores,
eles consistem, em geral, num grande equívoco e, para as mentes mais
enfraquecidas, numa soma de equívocos e verdades. Mas, considerados
como o mais elevado objetivo, tais sistemas não passam, em todo caso,
de um erro e, nessa medida, de algo reprovável. Por isso, muitos são
os homens que reprovam tais filósofos, já que seu objetivo não é o
mesmo que o deles; neste caso, trata"-se daqueles que se acham mais
distantes no tempo. Em contrapartida, aqueles que em geral se regozijam
com os grandes homens, também se alegram com tais sistemas, ainda que
sejam equivocados de fio a pavio: afinal de contas, eles contêm em si
um ponto incontestável, a saber, uma atmosfera pessoal, uma coloração \label{atmosferapessoal}
de que se pode lançar mão a fim de obter a imagem do filósofo; assim como
é possível inferir o tipo de solo a partir da vegetação em um determinado local. \label{possivelinferir}
Em todo caso, \textit{a maneira} de viver, bem como o modo de encarar as coisas \label{amaneirade}
humanas, já existiu e é, por conseguinte, possível: o ``sistema'' é a
vegetação que cresce em tal solo, ou, ao menos, uma parte deste sistema. 

Eu conto a história de tais filósofos de um modo simplificado: espero destacar \label{simplificado}
apenas o ponto de cada sistema que é um pedaço de
\textit{personalidade} e pertence àquele aspecto incontestável e
indiscutível, a ser preservado pela história; trata"-se de
uma tentativa inicial para recuperar tais naturezas mediante \label{comparacao}
comparação, bem como para recriar e fazer finalmente ressoar, uma vez
mais, a polifonia da natureza grega; a tarefa consiste em trazer à luz
aquilo que devemos \textit{sempre amar} e \textit{ter em altíssima conta}, 
e aquilo que nenhum conhecimento posterior poderá nos roubar: o \label{ograndehomem}
grande homem.


\chapter[Segundo prefácio]{Segundo prefácio}


\textsc{Essa tentativa} de contar a história dos antigos filósofos gregos\footnote{ Cumpre lembrar, à guisa de
esclarecimento, que este segundo prefácio foi escrito pela mãe do
filósofo alemão, sendo"-lhe ditado pelo filho, provavelmente, no inverno
de 1875. [Todas as notas são do tradutor, exceto quando indicadas.]}
distingue"-se de outras tentativas similares pela concisão. Logrou"-se
esta última pelo fato de que, em cada filósofo, aludiu"-se apenas a um
ínfimo número de suas doutrinas, quer dizer, operou"-se com sua incompletude.
Optou"-se, no entanto, por aquelas doutrinas nas quais
ressoa com mais intensidade o que há de pessoal num filósofo, ao passo
que uma enumeração completa de todas as proposições possíveis que nos
foram deixadas, tal como é de praxe nos manuais, só traz, em todo caso,
uma coisa à baila, a saber, o completo emudecimento diante do que há de
pessoal. Eis porque aqueles relatos terminam por se tornar tão
tediosos: pois, em sistemas que foram refutados, apenas o que há de
pessoal pode interessar"-nos, haja vista ser isto o eternamente
irrefutável. É possível formar a imagem de um homem a partir de três
anedotas; deixando de lado o excedente, esforço"-me então para
ressaltar, a partir de cada sistema, três anedotas.

\chapter{A filosofia na era trágica dos gregos}

\sectionitem

Há adversários da filosofia: e é bom colocar"-se à sua escuta; em
especial, se eles tratam de dissuadir as mentes doentias dos alemães da
metafísica, apregoando"-lhes, em contrapartida, a purificação por meio
da \textit{physis}, tal como Goethe ou, então, a cura mediante a
música, tal como Wagner. Os médicos do povo repudiam a filosofia;
àquele que pretende justificá"-la cabe, pois, mostrar com qual
finalidade os povos sadios dela se servem e se serviram. Caso consiga
mostrá"-lo, talvez os próprios enfermos terminem por compreender, em seu
favor, o motivo pelo qual justamente para eles a filosofia é danosa.
Há, com efeito, bons exemplos de uma saúde que pode existir sem
qualquer filosofia ou que dela faz um uso bem comedido e, quando não,
jocoso; tanto é assim que, em seu melhor período, os romanos viveram
sem filosofia. Onde se acharia, em compensação, o exemplo do
adoecimento de um povo ao qual a filosofia tivesse devolvido a saúde
perdida? Sempre que a filosofia se manifestava de modo solícito,
salvífico ou preventivo, isso se dava com os povos sãos, ao passo que,
com os doentes, ela os tornava ainda mais doentes. Sempre que um povo
se achava desmembrado e, ainda assim, ligado aos seus indivíduos por
uma tênue tensão, a filosofia jamais reatou tais indivíduos mais
intimamente ao todo. Sempre que alguém se colocava de bom grado à
parte, para cobrir"-se com o manto da auto"-suficiência, a filosofia
estava sempre pronta para isolá"-lo ainda mais e destruí"-lo mediante
isolamento. Lá onde não se encontra em seu pleno direito, ela é
perigosa: sendo que apenas a saúde de um povo, e não de todo povo, pode
conceder"-lhe tal direito.

Lancemos agora o olhar em direção àquela suprema autoridade que, num
dado povo, cumpre chamar de saúde. Os gregos, como os verdadeiramente
saudáveis, \textit{justificaram} a filosofia de uma vez por todas na
medida em que filosofaram; e muito mais do que quaisquer outros povos;
eles não puderam cessar nem mesmo no momento certo; pois, mesmo num
período pouco fértil, eles se mostraram ardorosos admiradores da
filosofia, mesmo que, por filosofia, eles entendessem as dóceis
sofistarias e os penteados sacrossantos da dogmática cristã. Mas, por
terem sido incapazes de parar no momento certo, eles próprios
encurtaram muitíssimo sua contribuição à posteridade bárbara, já que
esta, em sua inculta e truculenta juventude, teve de ser capturada por
aquelas redes e cordas artificialmente urdidas. 

Em contrapartida, os gregos souberam começar no momento certo, de sorte
que, quando se faz necessário dar início ao filosofar, eles ensinam
essa lição mais claramente do que qualquer outro povo. Não como algo
que se dá, em primeiro lugar, na adversidade: tal como presumem, com
efeito, aqueles que derivam a filosofia da aflição, mas, sim, na
felicidade, numa puberdade madura, no interior da serenidade flamejante
de um momento de vida vitorioso e corajoso. O fato de os gregos terem
filosofado em tal período nos ensina muito acerca daquilo que a
filosofia vem a ser e deve realizar, e, de resto, acerca dos próprios
gregos. Se eles tivessem sido, à época, iguais àqueles espíritos
experimentados e serenos, prosaicos e precoces, como imagina para si o
erudito filisteu de nossos dias, ou, então, se eles tivessem vivido
apenas num libidinoso flutuar, ressoar, respirar e sentir, tal como
gosta de supor o visionário iletrado, a fonte da filosofia jamais teria
vindo à luz entre eles. Teria havido, no máximo, um riacho prestes a
desaparecer na areia ou evaporar numa neblina, mas nunca aquele amplo
rio que se derrama e que, com seu orgulhoso bater de ondas, se nos dá a
conhecer como filosofia grega.

Com efeito, mostrou"-se com fervor quanto os gregos puderam encontrar e
aprender nas terras estrangeiras orientais, bem como quantas coisas
diversas eles apanharam de lá. Por certo, foi um espetáculo fantástico
quando aproximaram os supostos mestres do oriente dos possíveis
aprendizes da Grécia, trazendo à tona, aí então, Zoroastro ao lado de
Heráclito, os hindus ao lado dos eleatas, os egípcios ao lado de
Empédocles ou até mesmo Anaxágoras entre os judeus e Pitágoras entre os
chineses. No detalhe, pouco foi estabelecido a esse propósito; o pensamento como um todo nos agrada, mas desde que não nos sobrecarreguem com
a conclusão de que a filosofia foi, pois, simplesmente importada para a
Grécia, deixando de crescer a partir de um solo natural e endógeno, e
inclusive que ela, como algo estranho, teria antes arruinado do que
dado amparo aos gregos. Nada é mais tolo do que atribuir aos gregos uma
formação [\textit{Bildung}] autóctone, já que, ao contrário, 
eles absorveram em si toda formação viva de outros povos, logrando chegar assim
tão longe, porque sabiam justamente atirar a lança a partir do ponto em
que um outro povo a havia largado. No que tange à arte de aprender com
fecundidade, são dignos de admiração: e, como eles, \textit{devemos}
aprender, não a conhecer com vistas à erudição, mas a viver com nosso
vizinho, valendo"-se de tudo o que foi aprendido como suporte a partir
do qual se pode pular para além e mais acima dele. As perguntas sobre \label{perguntassobreosinicios}
os inícios da filosofia são absolutamente indiferentes, pois, no
início, há sempre o cru, o disforme, o vazio e o feio, e, em todas as
coisas, apenas os níveis mais elevados são levados em conta.
Aquele que, no lugar da filosofia grega, prefere ocupar"-se com as
filosofias egípcia e persa, porque estas são, quiçá, mais ``originais'' e
decerto mais antigas, procede de modo tão temerário quanto aqueles que
não conseguem sossegar enquanto não tiverem remetido a mitologia grega,
tão profunda e soberba, às trivialidades físicas, tais como o sol, o
relâmpago, o clima, a névoa, bem como aos seus inícios primordiais;
aqueles que, por exemplo, imaginam ter redescoberto, na limitada
adoração de uma única abóbada celeste pelos bárbaros indo"-germânicos,
uma forma mais pura de religião do que teria sido o culto politeísta
dos gregos. O caminho rumo aos inícios conduz, em todos os lugares, à
barbárie; e, aquele que se ocupa com os gregos, deve sempre levar em
conta o fato de que, em si mesmo e em todas as épocas, o desenfreado
impulso ao conhecimento barbariza tanto quanto o ódio ao conhecimento,
e que, por consideração à vida, por meio de uma necessidade ideal de
vida, os gregos domaram seu intrinsecamente insaciável impulso ao
conhecimento -- porque desejavam viver, de imediato, aquilo que
aprendiam. Os gregos também filosofaram como homens da cultura e com os
objetivos da cultura, e, por isso, eximiram"-se de inventar, uma vez
mais, os elementos da filosofia e da ciência a partir de alguma
arrogância autóctone; ao contrário, trataram rapidamente de completar,
elevar, erguer e purificar de tal modo os elementos por eles absorvidos
que, a partir de então, tornaram"-se inventores num sentido mais elevado
e numa esfera mais pura. Inventaram, a ser assim, as \textit{mentes tipicamente filosóficas}, 
sendo que a inteira posteridade não inventou mais
nada de essencial que se lhe acrescentasse.

Todo povo é exposto ao ridículo quando nos referimos a um grupo de
filósofos tão maravilhosamente idealizado como aquele pertinente aos
antigos mestres gregos Tales, Anaximandro, Heráclito, Parmênides,
Anaxágoras, Empédocles, Demócrito e Sócrates. Todos esses homens são
integrais e como que talhados a partir de uma única pedra. Entre o seu
pensar e o seu caráter, vigora uma rígida necessidade. Falta"-lhes toda
convenção, pois, à época, não havia nenhuma profissão de filósofo ou
erudito. Acham"-se, todos eles, numa solidão majestosa, como aqueles que
então viveram unica e exclusivamente o conhecimento. Todos possuem a
virtuosa energia dos antigos por meio da qual ultrapassam todos aqueles
que lhe sucedem, encontrando sua própria forma e aperfeiçoando"-a,
mediante metamorfose, naquilo que ela tem de mais ínfimo e portentoso.
Pois não lhes vinha em auxílio nenhum estilo facilitador. Assim é que,
juntos, eles formam aquilo que Schopenhauer, em contraposição à
república dos eruditos, chamou de república"-dos"-gênios: um gigante
chama pelo outro durante o árido intervalo das épocas e, não se
deixando perturbar por ruidosos anões travessos que se dependuram sob
eles, a elevada conversa entre os espíritos segue adiante.

Sobre essa elevada conversa entre os espíritos propus"-me a contar aquilo
que nossa moderna e insensível capacidade de escuta consegue ouvir e
compreender: isso significa, por certo, sua mínima parte. A mim me
parece que, em tal conversa, aqueles antigos sábios de Tales a Sócrates
falaram a respeito de tudo aquilo que, para a nossa ponderação,
constitui a peculiaridade helênica, ainda que o tenham feito sob a
forma mais geral. Em sua conversa, bem como em suas personalidades,
eles desenvolvem os grandes traços do gênio grego, do qual a inteira
história grega constitui somente uma reprodução sombreada, uma cópia
borrada e, portanto, que nos fala de modo impreciso. Se
interpretássemos corretamente a inteira vida do povo grego,
encontraríamos invariavelmente apenas a imagem refletida que irradia,
com as cores mais brilhantes, de seus mais esplêndidos gênios. Até
mesmo a primeira vivência da filosofia ocorrida em solo grego, a
sanção dos sete sábios, constitui uma linha clara e inesquecível na
imagem do mundo helênico. Outros povos possuem santos, enquanto os \label{santos}
gregos, por sua vez, têm sábios. Já foi dito, com razão, que um povo
não é tão bem caracterizado por meio de seus grandes homens quanto pela
maneira com que estes são por ele conhecidos e honrados. Noutros
tempos, o filósofo é um andarilho solitário e ocasional a mover"-se
furtivamente no mais hostil dos ambientes, ou, então, é aquele que abre
caminho com punhos cerrados. Apenas com os gregos o filósofo não é
ocasional: quando, nos séculos sexto e quinto, ele surge sob os enormes
perigos e seduções da laicização, despontando, por assim dizer, da
caverna de Trofônio\footnote{ Criador do Templo de Delfos, Trofônio foi
sepultado numa caverna que se tornara afamada pelos seus oráculos.} 
rumo ao centro da prodigalidade, do prazer da descoberta, da
riqueza e sensualidade das colônias gregas, pressentimos então que,
qual um nobre anunciador, ele vem à baila com a mesma finalidade com a
qual, naqueles séculos, a tragédia nasceu e com a qual os mistérios
órficos se nos dão a conhecer nos grotescos hieróglifos de suas
práticas. O juízo de tais filósofos acerca da vida e da existência é
incomparavelmente mais pleno de sentido do que um juízo moderno devido
ao fato de que eles tinham a vida diante de si numa prodigiosa perfeição
e porque, à diferença de nós, neles o sentimento do pensador não se
perde no antagonismo próprio ao desejo por liberdade, beleza, grandeza
de vida e impulso à verdade, e apenas indaga: de que vale, em geral, a
vida? A tarefa a ser levada a cabo por um filósofo no interior de uma \label{tarefaaserlevada}
efetiva cultura, formada segundo um estilo unitário, não se deixa
adivinhar com perfeita clareza a partir de nossas condições e
vivências, porque simplesmente não dispomos de tal cultura. Apenas uma
cultura como a grega pode responder à pergunta pertinente àquela tarefa
do filósofo, somente ela está apta, como disse, a justificar a
filosofia de modo geral, porque só ela sabe e pode demonstrar como e
por que razão o filósofo \textit{não} é um andarilho ocasional e
arbitrário, que se dispersa aqui e acolá. Há uma necessidade férrea que
prende o filósofo a uma legítima cultura: mas, e se tal cultura
inexiste? O filósofo é, aí então, um cometa incomensurável e, por isso,
assustador, sendo que, em casos mais favoráveis, ele chega a brilhar
como um astro principal no sistema solar da cultura. Eis porque os
gregos justificam o filósofo, por ele não ser, somente entre eles, um
mero cometa.

\sectionitem

Depois de tais considerações, não parecerá algo chocante se eu falar
sobre os filósofos pré"-platônicos como se tratasse de uma sociedade
congregada a fim de dedicar, pois, este escrito única e exclusivamente
a eles. Com Platão, inicia"-se algo inteiramente novo; ou, para falar \label{complatao}
com igual propriedade, pode"-se dizer que, em comparação com aquela
república"-dos"-gênios que vai de Tales até Sócrates, desde Platão falta
algo essencial aos filósofos. Aquele que conta expressar"-se com desdém
acerca desses mestres antigos pode chamá"-los de unilaterais e seus
epígonos, com Platão em seu ápice, de multifacetados. Mais acertado e
imparcial seria, porém, compreender esses últimos como personagens
filosoficamente mistas e, os primeiros, como os tipos puros. O próprio
Platão constitui a primeira e grande personagem mista, tanto em sua
filosofia quanto em sua personalidade. Em sua doutrina das ideias,
elementos socráticos, pitagóricos e heraclitianos acham"-se unidos: por
isso, ela não é um fenômeno tipicamente puro. Também como homem Platão
mistura os traços de Heráclito, auto"-suficiente e majestaticamente
reservado, de Pitágoras, melancolicamente compassivo e legislativo, bem
como os traços de Sócrates, dialético conhecedor de almas. Todos os
filósofos posteriores constituem tais personalidades mistas; lá onde
aparece algo de unilateral entre eles, como, por exemplo, entre os
cínicos, não se trata de um tipo, mas de uma caricatura. Muito
mais relevante, porém, é o fato de que são instituidores de seitas e
que as seitas por eles instituídas eram, em seu conjunto, organizações
que se opunham à cultura helênica e à unidade de estilo de então. À sua
maneira, buscavam redenção, mas somente para os indivíduos ou, no
máximo, para os grupos de amigos e jovens que lhes eram mais próximos.
A atividade dos filósofos mais antigos, ainda que lhes fosse
inconsciente, consiste numa cura e numa purificação em larga proporção;
o poderoso curso da cultura grega não deve ser detido, sendo que
perigos terríveis devem ser afastados de seu caminho, e ao filósofo, por
seu turno, cabe proteger e defender sua pátria. Agora, desde Platão,
ele se acha no exílio, conspirando contra sua terra natal. É uma
verdadeira infelicidade que nos tenha ficado tão pouca coisa de
tais mestres filosóficos antigos e que nos tenha escapado tudo o que
havia de completo a seu respeito. Devido a essa perda, medimos
involuntariamente tais mestres de acordo com falsas medidas e, pelo
fato meramente acidental de que nunca faltaram apreciadores e copistas
para Platão e Aristóteles, deixamo"-nos levar pela opinião desfavorável
sobre os primeiros. Alguns supõem existir um destino próprio para os
livros, um \textit{fatum libellorum}: este deveria ser, em todo caso,
muito maldoso se julgou por bem nos privar de Heráclito, do maravilhoso
poema de Empédocles, dos escritos de Demócrito, que os antigos
igualavam aos de Platão e que inclusive o superava em ingenuidade,
colocando"-nos forçosamente em mãos, como que em substituição, estoicos,
epicuristas e Cícero. Provavelmente, a mais portentosa parte do
pensamento grego, bem como sua expressão sob a forma de palavras, está
perdida para nós: um destino que não surpreenderá aquele que se recorda
do infortúnio de Scotus Rigena e Pascal, e que leva em conta o fato de
que, mesmo neste século iluminado, a primeira edição de \textit{O mundo
como vontade e representação} deve ter sido transformada em papel de
rascunho. Se alguém supõe a atuação de um poder fatalista específico
por detrás dessas coisas, então ele deve fazer e falar como Goethe:
``Que ninguém se queixe daquilo que é espúrio; pois também aqui é a voz
do poder que vos fala''.\footnote{ Primeiro verso do poema ``Paz de
espírito do andarilho''. Cf., a esse respeito, Goethe, Johann Wolfgang
von. \textit{Westöstlicher Divan}. In: \textit{Werke. Hamburger
Ausgabe}. Munique, Deutscher Taschenbuch Verlag (\textsc{dtv}), 2000, vol.~\textsc{ii},
p.~47.} Em particular, trata"-se de algo ainda mais poderoso do
que o poder da verdade. Muito raramente a humanidade produz um bom
livro no qual, com ousada liberdade, entoa"-se o grito de guerra da
verdade, a canção do heroísmo filosófico: e o fato de tal livro
perdurar por mais de um século ou transformar"-se em poeira, eis algo
que depende dos mais miseráveis acidentes, do repentino ofuscamento das
mentes, de convulsões supersticiosas e antipatias, e, por fim, de dedos
avessos à escrita ou até mesmo de vermes e clima chuvoso. Não queremos,
porém, queixar"-nos, senão, de modo bem diferente, fazer valer,
aqui, as reveladoras palavras de consolo dirigidas por Hamann aos
eruditos que se queixam sobre as obras perdidas: ``Ao artista que faz
passar uma lentilha através do buraco de uma agulha não seria o
bastante, para exercer a habilidade por ele aprendida, um alqueire de
lentilhas? Essa pergunta deveria ser dirigida a todos os eruditos que
não sabem valer"-se das obras dos antigos tão perspicazmente quanto
aquele que sabe valer"-se das lentilhas''.\footnote{ Conhecido por sua
cautela crítica frente aos poderes emancipatórios da razão moderna, bem
como por sua desconfiança diante das esperanças humanitárias da
\textit{Aufklärung}, Johann Georg Hamann (1730--1788), também conhecido como ``mago do norte'',
tornou"-se atuante como um dos patronos do célebre movimento
lítero"-filosófico intitulado \textit{Tempestade e ímpeto}
[\textit{Sturm und Drang}], influenciando diretamente seus demais
paladinos, em especial Johann Gottfried von Herder. O trecho citado
por Nietzsche pertence à introdução de \textit{Memoráveis
acontecimentos socráticos}, cuja continuação diz: ``Se tivéssemos mais
do que aquilo que o tempo quis nos presentear, então nós mesmos
seríamos forçados a lançar nossos carregamentos ao mar ou pôr fogo em
nossas bibliotecas'' (Hamann, Johann Georg. \textit{Sokratische
Denkwürdigkeiten}. Ditzingen, Reclam, 1993, p.~7).} No
nosso caso, caberia ainda acrescentar o fato de que não nos seria mais
necessária a transmissão de nenhuma outra palavra, anedota ou data além
daquelas que já nos foram transmitidas, e, inclusive, que poderia ter sido
preservado bem menos para estabelecer a doutrina geral com
que os gregos justificam a filosofia. Uma época que sofre da assim
chamada formação geral [\textit{allgemeine Bildung}], mas que se acha
desprovida de cultura e não possui qualquer unidade de estilo em sua
vida,\footnote{ Cabe aqui lembrar que, para Nietzsche, cultura não
significa simplesmente erudição; obra artística orgânica e
coletiva, ela se refere, de maneira bem diferente, ao conjunto de todas
as determinações da experiência humana consideradas, nos termos do
filosofar nietzschiano, como produto de uma determinada atividade
vital. O filósofo alemão reedita, aqui, não por acaso, a noção de
cultura contida na célebre passagem da \textit{Primeira consideração
extemporânea} -- que, aliás, é contemporânea ao escrito \textit{A filosofia na
era trágica dos gregos} --, onde o termo é definido como ``a unidade de
estilo em todas as manifestações vitais de um povo'' (Nietzsche,
Friedrich. \textit{Unzeitgemässe Betrachtungen.} \textit{Erstes Stück:
David Strauss, der Bekenner und der Schriftsteller}. In:
\textit{Sämtliche Werke}. \textit{Kritische Studienausgabe}. Edição
organizada por Giorgio Colli e Mazzino Montinari. Berlim/Nova York,
Walter de Gruyter, 1999, vol.~\textsc{i}, § 1, p.~163).} não saberá
proceder à filosofia a contento e muito menos dizer se esta última
teria sido proclamada pelo próprio gênio da verdade nas ruas e nos
mercados. Em tal época, a filosofia permanece, antes do mais, o
monólogo erudito do caminhante solitário, a presa apoderada pelo
indivíduo, uma espécie de acobertado segredo de armário, ou, então, o
bate"-papo inofensivo entre idosos acadêmicos e crianças. Ninguém deve
ter a ousadia de dar cumprimento em si à lei da filosofia, ninguém mais
vive filosoficamente com aquela simples fé humana que impelia um
antigo, seja lá onde ele estivesse, seja lá o que ele fizesse, a se
comportar como estoico, caso alguma vez houvesse feito profissão de fé
ao \textit{Stoa}. Sob a aparência erudita, todo moderno
filosofar é limitado politica e policialmente por governos, igrejas,
academias, costumes, modas e pela covardia humana: não vai além do
suspiro que exprime ``se fosse assim'', ou, então, do conhecimento que
diz ``era uma vez''. A filosofia acha"-se privada de direito e, por isso,
deveria ser abolida pelo homem moderno, caso ele fosse de algum modo
corajoso e probo, banindo"-a com palavras semelhantes àquelas que Platão
empregou para abolir o poeta trágico de sua república. É claro que lhe
restaria uma contra"-reação, tal como também restou aos poetas trágicos
contra Platão. Se alguma vez alguém a obrigasse a falar, ela bem que
poderia dizer algo como: ``Povo pobre de espírito! É então minha culpa
se, entre vós, eu perambulo pela terra qual uma cartomante, devendo
esconder"-me e disfarçar"-me como se fosse a pecadora e vós meus juízes?
Vede então minha irmã, a arte! Vale para ela o que vale para mim, já
que fomos extraviadas junto com os bárbaros e já não sabemos como nos
salvar. Falta"-nos aqui, é bem verdade, qualquer direito favorável; mas
os juízes frente aos quais nos é concedido direito, também vos julgarão
e lhes dirão: tendes primeiro uma cultura e, aí então, deveis descobrir
o que pode e quer a filosofia''.

\sectionitem

A filosofia grega parece ter início com uma ideia inconsistente, com a
sentença de que a água é a origem e como que o útero materno de todas
as coisas: é mesmo necessário deter"-se aí com calma e tomar isso a
sério? Sim, e por três motivos: primeiro, porque a sentença enuncia
algo a respeito da origem das coisas, e, segundo, porque ela o faz sem
imagem e fabulação; e, por fim, em terceiro lugar, porque nela está
contido, ainda que em estado embrionário, o seguinte pensamento: tudo é
um. O primeiro motivo mencionado ainda deixa Tales em companhia dos
religiosos e supersticiosos, mas já o segundo retira"-o de tal companhia
e se nos apresenta como investigador da natureza; todavia, em função
do terceiro motivo, Tales torna"-se o primeiro filósofo grego. Se ele
tivesse dito: ``da água fez"-se a terra'', teríamos, aí então, apenas uma
hipótese científica, uma suposição falsa, mas, ainda assim, de difícil
refutação. No entanto, ele se dirigiu para além do âmbito científico. Na
exposição dessa representação da unidade mediante a hipótese da água,
Tales não superou o nível elementar das concepções físicas de sua
época, mas, no máximo, saltou por cima delas. As exíguas e
desordenadas observações de ordem empírica que Tales havia feito acerca
da ocorrência e das transmutações da água, ou, melhor ainda, acerca da
umidade, teriam ao menos permitido ou mesmo sugerido uma imensa
generalização; aquilo que o impeliu a esta última foi um dogma	\label{foiumdogma}
metafísico que se origina numa intuição mística e que, juntamente com
as tentativas sempre renovadas de expressá"-lo mais e melhor,
encontramos em todas as filosofias a sentença: ``tudo é um''.

É curioso o modo truculento por meio do qual tal crença trata o inteiro
mundo empírico: precisamente com Tales pode"-se aprender como a
filosofia procedeu, em todos os tempos, quando quis ir além de seus
objetivos magicamente atraentes, transpondo os umbrais da experiência.
Aos saltos, ela avança sobre apoios leves: esperança e pressentimento
dão asas a seus pés. De modo canhestro, o entendimento calculante
segue"-lhe ofegante por detrás e busca melhores apoios a fim de atingir,
também ele, os objetivos sedutores que sua companheira, mais divina, já
havia alcançado. Vem à mente, aqui, a imagem de dois andarilhos num
tempestuoso rio silvestre que revira as pedras em seu curso: um deles,
servindo"-se das pedras, salta com pés ligeiros e, balançando"-se sobre
elas, segue mais e mais, ainda que, com isso, terminem por afundar nas
profundezas às suas costas. O outro permanece, ali, inconsolado em todos
os momentos, tendo antes que erigir fundamentos capazes de suportar
seus passos pesados e calculados, sendo que, por vezes, não lhe é dado
caminhar, não havendo, pois, nenhum deus que lhe ajude a seguir sobre o
rio. O que conduz, então, o pensamento filosófico tão rapidamente a seu
objetivo? Então ele se distingue do pensamento calculante e medidor
apenas pelo fato de voar mais rápido ao longo de amplos espaços? Não, pois
é um poder estranho e ilógico que ergue seus pés, a saber, a fantasia.
Elevado por ela, o pensamento filosófico segue saltando de
possibilidade em possibilidade, que, nesse meio tempo, são admitidas
como certezas: ele mesmo apanha, aqui e acolá, certezas que estão a \label{certezasque}
voar. Um pressentimento genial indica"-lhe tais certezas e, à distância,
ele como que adivinha tratar"-se, nesse ponto, de certezas
demonstráveis. O poder da fantasia é, porém, particularmente poderoso
no que tange à apreensão relampejantemente instantânea, bem como à
elucidação de semelhanças: mais tarde, a reflexão traz à baila suas
medidas e seus moldes, procurando substituir as semelhanças por	\label{medidaseseusmoldes}
identidades e o que se vê lado a lado por causalidades. Mas, ainda que
isso jamais pudesse ser possível, o indemonstrável filosofar, mesmo no
caso de Tales, ainda tem um valor; mesmo que todos os suportes se
quebrem, quando a lógica e a dureza próprias do âmbito empírico
visarem a sentença ``tudo é água'', sempre há, após a demolição do
edifício científico, algo que remanesce; e é justamente nesse resto que
se encontra uma força movente e, por assim dizer, a esperança de uma fertilidade futura.

É claro que, com isso, não pretendo dizer que o pensamento talvez ainda
retenha consigo, sob a forma de alguma restrição ou enfraquecimento,
ou, então, como alegoria, algum tipo de ``verdade'': como se, ao
idealizarmos um artista plástico junto à cachoeira, vendo, nas formas
que se lhes espirram, um jogo d'água artisticamente afigurador de
corpos humanos e animais, máscaras, plantas, penhascos, ninfas e
idosos, contendo praticamente todos os tipos existentes, pudéssemos
então dizer que, no caso dele, a proposição ``tudo é água'' estaria
confirmada. O valor do pensamento de Tales -- mesmo depois do		\label{pensamentodetales}
conhecimento de que é indemonstrável -- está, antes, no fato de ele não
ter sido almejado mítica e alegoricamente. Os gregos, dentre os quais
Tales se tornou notável tão rapidamente, eram os antípodas de todos os
realistas, na medida em que acreditavam, em rigor, apenas na realidade
dos homens e dos deuses, concebendo, de resto, a natureza somente como
um disfarce da mascarada e da metamorfose de tais homens"-deuses. Para
eles, o homem era como que a verdade e o coração das coisas, sendo todo
o restante tão"-só aparência e jogo enganoso. Precisamente por isso lhes
era incrivelmente árduo apreender o conceito como conceito: ao passo \label{incrivelmentearduo}
que, inversamente, tal como os modernos sublimam o que há de mais
pessoal sob a forma de abstrações, neles o que havia de mais abstrato
recaía sempre numa pessoa. Tales, no entanto, disse: ``não o homem,
senão a água é a realidade das coisas'', de sorte que passou a
acreditar na natureza, na medida em que, ao menos, acreditou na água.
Enquanto matemático e astrônomo, tornou"-se frio diante de tudo o que
havia de mítico e alegórico, e se não logrou desiludir"-se a ponto de
atingir a pura abstração ``tudo é um'', permanecendo, de resto, junto a
uma expressão física, ainda assim ele constituiu uma surpreendente
raridade para os gregos de sua época. Altamente singulares, os órficos
talvez possuíssem a capacidade de conceber abstrações, bem como a de
pensar sem imagens, num nível ainda mais elevado que o dele: apenas com
a ressalva de que lhes era dado expressá"-las unicamente sob a forma de
alegoria. Também Ferécides de Siros,\footnote{ Tio materno de
Pitágoras, Ferécides de Siros viveu por volta do século \textsc{vi} a.C.~e foi
considerado um dos Sete Sábios da Grécia.} contemporâneo de Tales
e co"-participante de algumas concepções físicas, dá"-lhes uma expressão
titubeante, situando"-se naquela zona intermediária na qual o mito se
une à alegoria: assim é, por exemplo, que ele ousa comparar a terra a
um carvalho alado cujas asas se estendem no ar e que Zeus, após
derrotar Cronos, cobre com um esplêndido e honorável manto, e sobre
o qual, com suas próprias mãos, borda as terras, as águas e os rios.
Comparado a esse filosofar obscuro"-alegórico, que mal se deixa traduzir
em imagens, Tales coloca"-se como um mestre mais criativo e que, sem
recorrer a fabulações fantásticas, começa a olhar para a natureza em
suas profundezas. Se, nisso, ele se valeu da ciência e daquilo que é
demonstrável para, logo em seguida, ultrapassá"-los, isso constitui, em
todo caso, uma característica típica da mente filosófica. A palavra
grega que designa o ``sábio'' pertence etimologicamente a \textit{sapio},
``eu degusto'', \textit{sapiens}, ``aquele que degusta'',
\textit{sisyphos}, ``o homem com o mais apurado gosto''; de acordo com a
consciência do povo, a arte peculiar do filósofo consiste, pois, num
apurado discernir e conhecer, num relevante diferenciar. Ele não é
astuto, se por astuto entendermos, aqui, alguém que tira proveito de
seus próprios assuntos; com razão, Aristóteles assevera: ``aquilo que
Tales e Anaxágoras sabem será denominado incomum, impactante, difícil e
divino, mas inútil, pois, para eles, isso não tinha a ver com os bens
humanos''.\footnote{ Nietzsche refere"-se, aqui, a uma passagem contida
no livro \textsc{vi} da \textit{Ética a Nicômaco} de Aristóteles. Ali, o
filósofo Estagirita escreve: ``Por isso dizemos que Anaxágoras, Tales e
os homens semelhantes a eles possuem sabedoria filosófica, mas não
prática, quando os vemos ignorar o que lhes é vantajoso; e também
dizemos que eles conhecem coisas notáveis, admiráveis, difíceis e
divinas, mas improfícuas. Isso, porque não são os bens humanos que eles
procuram'' (Aristóteles. \textit{Ética a Nicômaco}. Tradução de Leonel
Vallandro e Gerd Bornheim. In: ``Os Pensadores''. São Paulo, Abril
Cultural, 1984, vol.~\textsc{ii}, 1141b5, p.~146).} Por meio dessa
seleção e separação do que é incomum, impactante, difícil e divino, a
filosofia demarca para si o limite que a separa da ciência, da mesma
maneira que, mediante a ênfase do inútil, ela se separa da astúcia. Sem
lançar mão de tal seleção, de tal bom gosto, a ciência debruça"-se sobre
tudo que é passível de ser conhecido, pretendendo, com cega avidez,
conhecer tudo a qualquer custo; o pensar filosófico, ao contrário,
põe"-se sempre a caminho das coisas que são mais dignas de serem
conhecidas, dos grandes e relevantes conhecimentos. Mas o conceito de
grandeza é mutável, tanto no âmbito moral quanto no estético: desse
modo, a filosofia começa com uma legislação acerca da grandeza, de
sorte que a ela se acha intimamente ligada a uma atividade de nomeação
[\textit{Namengeben}]. ``Isto é grande'', diz ela, e, com isso, eleva o
homem sobre a cega e incontida ânsia de seu impulso ao conhecimento.
Por meio do conceito de grandeza, ela amansa esse impulso: sobretudo
por considerar alcançável e alcançado o maior de todos os
conhecimentos, a saber, aquele pertinente à essência e ao núcleo das
coisas. Quando Tales diz ``tudo é água'', o homem então estremece e
eleva"-se para além do tatear e rastejar vermiculares das ciências
particulares, pressentindo a derradeira solução das coisas e superando,
mediante tal pressentimento, o viés comum dos níveis inferiores de
conhecimento. O filósofo procura ecoar em si a sonoridade total do
mundo para, aí então, exteriorizá"-la em conceitos: na medida em que é
contemplativo como o artista plástico e compassivo como o religioso,
perscrutando fins e causalidades qual o homem de ciência, sentindo"-se
inflado a ponto de atingir o macrocosmo, ele retém ainda a sobriedade
de se considerar friamente como o reflexo do mundo, isto é, aquela
sobriedade que possui o artista dramático quando se transmuda em outros
corpos, falando a partir deles e, a despeito disso, sabendo projetar
essa transmutação para fora em versos escritos. Aqui, aquilo que o
verso é para o poeta, é o pensar dialético para o filósofo: apanha"-o a
fim de agarrar"-se em seu enfeitiçamento e petrificá"-lo. E, assim como a
palavra e o verso são, para o dramaturgo, tão"-só o balbucio numa língua
estrangeira com vistas à expressão daquilo que ele viu e viveu, assim
também a expressão daquela profunda intuição filosófica mediante
dialética e reflexão científica é, por um lado, o único meio de
comunicar aquilo que se contemplou, mas, por outro, um meio pobre,
consistindo, no fundo, numa transposição metafórica cabalmente
enganadora para uma esfera e língua distintas. Foi assim que Tales
contemplou a unidade daquilo que existe: e, como quis comunicar"-se,
terminou por falar sobre a água!

\sectionitem

Enquanto o tipo universal de filósofo deixa"-se sublinhar apenas
nebulosamente a partir da imagem de Tales, a imagem de seu grande
sucessor nos fala, em contrapartida, de modo bem mais claro.
Anaximandro de Mileto, o primeiro escritor filosófico da	\label{primeiroescritor}
Antiguidade, escreve tal como o típico filósofo irá justamente
escrever, conquanto não lhe sejam furtadas, por exigências estranhas, a
imparcialidade e a ingenuidade: numa escrita cinzelada e altamente
estilizada, cada sentença é como que testemunha de uma nova iluminação,
expressando o ato de permanecer em sublimes contemplações. O pensamento
e sua forma são pedras norteadoras no caminho rumo àquela mais alta
sabedoria. Com tal perspicácia lapidar, Anaximandro diz uma vez: ``Lá	\label{laondeascoisas}
onde as coisas têm seu surgimento, para lá também devem ir ao fundo,
segundo a necessidade; pois têm de pagar expiação e ser justiçadas por
suas injustiças, conforme a ordem do tempo''.\footnote{ A fim de evitar
uma inflexão muito radical no horizonte hermenêutico condizente com as
ponderações de Nietzsche, passaremos ao largo das outras possíveis
versões dos fragmentos pré"-socráticos e de sua respectiva doxografia,
já de si complexa e especializada. Contudo, de acordo com a
possibilidade, tratamos de trazer à baila outras traduções. Eis, por
exemplo, o lapidar fragmento de Anaximandro -- na tradução de José
Cavalcante de Souza: ``(Em discurso direto): [\ldots] Princípio dos seres\ldots\
ele disse (que era) o ilimitado\ldots Pois donde a geração é para os
seres, é para onde também a corrupção se gera segundo o necessário;
pois concedem eles mesmos justiça e deferência uns aos outros pela
injustiça, segundo a ordenação do termo'' (Cf.~\textit{Os
pré"-socráticos}. In: ``Os Pensadores''. São Paulo, Abril Cultural, 1973,
p.~22).} Sentença enigmática de um verdadeiro pessimista ou
divisa oracular sobre a pedra angular da filosofia grega, como iremos
interpretá"-la?

Nos \textit{Parerga}, vol.~\textsc{ii}, p.~327, o único moralista sério de nosso
século nos instila na alma uma consideração semelhante: ``O critério
acertado para julgar todo homem consiste em considerá"-lo justamente
como um ser que não deveria de modo algum existir, mas que expia sua
existência por meio de toda sorte de sofrimento e da morte: o que se
pode esperar de tal ser? Não somos todos, afinal de contas, pecadores
condenados à morte? Expiamos nosso nascimento, primeiramente, por meio
da vida e, depois, por meio da morte'' (P.~\textsc{ii}, 22).\footnote{ A versão
inalterada e original dessa passagem dos \textit{Parerga}, portentosa
coleção de ensaios e aforismos que Schopenhauer dá a conhecer em 1850,
está contida, a título de suplemento à doutrina do sofrimento do mundo,
no capítulo 12 do segundo volume da obra. A esse respeito, cf.
Schopenhauer, Arthur. \textit{Parerga und Paralipomena}. Frankfurt am
Main, Suhrkamp, 1986, vol.~\textsc{ii}, cap.~12,  § 156a, p.~359.} Quem lê
essa doutrina a partir da fisiognomia de nosso universal destino humano
e reconhece a má constituição fundamental de toda vida humana no fato
de que ninguém suporta ser observado em detalhes ou da maneira mais
próxima possível -- ainda que nossa época, acostumada à epidemia
biográfica, pareça pensar de outro modo, sendo, de resto, mais
condescendente com a dignidade do homem; quem, como Schopenhauer,
escutou das ``alturas dos ares hindus'' a palavra sagrada acerca do valor
moral da existência não conseguirá, pois, abster"-se de fazer uma
metáfora extremamente antropomórfica, bem como de retirar aquela
doutrina melancólica dos limites da vida humana para, mediante
transposição, aplicá"-la ao caráter universal de tudo o que existe. Pode
não ser lógico, mas, em todo caso, é algo bem humano, e, ademais,
coaduna"-se bem com o estilo do salto filosófico anteriormente descrito,
reconhecer, agora, com Anaximandro, todo vir"-a"-ser como uma emancipação \label{sereterno}
do ser eterno digna de punição, isto é, como algo injusto que deve ser
expiado com o declínio. Tudo o que já veio a ser se perderá uma vez
mais, quer pensemos, aqui, na vida humana, quer pensemos na água, no
calor, ou, então, no frio: em toda parte, onde propriedades
determinadas puderem ser percebidas, podemos profetizar, conforme uma
monstruosa prova experimental [\textit{Erfahrungs"-Beweis}], o declínio
de tais propriedades. Assim, um ser que possui propriedades
determinadas, e que nelas consiste, jamais poderá ser origem e princípio
das coisas; o que existe verdadeiramente não pode, concluiu
Anaximandro, possuir quaisquer propriedades determinadas, pois, do
contrário, teria surgido e, como todas as outras coisas, teria de
declinar. Para que o vir"-a"-ser não deixe de existir, o ser primordial
tem de ser indeterminado. A imortalidade e eternidade do ser primordial
assentam"-se, não numa infinitude e inesgotabilidade -- tal como, em
geral, supõem os intérpretes de Anaximandro --, mas em não possuir as
qualidades determinadas que conduzem ao declínio: eis porque ele também
carrega o nome de ``o indeterminado''. O ser primordial assim denominado
eleva"-se sobre o vir"-a"-ser e, justamente por isso, assegura a
eternidade, assim como o constante curso do vir"-a"-ser. Essa última \label{essaultimaunidade}
unidade naquele ``indeterminado'', ventre materno de todas as coisas, só
pode, com efeito, ser descrita negativamente pelo homem, isto é, como
algo a que não pode ser concedido qualquer predicado advindo do mundo
existente do vir"-a"-ser e que, devido a isso, poderia valer como algo
semelhante à ``coisa em si'' kantiana.

Por certo, aquele que for capaz de discutir com outros a propósito do
que teria sido, em rigor, uma tal matéria primordial, se se tratava,
por assim dizer, de algo intermediário entre o ar e a água ou, talvez,
entre o ar e o fogo, não entendeu, nem de longe, o nosso filósofo:  o
mesmo deve ser dito àqueles que seriamente se perguntaram se
Anaximandro teria pensado sua matéria primordial como mistura de todas
as matérias existentes. Temos de voltar nosso olhar, antes do mais,
justamente para lá, onde possamos aprender que Anaximandro já não mais
tratava a pergunta pela procedência deste mundo em termos
exclusivamente físicos, dirigindo"-nos àquela sentença lapidar
introduzida anteriormente. Se, na pluralidade das coisas surgidas, ele
viu, antes, uma soma de injustiças de coisas a serem expiadas, é porque
foi o primeiro grego a apanhar com pulso firme o novelo do mais
profundo problema ético. Como pode perecer algo que possui direito a
existir! De onde parte aquele incansável vir"-a"-ser e nascer, de onde
vem aquela expressão de dolorida contorção sobre a face da natureza, de
onde surge o interminável lamento de morte em todos os âmbitos da
existência? Desse mundo próprio àquilo que é injusto, do impertinente
decaimento da unidade primordial das coisas, Anaximandro pôs"-se em fuga
rumo a uma pequena cidade metafísica, sobre a qual, debruçado, ele
agora deixa seu olhar rolar na amplidão para, ao fim e ao cabo, após
meditativo silêncio, dirigir a seguinte pergunta a todos os seres: o
que vale vossa existência? E, se ela de nada vale, então com qual \label{oquevale}
finalidade estais aí? Por vossa culpa, pelo que observo, permaneceis
nesta existência. Tereis de expiá"-la com a morte. Vede como vossa
terra se empalidece; os mares diminuem e secam, sendo que a concha
sobre a montanha vos dá indícios de quanto já secaram; o fogo já
destrói vosso mundo, que, por fim, terminará por se absorver em vapor e
fumaça. Mas, de novo, tal mundo da transitoriedade sempre tornará a
reconstruir"-se: quem poderia redimir"-vos da maldição do vir"-a"-ser?

Não foi qualquer tipo de vida que se tornou bem"-vinda ao homem que 
colocou tal pergunta, cujo pensar esvoaçante rompe continuamente os
cordões do empírico a fim de empreender, de imediato, o mais elevado voo
supralunar. Sem objeções, acreditamos na tradição segundo a qual ele \label{acreditamosnatradicao}
caminhava aqui e acolá com trajes particularmente dignos, demonstrando
um orgulho verdadeiramente trágico em seus gestos e hábitos de vida.
Viveu tal como escreveu; falava tão livremente quanto se vestia, erguia
a mão e apoiava o pé como se esta existência fosse uma tragédia na qual
nasceu como herói para participar dela. Nisso tudo, ele
foi o grande modelo de Empédocles. Seus concidadãos escolheram"-no para
dirigir uma colônia de imigrantes -- talvez se regozijassem em honrá"-lo
e, simultaneamente, em se verem livres deles. Seu pensamento também partiu
e fundou colônias: em Éfeso e Eleia não se podia livrar"-se dele, mas, 
quando alguém não era capaz de se decidir por permanecer na mesma
posição em que ele se encontrava, sabia"-se contudo que se tinha 
sido conduzido até ali por ele, a partir de onde se podia agora, 
sem ele, prosseguir adiante.

Tales mostra a necessidade de simplificar o âmbito da multiplicidade e
de reduzi"-lo a um mero desdobramento ou dissimulação de \textit{uma}
única qualidade existente, a saber, a água. Anaximandro ultrapassa"-o
mediante dois passos. Primeiramente, ele se pergunta: se há, de fato,
uma unidade eterna, como é então possível aquela multiplicidade? E
obtém a resposta a partir do caráter contraditório, autocorrosivo e
negador de tal multiplicidade. A própria existência desta última
torna"-se, para ele, um fenômeno moral, de sorte que não é justificada,
mas antes se expia continuamente a si mesma por meio do declínio. Mas
eis que então lhe ocorre a pergunta: por que tudo o que veio a ser
ainda não declinou há muito tempo, já que, com efeito, toda uma
eternidade de tempo já transcorreu? De onde vem o rio sempre restaurado
do vir"-a"-ser? Ele consegue salvar"-se dessa pergunta somente por meio de
possibilidades místicas: o eterno vir"-a"-ser só pode ter sua origem no
ser eterno, sendo que as condições do decaimento de tal ser em direção
a um vir"-a"-ser na injustiça são sempre as mesmas, a constelação das
coisas é criada de tal modo que já não se pode prever nenhum fim para
aquele retirar"-se do ser individual para fora do âmago do
``indeterminado''. Anaximandro deteve"-se aqui: quer dizer, ele permaneceu
em meio às profundas nuvens que, como gigantescos fantasmas, se
assentavam sobre as montanhas de tal visão de mundo. Quanto mais se
quis aproximar"-se do problema de como pode, em linhas gerais, por meio
do declínio, o determinado surgir do indeterminado, o temporal do
eterno, o injusto do justo, tanto maior se tornou a noite.

\sectionitem

No meio dessa noite mística na qual se achava encoberto o problema do \label{noitemistica}
vir"-a"-ser de Anaximandro, adentrou Heráclito de Éfeso e
iluminou"-a por meio de um relâmpago divino. ``Contemplo o vir"-a"-ser'',
clama ele, ``e ninguém enxergou tão atentamente esse eterno bater de
ondas e ritmo das coisas. E o que foi que vi? Regularidades, certezas
indefectíveis, caminhos sempre iguais do que é justo, Eríneas
judicantes por detrás de todas as transgressões das leis, o inteiro
mundo como o espetáculo de uma justiça dominante e de forças naturais,
presentes demoniacamente em todas as partes, submetidas a seu serviço.
O que vi não foi a punição daquilo que veio a ser, mas a
justificação do vir"-a"-ser. Quando se desvelou a iniquidade, o
decaimento, sob formas invioláveis, em leis divinamente observadas? Lá
onde impera a injustiça, existe arbítrio, desordem, desregramento,
contradição; mas, lá onde reina tão"-só a lei e a filha de Zeus, a
\textit{Dik\=e}, tal como ocorre neste mundo, como poderia ser, aí, a
esfera da culpa, da expiação, da condenação e, por assim dizer, a sala
de execução de todos os condenados?''

Dessa intuição Heráclito retirou duas negações interdependentes, que são
trazidas à plena luz apenas por meio da comparação com as proposições
de seus antecessores. Ele denegou, primeiramente, a dualidade de mundos
inteiramente distintos, cuja suposição Anaximandro havia sido obrigado
a fazer; já não distinguia um mundo físico de um mundo metafísico, um
âmbito de qualidades determinadas de um âmbito de indefinível
indeterminação. Agora, depois desse primeiro passo, ele também já não
podia mais ser impedido de empreender uma audácia bem maior da negação:
ele denegou, em linhas gerais, o ser. Pois esse único mundo que lhe \label{esseunicomundo}
sobrou -- escudado ao seu redor por leis eternas não escritas, fluindo
de cima a baixo conforme a brônzea batida do ritmo -- não mostra, em
nenhum lugar, uma persistência, uma indestrutibilidade, um lugar seguro na correnteza. 
Ainda mais alto do que Anaximandro, Heráclito exclamou: ``Nada vejo
senão o vir"-a"-ser. Não vos deixeis iludir! Se acreditais ver, em algum
lugar, terra firme no mar do vir"-a"-ser e do perecer, isso se deve à
vossa visão limitada, e não à essência das coisas. Utilizais nomes das
coisas como se estas tivessem uma duração rígida: mas a própria correnteza, na
qual entrais pela segunda vez, já não é mais a mesma que a da primeira vez''.

Como sua propriedade magnífica, Heráclito possui a mais elevada força da
representação intuitiva; ao passo que, no que tange ao outro tipo de
representação, que se consuma em conceitos e combinações lógicas, quer
dizer, no que diz respeito à razão, ele se mostra frio, insensível,
inclusive hostil, sendo que parece obter um certo prazer quando
consegue contradizê"-la mediante uma verdade intuitivamente alcançada: e
isso ele faz em sentenças tais como, por exemplo, ``tudo possui sempre o
contrário em si'', mas de modo tão intrépido que Aristóteles lhe imputa,
diante do tribunal da razão, o mais alto delito, a saber, o de ter pecado
contra o princípio de contradição.\footnote{ Nietzsche alude, aqui, ao
seguinte trecho da \textit{Metafísica} de Aristóteles: ``É impossível
que o mesmo seja atribuído e não seja atribuído ao mesmo tempo a um
mesmo [subjacente] e conforme o mesmo aspecto (e esteja delimitado em
acréscimo tudo aquilo que acrescentaríamos contra as dificuldades
dialéticas); ora, este é o mais firme de todos os princípios; pois ele
comporta a definição mencionada. Pois é impossível que quem quer que
seja considere que um mesmo [fato] é e não é -- tal como alguns julgam
que Heráclito afirmava'' (Cf.~Aristóteles. \textit{Metafísica}. Tradução de
Lucas Angioni. In: ``Textos Didáticos'' . Campinas, \textsc{ifch}/Unicamp, 2001,
livro \textsc{iv}, 1005b17, p.~35).} A representação intuitiva abrange,
porém, duas instâncias: primeiramente, o mundo atual, mutável,
variegado, que nos pressiona em todas as experiências, e, em
seguida, as condições por meio das quais se torna possível, antes de
tudo, toda e qualquer experiência deste mundo, isto é, tempo e espaço.
Pois, ainda que destituídos de conteúdos determinados, estes últimos
podem ser intuitivamente percebidos, e, a ser assim, contemplados em si
mesmos pura e independentemente de qualquer experiência. Se Heráclito
encara o tempo dessa maneira, dissociado de todas as experiências,
então tinha, nele, o mais instrutivo monograma de tudo aquilo que, de
modo geral, ocorre no âmbito da representação intuitiva. Assim como ele
discerniu o tempo, assim também o discerniu, por exemplo, Schopenhauer,
que, em repetição, assevera a seu respeito: que, nele, todo instante
existe tão"-só à medida que devorou o instante precedente, seu pai, a
fim de ser ele próprio devorado uma vez mais de modo igualmente rápido;
que passado e futuro são tão nulos quanto qualquer sonho, sendo o
presente apenas o limite inconstante e inextenso entre ambos; mas que,
como o tempo, também o espaço e, assim como este, tudo aquilo que se
acha simultaneamente no tempo e no espaço possui apenas uma existência
relativa, existindo apenas por meio e em função de alguma outra coisa
que se lhe assemelha, isto é, uma coisa que ela mesma apenas existe do mesmo
modo. Essa é uma verdade da mais elevada intuição, imediata e
acessível a qualquer um, e, justamente por isso, muito difícil de ser
atingida conceitual e racionalmente. Aquele que a tem diante dos olhos,
porém, tem de avançar imediatamente para a consequência heraclitiana e
dizer que a inteira essência da efetividade consiste, de fato, somente
no efetuar, sendo"-lhe denegado qualquer outro modo de ser; tal como, de
igual maneira, Schopenhauer a exibiu (\textit{O mundo como vontade e
representação} \textsc{i}, p.~10): ``apenas à medida que se efetua ela preenche o
espaço, preenche o tempo: sua atuação sobre o objeto imediato determina
a intuição sem a qual ela própria não existe; a consequência da atuação
de qualquer outro objeto material sobre algum outro objeto só se torna
discernível se, desta feita, este último atuar sobre o objeto imediato
de um modo diferente que outrora, apenas nisso.
Causa e efeito são, por conseguinte, a inteira essência da matéria: seu ser é
seu efetuar. Por isso, é extremamente bem"-vindo que, em alemão, o
epítome de tudo o que é material seja denominado efetividade
[\textit{Wirklichkeit}], palavra esta que é muito mais reveladora do
que realidade [\textit{Realität}]. Aquilo sobre o qual ela atua é, uma
vez mais, sempre matéria: seu inteiro ser e sua essência consistem,
pois, apenas na mudança regular que \textit{uma} de suas partes produz
sobre a outra, de sorte que se trata de algo totalmente relativo, de
acordo com uma relação que é válida somente no interior de seus
limites, portanto preciso como o tempo e o
espaço.''\footnote{ Cf.~Schopenhauer, Arthur. \textit{Die Welt als Wille
und Vorstellung}. Frankfurt am Main, Suhrkamp, 1986, vol.~\textsc{i}, 
livro \textsc{i}, § 4, p.~38.}

O eterno e único vir"-a"-ser, a inteira impermanência de tudo o que é
efetivo, que apenas atua e vem a ser continuamente, mas nunca é, tal
como ensina Heráclito, consiste numa representação assustadora e
atordoante, que, em sua influência, se aproxima ao máximo da sensação de
quem, num abalo sísmico, perde a confiança na terra bem firmada. Foi \label{abalosismico}
necessária uma força espantosa para transpor esse efeito no seu
contrário, no sublime e na feliz admiração. Heráclito logrou
isto por meio de uma observação acerca da própria procedência de todo
vir"-a"-ser e perecer, que ele compreende sob a forma da polaridade, como
o desmembramento de uma força em duas atividades opostas e
qualitativamente diferentes, mas que se esforçam por sua reunificação.
Uma qualidade aparta"-se continuamente de si mesma e separa"-se em seus
contrários: e, de novo, esses contrários se esforçam continuamente
um em direção ao outro. Com efeito, o povo acredita reconhecer algo
rígido, acabado e sólido; em verdade, em cada instante há luz e
escuridão, amargo e doce, um junto ao outro e presos entre si, como
dois lutadores dos quais ora um ora outro adquire a hegemonia.
Segundo Heráclito, o mel é simultaneamente doce e amargo, sendo que o
próprio mundo é uma vasilha que tem de ser permanentemente
agitada. Todo vir"-a"-ser surge da guerra dos opostos: as qualidades \label{guerradosopostos}
determinadas, que se nos aparecem como sendo duradouras, exprimem
tão"-só a prevalência momentânea de um dos combatentes, 
mas, com isso, a
guerra não chega a seu termo, porém a luta segue pela eternidade.
Tudo se dá de acordo com esse conflito, e é precisamente esse conflito
que revela a justiça eterna.\footnote{ Cf., a esse propósito, o
seguinte fragmento de Heráclito: ``É preciso saber que o combate é
o"-que"-é"-com, justiça (é) discórdia, e que todas (as coisas) vêm a ser
segundo discórdia e necessidade'' (\textit{Os pré"-socráticos}. Tradução
de José Cavalcante de Souza. In: ``Os Pensadores''. São Paulo, Abril
Cultural, 1973, p.~93).} Trata"-se de uma representação
fantástica, gerada a partir da mais pura nascente do helenismo, que
concebe o conflito como o contínuo reinado de uma justiça unitária,
severa e ligada a leis eternas. Apenas um grego estava apto a descobrir
essa representação como fundamento de uma cosmodiceia; trata"-se da boa
Éris de Hesíodo transfigurada em princípio cosmológico, o pensamento de
disputa do grego individual, bem como do estado grego, transposto dos
ginásios e das palestras, dos \textit{ag\=onoi} artísticos, da peleja dos partidos
políticos e das cidades entre si, para aquilo que há de mais universal,
de sorte que, agora, a engrenagem do cosmo move"-se nele. Assim como cada
grego luta como se apenas ele estivesse com a razão, e, a cada
instante, uma medida infinitamente segura da sentença judicial
determina para onde pende a vitória, assim pelejam as qualidades entre
si, conforme leis e medidas inabaláveis e imanentes à luta. As próprias
coisas, em cuja fixidez e permanência se fiam as restritas mentes humanas e
animal, não possuem, pois, em rigor, nenhuma existência; são como que o
brilho rápido e a faísca lampejante de espadas sacadas da bainha, são,
enfim, o esplendor da vitória na luta das qualidades opostas.

Essa luta própria a todo vir"-a"-ser, aquela eterna mudança da vitória,
descreve, uma vez mais, Schopenhauer (\textit{O mundo como vontade e
representação} \textsc{i}, p.~175): ``A matéria permanente precisa,
incessantemente, cambiar de forma, pois, segundo o fio condutor da
causalidade, fenômenos mecânicos, físicos, químicos e orgânicos
concorrem ardorosamente entre si para aparecer, abocanhando um do outro
a própria matéria, já que cada um deles deseja manifestar sua ideia.
Esse conflito pode ser acompanhado através de toda natureza, que,
inclusive, consiste justamente apenas nele''.\footnote{ Schopenhauer,
Arthur.~\textit{Die Welt als Wille und Vorstellung}. Frankfurt am Main,
Suhrkamp, 1986, vol.~\textsc{i}, livro \textsc{ii}, § 27, p.~218.} As páginas que se
seguem fornecem as mais notáveis ilustrações desse conflito: mas com a
ressalva de que o tom geral dessas descrições permanece sempre
diferente daquele presente em Heráclito, na medida que, para
Schopenhauer, a luta é uma prova da cisão interna da vontade de vida,
um devorar"-se"-a"-si"-mesmo desse impulso opressivo e sombrio, qual um
fenômeno completamente horrível e que de maneira alguma traz
felicidade. A arena e o objeto dessa luta é a matéria, que as forças
naturais procuram tomar para si em sua mútua relação, assim como espaço
e tempo, cuja união por meio da causalidade consiste justamente na matéria. 

\sectionitem

Enquanto a imaginação de Heráclito mensurava o universo incansavelmente
movido, a ``efetividade'', com o olho do feliz espectador que vê
inumeráveis pares combaterem numa alegre competição sob a égide de
rígidos árbitros, adveio"-lhe um pressentimento mais elevado; ele já não
podia tomar os pares beligerantes e os juízes como coisas separadas
entre si, haja vista que os próprios juízes pareciam lutar e os
próprios lutadores pareciam julgar a si mesmos -- e já que, no fundo,
percebeu apenas uma justiça eternamente dominante, ele ousou exclamar: 
``O próprio conflito do múltiplo é a única justiça! E, em geral: o um é o
múltiplo. Pois, o que são todas aquelas qualidades de acordo com a
essência? Deuses imortais? São essências separadas atuantes por
si desde o início e sem término? E se o mundo que vemos conhece apenas
vir"-a"-ser e perecer, mas nenhuma fixidez, aquelas qualidades deveriam,
quiçá, formar um outro tipo de mundo metafísico, que, com efeito, não
seria nenhum mundo da unidade, tal como o buscou Anaximandro por detrás
do tremulante véu da multiplicidade, mas antes um mundo de
multiplicidades eternas e essenciais?''. Mas será que, por um desvio,
Heráclito não reincidiu na dupla ordenação do mundo, malgrado a
virulência com a qual ele a negou, com um Olimpo de demônios e
numerosos deuses imortais -- a saber, \textit{muitas} realidades -- e com
um mundo de homens que vê apenas a nuvem de poeira da luta olímpica e o
esplendor das lanças divinas -- quer dizer, apenas um vir"-a"-ser?
Anaximandro pôs"-se em fuga justamente das qualidades determinadas rumo
ao seio do ``indeterminado''; já que tais
qualidades vinham a ser e pereciam, denegou"-lhes a existência
verdadeira e essencial; mas não deveria parecer, agora, que o vir"-a"-ser
é somente o vir"-a"-ser"-visível [\textit{das} \textit{Sichtbarwerden}] de
uma luta de qualidades eternas? Não se deveria remontar à própria
fragilidade do conhecimento humano, quando falamos sobre o vir"-a"-ser -- 
ao passo que, na essência das coisas, talvez não exista nenhum
vir"-a"-ser, mas tão"-só um lado a lado de muitas realidades verdadeiras e
indestrutíveis que não vieram a ser?

Esses são escapes e labirintos não heraclitianos; de novo, ele exclama:
``o um é o múltiplo''. As diversas qualidades
perceptíveis não são nem essencialidades eternas nem \textit{phantasmata} de 
nossos sentidos (mais tarde, Anaxágoras irá concebê"-las da primeira
maneira, e, deste último modo, Parmênides), não são nem ser fixo e
auto"-suficiente nem aparência fugidia a vagar nas mentes humanas. A
terceira e única possibilidade restante para Heráclito não poderá ser
adivinhada por ninguém mediante sagacidade dialética e, por assim
dizer, através de cálculo: pois, o que descobriu, aqui, é uma raridade
no próprio domínio das incredulidades místicas e das imprevistas
metáforas cósmicas. O mundo é o \textit{jogo} de Zeus, ou, então,
para falar fisicamente, do fogo consigo próprio, sendo que apenas nesse
sentido o um é simultaneamente o múltiplo. 

A fim de esclarecer, em primeiro lugar, a introdução do fogo como força
configuradora do mundo, lembro"-me da maneira pela qual Anaximandro
havia dado seguimento à teoria da água como origem das coisas.
Fiando"-se, no essencial, em Tales, inclusive fortalecendo e
multiplicando suas observações, Anaximandro não estava convencido,
contudo, de que não houvesse nenhum outro nível de qualidades anterior
e como que por detrás da água: mas antes, de maneira bem diferente,
parecia"-lhe que o próprio úmido se formava a partir do quente e do
frio, de sorte que estes últimos deveriam ser, portanto, níveis
preliminares da água, isto é, qualidades ainda mais originais. O
vir"-a"-ser tem início com a separação dessas qualidades próprias ao ser
primordial do ``indeterminado''. Heráclito,
que, como físico, se subordinou ao significado de Anaximandro, interpretou
para si esse calor anaximândrico como alento, respiração quente, vapores
secos, em suma, como elemento ígneo: acerca desse fogo enuncia, então,
a mesma coisa que Tales e Anaximandro haviam enunciado acerca da água,
que ele perfaz o caminho do vir"-a"-ser em inúmeras transmutações, e, em
especial, nos três principais estados, como algo quente, úmido e
sólido. Pois, ao descender, a água transforma"-se em terra e, ao
ascender, torna"-se fogo; ou, de modo mais preciso, como Heráclito
parece ter se expressado: do mar ascendem apenas os vapores puros, que
servem de alimento ao fogo celestial dos astros, da terra elevam"-se
apenas os vapores obscuros, nebulosos, dos quais o úmido retira seu
alimento. Os vapores puros são a transição do mar ao fogo, 
e os impuros a transição da terra à água. Assim seguem continuamente
os dois caminhos de transmutação do fogo, para cima e para baixo, de lá
para cá, lado a lado, do fogo à água, desta última à terra, da terra
novamente à água, e da água ao fogo.\footnote{ Cf., a esse respeito, o
seguinte fragmento heraclitiano: ``Direções do fogo: primeiro mar, e do
mar metade terra, metade incandescência\ldots\ Terra dilui"-se em mar e se
mede no mesmo \textit{logos}, tal qual era antes de se tornar terra'' (\textit{Os
pré"-socráticos}. Tradução de José Cavalcante de Souza. In: ``Os
Pensadores''. São Paulo, Abril Cultural, 1973, p.~88).}
Enquanto Heráclito é um partidário de Anaximandro naquilo que essas
representações contêm de mais importante, como, por exemplo, que o fogo
é conservado pela evaporação, ou, então, que a água se dissolve parte
em terra, parte em fogo, dele independe e com ele está em contradição
na medida em que elimina o frio do processo físico, ao passo que
Anaximandro o havia colocado ao lado da água de igual para igual, para
fazer surgir, de ambos, o úmido. Para Heráclito, fazer isso era, por
certo, uma necessidade: afinal, se tudo tem de ser fogo, então, em que
pese todas as possibilidades de sua transformação, não pode existir
nada que seja seu contrário absoluto; assim, ele terá interpretado
aquilo a que se nomeia frio somente como grau do quente e pôde
justificar essa interpretação sem dificuldades. Muito mais relevante,
porém, do que esse afastamento da doutrina de Anaximandro é uma
concordância ulterior: como este último, ele acredita num ocaso do mundo
que volta a ocorrer periodicamente e num emergir sempre renovado de um
outro mundo a partir do incêndio universal que a tudo aniquila. O
período no qual o mundo se apressa para defrontar"-se com tal incêndio
universal e com aquela dissolução no puro fogo é por ele caracterizado,
do modo mais impressionante, como um desejar e necessitar, um deixar"-se
engolir completamente pelo fogo como saciedade; e resta"-nos ainda a
pergunta pelo modo como ele compreendeu e denominou o novo e nascente
impulso de formação do mundo, o derramar"-se nas formas da
multiplicidade. O provérbio grego parece prestar"-nos ajuda com o
pensamento segundo o qual a ``saciedade engendra o crime
(a \textit{hybris})''; e, de fato, pode"-se por um instante
indagar se Heráclito extraiu, talvez, aquele retorno à multiplicidade
da \textit{hybris}. Que se leve uma vez a sério esse pensamento: à sua
plena luz transmuda"-se, diante de nossos olhares, a face de Heráclito,
o brilhar altivo de seus olhos se extingue, um franzimento de dolorosa
resignação e de impotência torna"-se pregnante, e parece que sabemos o
motivo pelo qual a Antiguidade mais tardia o nomeou o
``filósofo que chora''. Acaso não é, agora, o
inteiro processo do mundo um ato de punição da \textit{hybris}? A
multiplicidade o resultado de um ato criminoso? A transmutação do puro
no impuro uma decorrência da injustiça? Não é a culpa, agora,
transposta para o núcleo das coisas, e, com isso, o mundo do vir"-a"-ser
e dos indivíduos não se vê livre dela, achando"-se, ao mesmo
tempo, condenado a sempre retomar sobre si suas consequências?

\sectionitem

Aquela palavra temerária, \textit{hybris}, é, de fato, a pedra de toque
para todo heraclitiano; ele pode mostrar, aqui, se compreendeu ou
desconheceu seu mestre. Há, neste mundo, culpa, injustiça, contradição e sofrimento?

Sim, exclama Heráclito, mas apenas para o homem limitado, que vê as
coisas separadas umas das outras e não em conjunto, não para o deus
contuitivo [\textit{contuitiven} \textit{Gott}]; para ele, tudo o que
está em conflito converge numa harmonia, certamente invisível ao
olho humano comum, mas compreensível àquele que, como Heráclito, se
assemelha ao deus contemplativo. Diante de seu olhar ígneo, não permanece 
nenhuma gota de injustiça no mundo que se derramou ao seu redor; e até
mesmo aquele impulso cardeal que o fogo puro pode
insuflar em formas tão impuras, é por ele superado mediante uma 
sublime analogia. Neste mundo, um vir"-a"-ser e perecer, um erigir e
destruir, sem qualquer imputação moral e numa inocência eternamente
igual, possuem apenas o jogo do artista e da criança. E assim como jogam 
a criança e o artista, joga também o fogo eternamente
vivo, erigindo e destruindo, em inocência -- e, esse jogo, o
\textit{Aiôn} joga consigo próprio.\footnote{ A propósito da relação
entre \textit{Aiôn} (tempo), jogo e criança, cf.~o seguinte fragmento:
``Tempo é criança brincando, jogando; de criança o reinado'' (\textit{Os
pré"-socráticos}. Tradução de José Cavalcante de Souza. In: ``Os
Pensadores''. São Paulo, Abril Cultural, 1973, p.~90).}
Transmudando"-se em água e terra, ele ergue, como a criança, montes de
areia à beira do mar, edificando e destruindo; de tempos em tempos, dá
início ao jogo de novo. Um momento de saciedade: aí então, é de novo
tomado pela necessidade, tal como esta impele o artista ao ato de
criar. Não é o ímpeto criminoso, mas o impulso lúdico, sempre a
despertar uma vez mais, que exorta outros mundos à vida. Vez ou outra,
a criança joga fora o brinquedo: mas, de súbito, recomeça tudo com
humor inocente. Assim que constrói, porém, conecta, une e forma com
regularidade e de acordo com ordenações internas.

Apenas o homem estético contempla o mundo dessa maneira, aquele que
descobriu com o artista e com o surgimento da obra de arte como o
conflito da multiplicidade ainda pode trazer, em si, lei e direito,
como o artista, contemplativamente, põe"-se a operar sobre a obra de
arte, enfim, como necessidade e jogo, conflito e harmonia, devem
irmanar"-se com vistas à produção da obra de arte.

Quem ainda irá exigir de tal filosofia uma ética, com os necessários
imperativos ``tu deves'', ou ainda tratará de
imputar a Heráclito, censurando"-o, tal deficiência! O homem é, até sua
última fibra, necessidade e ``não livre'' do
começo ao fim -- se por liberdade entende"-se, aqui, a exigência
desvairada de poder mudar, como uma vestimenta, sua \textit{essentia}
a bel"-prazer, uma exigência que, até agora, toda filosofia séria
recusou com o devido escárnio. Que tão poucos homens vivam com
consciência no \textit{logos} e em acordo com o olho do artista que
tudo abarca, isso se deve ao fato de que suas almas são molhadas e que
os olhos e ouvidos dos homens, e, em geral, seu intelecto, são
testemunhos ruins quando ``lodo úmido toma suas
almas''. Não se pergunta por que razão isto se dá, tampouco se pergunta por que o fogo se torna água e terra.
Heráclito decerto não tem motivo algum para \textit{ter} de provar
(como tinha Leibniz) que este mundo é o melhor dentre todos,
basta"-lhe que o mundo seja o belo e inocente jogo do \textit{Aiôn}.
A seu ver, o homem equivale, em linhas gerais, a um ser
irracional: o que não conflita com o cumprimento, em toda sua essência,
da lei toda"-poderosa da razão. Ele não ocupa, em absoluto, uma posição
particularmente privilegiada na natureza, cujo fenômeno mais elevado
não é o homem prosaico, mas o fogo, por exemplo, enquanto astro. Na medida
em que o homem participa, por necessidade, do fogo,
então ele se torna algo mais racional; mas, na medida em que
consiste de água e terra, então algo vai mal com sua razão. Não existe
uma obrigação que o forçasse a conhecer o \textit{logos}, porque ele é
homem. Porém, por que há água, por que há terra? Este é, para
Heráclito, um problema bem mais sério do que a pergunta pela razão de
os homens serem tão tolos e ruins. Tanto no mais elevado dos homens
como no homem mais torpe revela"-se a mesma regularidade e justiça imanentes. 
Se, no entanto, alguém quisesse lançar a seguinte questão a
Heráclito: por que o fogo não permanece sempre fogo, por que ora é
água, ora é terra? Então ele simplesmente responderia:
``trata"-se de um jogo, não o considereis de modo tão
patético, e, antes de tudo, não o considereis
moralmente!''. Heráclito descreve somente o mundo
existente e adquire, nele, a satisfação contemplativa com a qual o
artista contempla sua obra que devém. Só aqueles que têm algum motivo
para não estar contentes com sua descrição natural do homem o
consideram sombrio, melancólico, prenhe de lágrimas, escuro, bilioso,
pessimista e, em geral, digno de ódio. Mas ele os tomaria, junto com
suas antipatias e simpatias, seu amor e ódio, por seres indiferentes,
prestando"-lhes ensinamentos tais como: ``Os cães latem a
todo aquele que desconhecem'', ou, então,
``para o asno, a palha é preferível ao
ouro''.\footnote{ Mencionado por Aristóteles no livro \textsc{ix}
da \textit{Ética a Nicômaco}: ``Cão, cavalo e homem têm prazeres
diferentes e, como diz Heráclito, `os asnos prefeririam as varreduras
ao ouro'; porque o alimento é mais agradável do que o ouro para eles''
(Aristóteles. \textit{Ética a Nicômaco}. Tradução de Leonel Vallandro e
Gerd Bornheim. In: ``Os Pensadores''. São Paulo, Abril Cultural, 1984,
vol.~\textsc{ii}, 1176a5, p.~226).}

Provêm também desses insatisfeitos as incontáveis queixas sobre a
obscuridade do estilo heraclitiano: provavelmente, nenhum homem jamais
escreveu de modo mais claro e iluminador. É certo que muito conciso, e,
portanto, obscuro para aqueles que costumam ler correndo. Mas,
como deveria um filósofo escrever, de propósito, sem clareza -- como
habitualmente se diz de Heráclito --, eis algo completamente
inexplicável: caso ele não tivesse nenhuma razão para esconder
pensamentos, ou, então, fosse suficientemente astucioso para ocultar,
sob palavras, sua ausência de pensamento. Como diz Schopenhauer, é
preciso precaver"-se contra possíveis mal"-entendidos por meio da
clareza, inclusive nas situações da habitual vida prática; como poderia
então alguém se expressar de modo impreciso, até mesmo enigmático, a
respeito do mais difícil, abstruso e quase inatingível objeto do
pensar, isto é, a propósito daquilo que é a tarefa mesma da filosofia?
Mas, no que tange à concisão, Jean Paul fornece uma boa lição:
``No todo, é apropriado que tudo o que é grande -- pleno de
sentido para aquele cuja sensibilidade é rara -- seja falado apenas
concisamente e (portanto) de modo obscuro, para que o espírito fútil
prefira, antes, explicá"-lo como algo sem sentido a traduzi"-lo em seu
próprio vazio de sentido. Pois, os espíritos comuns têm a medonha
capacidade de ver, nos dizeres mais ricos e profundos, única e
exclusivamente a sua opinião diária''. Ademais, e apesar
disso, Heráclito não escapou aos ``espíritos
fúteis''; os próprios estoicos já o haviam interpretado
superficialmente, limitando sua percepção estética fundamental acerca
do jogo do mundo à consideração ordinária das consequências do mundo, e isso, com
efeito, com vistas aos benefícios dos homens: de sorte que, em
tais mentes, a sua física converteu"-se num otimismo rude, que exige
continuamente de fulano e sicrano: \mbox{\textit{plaudite amici}.} 

\sectionitem

Heráclito era orgulhoso: e quando o que está em jogo, num filósofo, é o
orgulho, então se trata de um grande orgulho. Seu atuar nunca o remete a
um ``público'', ao aval das massas e ao ruidoso coro dos contemporâneos.
É da essência do filósofo trilhar a estrada sozinho. Seu talento é o
mais raro, e, num certo sentido, o mais inatural, sendo ele mesmo,
nisso, hostil e segregante para com os outros talentos afins. O muro de
sua auto"-suficiência [\textit{Selbstgenugsamkeit}] deve ser de
diamante, caso não deva ser demolido e quebrado, pois, afinal, tudo
se lança contra ele. Sua viagem rumo à imortalidade torna"-se mais difícil,
obstáculo a qualquer outra; e, no entanto, ninguém pode acreditar
mais firmemente do que o próprio filósofo em chegar à meta através dela
-- porque não sabe onde mais deve colocar"-se, a não ser sobre as asas
amplamente estendidas de todos os tempos; pois a desatenção para com o
hodierno e o momentâneo está na essência da grande natureza filosófica.
Ele tem a verdade: a roda do tempo pode rolar para onde bem entender,
mas jamais poderá escapar da verdade. É importante tomar conhecimento
de que tais homens já existiram. Nunca se poderia, por exemplo,
imaginar o orgulho de Heráclito como uma vã possibilidade. Segundo sua
essência, todo esforço pelo conhecimento parece ser, em si, eternamente
insaciado e insaciável. Por conta disso, ninguém poderá acreditar, caso
não esteja instruído pela história, numa consideração de si tão
majestosa e no convencimento de ser o único e feliz candidato à
verdade. Tais homens vivem em seu próprio sistema solar; é nele que se
deve procurá"-los. Também um Pitágoras e um Empédocles tratavam a si
mesmos com uma apreciação supra"-humana, sim, com uma reverência
quase religiosa; mas o laço da compaixão, atado à grande convicção
acerca da transmigração de almas e da unidade de todo ser vivente,
levou"-os novamente aos outros homens, à sua saúde e recuperação. Mas só
se pode pressentir, com assombro, alguma coisa a respeito do sentimento
de solidão que permeava o eremita efésio do templo de Ártemis no mais
selvagem deserto montanhoso. Dele não flui nenhum onipotente sentimento
de estímulos compassivos, nenhuma ânsia de querer socorrer, curar ou
recuperar. Trata"-se de um astro sem atmosfera. Voltado flamejantemente
para dentro, seu olho dirige"-se, apenas em aparência, funesta e
gelidamente para fora. À sua volta, imediatamente na viga mestra de seu
orgulho, batem as ondas da loucura e da absurdidade; com repugnância,
ele se afasta delas. Mas mesmo os homens de coração aberto evitam essa
máscara como que forjada em bronze; um tal ser pode vir à luz de modo
mais palpável num remoto santuário, em meio a imagens divinas e ao lado
de uma fria arquitetura de calma sublimidade. Entre homens, Heráclito
era, como homem, inacreditável; e, se por acaso, foi surpreendido ao
prestar atenção no jogo de crianças barulhentas, ainda assim, ao
deter"-se em tal jogo, ele refletiu sobre algo que um homem, em tal
oportunidade, jamais havia refletido: o jogo da grande criança do mundo
[\textit{des grossen Weltenkindes}], Zeus. Não carecia dos homens, nem
mesmo quanto aos seus conhecimentos; em tudo acerca daquilo que se
poderia indagar sobre eles e do que, antes dele, os outros sábios
haviam esforçado"-se para indagar, nada havia que lhe tocasse. Era com
apreciação desdenhosa que falava sobre tais homens questionadores,
colecionadores, em suma, ``históricos''. ``Busquei e investiguei a mim
mesmo'',\footnote{ Mais conciso, o fragmento de Heráclito afirma:
``Procurei"-me a mim mesmo'' (\textit{Os pré"-socráticos}. Tradução de José
Cavalcante de Souza. In: ``Os Pensadores''. São Paulo, Abril Cultural,
1973, p.~94).} disse a seu próprio respeito, com uma palavra por
meio da qual se indica o investigar de um oráculo: como se ele, e mais
ninguém, fosse o verdadeiro executor e realizador da divisa délfica
``Conhece"-te a ti mesmo''.

O que ele escutou todavia desse oráculo, reputou"-o sabedoria imortal e
eternamente digna de interpretação, de efeito ilimitado e a perder de
vista, conforme o modelo dos discursos proféticos da Sibila.\footnote{ Fiéis 
depositárias de divinas revelações, as Sibilas são descritas, na
mitologia greco"-romana, como lendárias profetisas. Sob a égide e a
inspiração de Apolo, a elas cabia assumir a perspectiva de uma
existência consagrada aos lampejos da sabedoria oracular,
comunicando"-se com a esfera que designa o divino e transmitindo, sob a
forma de profecias, suas enigmáticas mensagens. A sequência do texto de
Nietzsche procura fiar"-se, em boa medida, na imagem pregnante contida
no seguinte fragmento de Heráclito: ``E a Sibila com delirante boca sem
risos, sem belezas, sem perfumes ressoando mil anos ultrapassa com a
voz, pelo deus nela'' (\textit{Os pré"-socráticos}. Tradução de José
Cavalcante de Souza. In: ``Os Pensadores''. São Paulo, Abril Cultural,
1973, p.~94).} É o que basta para a humanidade mais tardia: que
se deixe interpretar apenas como sentenças oraculares aquilo que ele,
tal como o deus délfico, ``nem enuncia nem oculta''. Ainda que por ele
vaticinado ``sem sorriso, enfeites e essências aromáticas'', mas, antes
do mais, com a ``boca espumante'', é \textit{preciso} adentrar nos
milênios do futuro. Pois o mundo necessita eternamente da verdade, e, a
ser assim, necessita eternamente de Heráclito: mesmo que ele não
necessite do mundo. O que \textit{lhe} importa sua fama? A fama de
``mortais sempre a fluir!'', como ele exclama em tom escarninho. Sua fama
tem alguma importância aos homens, mas não a ele, a imortalidade da
humanidade necessita dele, não ele da imortalidade do homem Heráclito.
O que ele viu, \textit{a doutrina da lei no vir"-a"-ser e do jogo na
necessidade}, deve ser, doravante, eternamente vislumbrado: ele
levantou a cortina deste que é o maior de todos os espetáculos.

\sectionitem

Enquanto cada palavra de Heráclito expressa o orgulho e a
majestade da verdade, mas da verdade apreendida nas intuições, e não
daquela que subiu pela escada de corda da lógica; enquanto ele, num
arroubo sibilino, contempla, mas não espia, conhece, mas não calcula:
com seu contemporâneo Parmênides uma contra"-imagem coloca"-se \label{comseucontemporaneo}
ao seu lado, fazendo igualmente as vezes de um tipo de profeta da
verdade, mas como que formado de gelo, e não de fogo, emanando luz fria
e perfurante ao seu redor. Possivelmente apenas em idade mais
avançada Parmênides chegou a ter um momento da mais pura abstração,
desanuviada por qualquer efetividade e completamente exangue; esse momento --
não"-grego como nenhum outro nos dois séculos da
era trágica --, cujo produto é a doutrina do ser, tornou"-se um divisor
de águas para a sua própria vida, separando"-a em dois períodos; mas, ao
mesmo tempo, esse momento divide o pensamento pré"-socrático em duas
metades, cuja primeira pode ser chamada de anaximândrica e a
segunda, por seu turno, justamente de parmenidiana. O primeiro e mais
antigo período pertinente ao próprio filosofar de Parmênides ainda traz
consigo, igualmente, a face de Anaximandro; ele cria, como resposta às
questões de Anaximandro, um organizado sistema físico"-filosófico. Mais
tarde, quando aquele gélido tremor da abstração se apoderou dele e a
mais simples sentença discursiva do ser e não"-ser foi por ele
elaborada, seu próprio sistema terminou por se encontrar também entre as diversas
doutrinas anteriores que ele mesmo havia negado. Contudo, ele não
parece ter perdido toda piedade patriarcal pela pujante e bem"-nascida
criança de sua juventude, e, por isso, prestou um favor a si mesmo ao
dizer: ``há, com efeito, apenas um caminho correto; se, porém, alguém
quiser tomar algum outro, então, segundo sua qualidade e
consequência, a minha concepção mais antiga é a única acertada''.
Protegendo"-se mediante esse enfoque, ele consagrou ao seu sistema
físico anterior um espaço digno e amplo no interior mesmo daquele
enorme poema sobre a natureza, que, em rigor, deveria proclamar a nova
compreensão como o único sinal norteador rumo à verdade. Esse respeito
paternal, ainda que tenha ocasionado o surgimento furtivo de um erro, é
um resíduo de sentimento humano em meio a uma natureza totalmente
petrificada pela rigidez lógica e quase transformada numa máquina de pensar.

Parmênides, cujo contato pessoal com Anaximandro não me parece ser
inacreditável e cujo ponto de partida na doutrina deste último não é
apenas verossímil, mas evidente, tinha a mesma desconfiança com
relação à separação total entre um mundo que apenas é e um mundo que
apenas vem a ser, desconfiança que também se apoderou de Heráclito e
que, em geral, levou"-o à negação do ser. Ambos buscaram uma saída para
a oposição e divisão de uma dupla ordenação de mundo. Aquele salto no
indeterminado, no indeterminável, por meio do qual Anaximandro havia
escapado de uma vez por todas do âmbito do vir"-a"-ser e de suas
qualidades empiricamente dadas, não foi aceito com muita facilidade por
mentes tão independentes como as de Heráclito e Parmênides; eles
procuraram, antes, avançar o tanto quanto lhes fosse possível e
pouparam o salto até o ponto em que o pé já não mais encontra apoio, de
sorte que, para não cair, é necessário pular. Ambos contemplaram
repetidamente aquele mesmo mundo que Anaximandro havia tão
melancolicamente condenado e elucidado como local do crime, e, ao mesmo
tempo, como o lugar de penitência para a injustiça do vir"-a"-ser. Em seu
contemplar, Heráclito descobriu, como já sabemos, aquela maravilhosa
ordenação, regularidade e segurança que se revelam em todo vir"-a"-ser:
concluiu, daí, que o vir"-a"-ser mesmo não poderia ser nada criminoso e
injusto. De maneira bem outra foi o olhar de Parmênides, que comparou as
qualidades entre si e acreditou ter descoberto que elas não eram
todas semelhantes, mas, antes, que tinham de ser ordenadas sob duas rubricas.
Quando comparou, por exemplo, luz e escuridão, então esta segunda
qualidade era evidentemente apenas a \textit{negação} da primeira; e
assim foi que ele distinguiu qualidades positivas e negativas,
esforçando"-se seriamente para reencontrar e marcar esta oposição básica
em todo o âmbito da natureza. Seu método, aqui, era o seguinte: ele
tomava um par de opostos como, por exemplo, leve e pesado, fino e \label{eletomavaumpar}
espesso, ativo e passivo, e mantinha"-os à luz daquela oposição exemplar
entre luz e escuridão; aquilo que correspondia à luz era o positivo,
aquilo que correspondia à escuridão, a propriedade negativa. Se tomava,
digamos, o pesado e o leve, então o leve recaía sobre o lado da luz, ao
passo que o pesado, sobre o lado da escuridão; e, desse modo, o pesado
era"-lhe apenas a negação do leve, mas este, por seu turno, era"-lhe uma
propriedade positiva. Desse método decorre, já, uma obstinada aptidão
para o procedimento lógico"-abstrato, impermeável às influências dos
sentidos. O pesado decerto parece oferecer"-se incessantemente aos 
sentidos como uma qualidade positiva; algo que não impediu
Parmênides de imprimir"-lhe o selo da negação. Da mesma forma, ele
descreveu a terra em oposição ao fogo, o frio em oposição ao quente, o
espesso em oposição ao fino, o feminino em oposição ao masculino, o
passivo em oposição ao ativo, apenas como negações: de sorte que,
diante de seu olhar, nosso mundo empírico dividiu"-se em duas esferas
separadas, a saber, aquela pertinente às propriedades positivas -- com um
caráter diáfano, ígneo, quente, leve, fino e ativamente masculino -- e
aquela consoante às propriedades negativas. Estas últimas expressam, em
rigor, apenas a escassez, a ausência das outras, isto é, das
propriedades positivas; assim, ele descreveu a esfera à qual faltam as
propriedades positivas como escura, telúrica, fria, pesada, espessa e,
em geral, com caráter femininamente passivo. Em vez das expressões
``positivo'' e ``negativo'', valeu"-se dos termos fixos ``existente'' e
``não"-existente'', e, com isso, em contraposição a Anaximandro, chegou ao
princípio de que este nosso próprio mundo contém algo de existente; e,
por certo, também algo de não"-existente. O existente não deve ser
buscado fora do mundo e tampouco para além de nosso horizonte; 
ao contrário, diante de nós e em todas as partes, 
em todo vir"-a"-ser, está contido algo de existente e em atividade.

Mas, desta feita, restou"-lhe a tarefa de fornecer a resposta mais
precisa à pergunta: o que é o vir"-a"-ser? E aqui foi o momento em que
teve de saltar para não cair, apesar de que, talvez, para naturezas
tais como a de Parmênides, todo saltar equivalesse a cair. Basta dizer
que agora adentramos na neblina, na mística das \textit{qualitates}
\textit{occultae} e, inclusive, na mitologia. Como Heráclito, Parmênides
contempla o vir"-a"-ser e a infixidez universais, mas só pode interpretar
para si um perecer na medida em que o não"-existente também lhe seja
imputado. Pois como poderia o existente carregar a culpa do perecer!
Mas, o devir [\textit{das Entstehen}] precisa, também ele, consumar"-se
mediante o auxílio do não"-existente: pois o existente está sempre aí e
não poderia absolutamente surgir a partir de si e tampouco explicar
qualquer devir. Assim, tanto o devir como o perecer são engendrados
pelas propriedades negativas. Mas, já que aquilo que veio a ser possui
um conteúdo e o que pereceu perdeu um conteúdo, pressupõe"-se que as
propriedades positivas -- quer dizer, justamente tal conteúdo -- tomam
parte igualmente dos dois processos. Disso resulta, em suma, o
princípio segundo o qual ``tanto o existente quanto o não"-existente são
necessários para o vir"-a"-ser; quando ambos atuam juntos, dá"-se então um
vir"-a"-ser''. Mas como o positivo e o negativo operam um com o outro? Não
deveriam eles, ao contrário, evadir"-se eternamente um do outro como
coisas contrárias e, ao fazê"-lo, tornar impossível todo vir"-a"-ser?
Parmênides apela, aqui, para uma \textit{qualitas occulta}, para uma
tendência mística dos opostos a irmanarem"-se e atraírem"-se entre si,
sendo que sensibiliza essa oposição por meio do nome de Afrodite e
mediante a relação empiricamente conhecida entre o masculino e o
feminino.\footnote{ O que se deixa ilustrar, por exemplo, com o
seguinte fragmento: ``Pois os mais estreitos encheram"-se de fogo sem
mistura,/e os seguintes, de noite, e entre (os dois) projeta"-se parte
de chama;/mas no meio destes a Divindade que tudo governa;/pois em
tudo ela rege odioso parto e união/mandando ao macho unir"-se a fêmea
e pelo contrário/o macho à fêmea'' (\textit{Os pré"-socráticos}.
Tradução de José Cavalcante de Souza. In: ``Os Pensadores''. São Paulo,
Abril Cultural, 1973, p.~150).} É o poder de Afrodite que acopla
os opostos, unindo o existente ao não"-existente. Um desejo aproxima os
elementos que conflitam e repelem"-se mutuamente: o resultado é um
vir"-a"-ser. Quando o desejo é saciado, o ódio e o conflito interno
dividem uma vez mais o existente e o não"-existente -- e, aí então, o
homem diz: ``a coisa perece''.

\sectionitem

Mas ninguém assedia impunemente abstrações tão amedrontadoras tais como
o ``existente'' e o ``não"-existente''; quando as tocamos, o sangue aos
poucos se congela. Houve um dia em que Parmênides teve uma ideia
insólita, que pareceu retirar a tal ponto o valor de todas as suas
combinações anteriores que ele teve vontade de colocá"-las de lado
como se fossem um saco de moedas velhas e desgastadas. Supõe"-se comumente que uma
impressão exterior, e não só a consequência internamente atuante de
conceitos tais como ``existente'' e ``não"-existente'', também concorreu
ativamente para o achado daquele dia, a saber, o contato com a teologia 
daquele velho e muitíssimo experimentado rapsodo, cantor de um
endeusamento místico da natureza, do colofônio \textit{Xenófanes}. Ao
longo de uma vida extraordinária, Xenófanes viveu como poeta errante e
tornou"-se, por meio de suas viagens, um homem bem mais instruído e
instrutivo, que sabia questionar e narrar; por isso, Heráclito o
computou entre os poliistoriadores [\textit{Polyhistoren}] e, em
geral, entre as naturezas ``históricas'' no sentido já assinalado. De
onde e quando lhe adveio a inclinação mística em direção ao uno e ao
eternamente imóvel, eis o que ninguém poderá recontar; talvez seja,
antes de tudo, a concepção de um velho homem que, por fim, sossegou"-se, \label{velhohomem}
daquele que, depois da mobilidade de seus descaminhos e do incansável
aprender e investigar, encontra, diante da alma, o que há de mais
elevado e grandioso na visão de uma tranquilidade divina, na
permanência de todas as coisas no interior de uma paz
primordial panteísta. A mim me parece, aliás, puramente acidental o fato de dois
homens terem vivido juntos por um longo período justamente no mesmo
lugar, em Eleia, trazendo na mente, cada qual, uma concepção de
unidade: eles não formam nenhuma escola e tampouco partilham algo que
um pudesse aprender do outro e que, depois, fosse ensinado adiante.
Pois a origem de tal concepção de unidade é, num caso, totalmente
diferente da outra, inclusive fazendo"-lhe oposição; e se um deles
chegou a conhecer, em geral, a doutrina do outro, então teve de
traduzi"-la para si mesmo em sua própria linguagem, simplesmente para
poder entendê"-la. Mas, nesta tradução, perde"-se de qualquer modo
justamente o que há de específico na doutrina do outro. Enquanto
Parmênides atingiu a unidade do existente apenas por meio de uma
consequência supostamente lógica, tecendo"-a a partir do conceito de ser
e não"-ser, Xenófanes é um místico religioso e, com tal unidade mística,
pertence legitimamente ao século sexto. Ainda que não fosse uma
personalidade tão transformadora como Pitágoras, ele comungava, porém,
com este último, da mesma tendência e impulso de melhorar, purificar e
curar os homens. É o instrutor de ética, mas ainda no nível do rapsodo;
numa época ulterior, teria sido um sofista. Não havia, na Grécia, quem
lhe igualasse na ousada reprovação dos costumes e apreciações vigentes;
para tanto, ele de modo algum se recolheu na solidão, tal como
Heráclito e Platão, mas, ao contrário, colocava"-se precisamente diante
daquele público cuja admiração entusiástica de Homero, cuja apaixonada
ânsia pelas honras dos festivais de ginástica, cuja adoração de pedras
esculpidas à semelhança do homem, ele tanto atacava irada e
sarcasticamente, embora não o fizesse como o queixoso
Térsites.\footnote{ Guerreiro aqueu da Guerra de Troia, Térsites é
descrito com fortes cores por Homero no Canto \textsc{ii} de \textit{A Ilíada}.
``Mais feio entre os gregos'', a personagem tornou"-se atuante pela falta
de decoro e pela incomparável rabugice. A título de ilustração, vale ler: 
``Todos, nos seus lugares, sentaram"-se, quietos./Só Tersites crocita, corvo
boquirroto,/a cabeça atulhada de frases sem ordem,/sem tino, desatinos
farpas contra os reis,/tudo para atiçar o riso dos Aqueus./Era o homem mais feio 
jamais vindo a Ílion:/vesgo, manco de um pé, 
ombros curvos em arco,/esquálido, cabeça pontiaguda, calva/à mostra, 
odioso para Aquiles e Odisseu, que a ambos insultava e que agora
ao divino Agamêmnon afronta com sua voz estrídula'' (Cf.~\textit{Ilíada de Homero}. Tradução de
Haroldo de Campos, São Paulo, Mandarim, 2001, Canto \textsc{ii}, v.~211--221).}
Com ele, a liberdade do indivíduo atinge o seu ápice; e, neste afastamento quase
indelineável de todas as convenções, está mais aparentado com
Parmênides do que com aquela derradeira unidade divina, que ele uma vez
vislumbrou num ângulo de visão digno daquele século, mas que, com o ser
único de Parmênides, não tem em comum sequer a expressão e o nome, e
muito menos a origem.

Foi, antes, numa situação oposta a esta que Parmênides achou a doutrina
do ser. Naquele dia e nesse estado de coisas, ele pôs à prova os seus
dois opostos mutuamente atuantes, cujo desejo e ódio constituem o mundo
e o vir"-a"-ser, o existente e o não"-existente, as propriedades positivas
e negativas -- e, de repente, deteve"-se com desconfiança no conceito de
propriedade negativa, no não"-existente. Aquilo que não é pode, pois,
ser uma propriedade? Ou, para indagar de modo ainda mais fundamental:
aquilo que não é, pode ser? Ora, a única forma de conhecimento no qual
depositamos imediatamente uma confiança incondicional e cuja negação
equivale à insânia é a tautologia \textsc{a} $=$ \textsc{a}. Mas é justamente esse
conhecimento tautológico que lhe proclama de modo implacável: o que não
é, é nada! O que é, é! De pronto, ele sentiu um monstruoso pecado
lógico pesar sobre sua vida; até então havia sempre admitido, mas sem
grandes preocupações, que talvez \textit{existissem} propriedades
negativas, algo de não"-existente em geral, que, portanto, para
expressar formalmente, \textsc{a} $=$ \textsc{não a} pudesse existir; isto, porém, só
poderia ser aventado pela completa malevolência do pensar. Mas, como
ele mesmo se deu conta, a grande maioria dos homens julga com a mesma
malevolência: ele próprio havia apenas participado do delito universal
contra a lógica. Contudo, o mesmo instante que lhe indica este delito
termina por iluminá"-lo com a glória de uma descoberta; ele encontrou,
para além de toda ilusão humana, um princípio, a chave para o segredo
do mundo, e, agora, sob o amparo da mão firme e abominável da verdade
tautológica sobre o ser, ele desce rumo ao abismo das coisas.

No caminho para lá ele encontra Heráclito -- um encontro mal"-afortunado!
Para ele, crente de que tudo se devia à mais rigorosa separação entre
ser e não"-ser, o jogo de antinomias de Heráclito tinha de ser,
sobretudo, algo profundamente detestável; uma proposição tal como, por
exemplo, ``somos e não somos a um só tempo'', ou, então, ``ser e não"-ser
são e não são simultaneamente idênticos'', isto é, uma proposição por
meio da qual tudo o que até então ele havia justamente iluminado e
desemaranhado tornou"-se novamente anuviado e impossível de
desembaraçar, impeliram"-no à fúria: fora daqui com os homens, gritava
ele, que parecem ter duas cabeças, mas que nada sabem! Com eles, tudo
está em fluxo, inclusive seu pensar! Admiram detidamente as coisas, mas
devem ser igualmente surdos e cegos para misturarem, desse modo, os
opostos uns com os outros! A incompreensão da massa, glorificada por
meio de antinomias lúcidas e louvada como o cume de todo conhecimento,
era"-lhe uma vivência dolorosa e incompreensível.\footnote{ Sem pretender detectar, aqui, os possíveis erros de
``calibragem'' do texto nietzschiano, convém trazer à baila, à guisa de
exemplo, o fragmento que se acha à base da caracterização feita pelo
filósofo alemão: ``Necessário é o dizer e pensar que (o) ente é; pois é
ser,/e nada não é; isto eu te mando considerar./Pois primeiro desta
via de inquérito eu te afasto,/mas depois daquela outra, em que
mortais que nada sabem/erram, duplas cabeças, pois o imediato em seus/peitos
dirige errante pensamento; e são levados/como surdos e
cegos, perplexas, indecisas massas, para os quais ser e não ser é
reputado o mesmo/e não o mesmo, e de tudo é reversível o caminho''
(\textit{Os pré"-socráticos}. Tradução de José Cavalcante de Souza. In:
``Os Pensadores''. São Paulo, Abril Cultural, 1973, p.~148). 
Ver também Introdução, p.~\pageref{colli}.}

Então ele mergulhou no banho frio de suas formidáveis abstrações.
Aquilo que verdadeiramente existe, deve existir no presente eterno, de
sorte que, a seu respeito, não pode ser dito que ``foi'' ou ``virá a ser''.
O existente não pode ter vindo a ser: pois de onde poderia ter vindo?
Do não"-existente? Este, contudo, nada é e nada pode engendrar. Do
existente então? Este não engendraria nada a não ser a si próprio. O
mesmo se dá com o perecer; é tão impossível quanto o vir"-a"-ser, quanto
toda mudança, todo crescimento, todo decréscimo. O princípio vigora em
toda parte: tudo aquilo sobre o que pode ser dito que ``existiu'' ou
``existirá'' não é, mas do ser jamais pode ser dito que ``não é''. Aquilo
que existe é indivisível, pois onde está o segundo poder que seria
capaz de dividi"-lo? É imóvel, pois para onde iria mover"-se? Não pode
ser nem infinitamente grande nem infinitamente pequeno, pois é acabado,
e uma infinitude dada como acabada é uma contradição. Permanece, então,
suspenso no ar, delimitado, acabado, imóvel, em todas as partes em
equilíbrio, igualmente perfeito em todos os pontos, como uma esfera,
mas não num espaço; pois, do contrário, este espaço seria um segundo
ser existente. Não podem existir, porém, vários existentes, pois, para
apartá"-los, deveria haver algo que não fosse um ser existente: uma
conjectura que se auto"-suprime. Há, pois, apenas a unidade eterna.

Contudo, assim que Parmênides voltou a dirigir seu olhar para o mundo do
vir"-a"-ser, cuja existência ele havia, antes, procurado compreender por
meio de dedálicas combinações, aborreceu"-se com seus olhos pelo fato de
enxergarem o vir"-a"-ser em toda parte, e com seus ouvidos, por se
colocarem à escuta do mesmo. ``Não vos deixeis levar apenas pelo olho
grosseiro'', diz, agora, seu imperativo, ``e tampouco pelo estridente
ouvido ou pela língua, mas colocai tudo à prova apenas mediante a força
do pensamento!'' Com isso, ele levou a cabo a primeira crítica do
aparato cognitivo, sumamente importante não obstante sua insuficiência
e consequências calamitosas; ao desmembrar abruptamente os sentidos da
capacidade de pensar abstrações, quer dizer, da razão, como se se tratassem
de duas faculdades separadas de ponta a ponta, ele lançou por terra o
próprio intelecto e incitou àquela separação totalmente equivocada
entre ``espírito'' e ``corpo'', que, sobretudo desde Platão, recai
sobre a filosofia qual uma maldição. Todas as percepções sensíveis,
julga Parmênides, fornecem apenas ilusões; e sua ilusão principal
consiste justamente em fazer acreditar que o não"-existente também
existe, que o vir"-a"-ser, também ele, possui um ser. Toda aquela
multiplicidade e diversidade de cores do mundo empiricamente conhecido,
a mudança de suas qualidades, a ordenação de seus altos e baixos,
foram, pois, impiedosamente desprezadas como uma mera aparência e
ilusão; daqui nada se aprende, de sorte que todo esforço despendido com
tal mundo enganador, frívolo do começo ao fim e como que fraudado pelos
sentidos, é desperdiçado. Quem assim julga, no conjunto, tal como fez
Parmênides, deixa de ser com isso um investigador da natureza em
particular; seu interesse pelos fenômenos atrofia"-se, engendrando para
si um ódio pelo fato de não poder livrar"-se deste eterno embuste dos \label{odio}
sentidos. A verdade deve habitar, agora, apenas nas mais pálidas e
abstratas generalidades, nos estojos vazios das mais indeterminadas
palavras, como num abrigo feito de teia de aranha; e, junto a tal
``verdade'', senta"-se então o filósofo, enredado em fórmulas e tão
esvaído em sangue quanto uma abstração. A aranha quer, afinal de
contas, o sangue de suas vítimas; mas o filósofo parmenidiano detesta
justamente o sangue de sua vítima, o sangue da empiria por ele imolada.

\sectionitem

E esse foi um grego cujo florescimento é mais ou menos simultâneo à
emergência da revolução iônica. À época, para um grego era possível
evadir"-se da pródiga efetividade tal como se fosse um mero esquematismo
trapaceiro da imaginação -- não como Platão, rumo à terra das ideias
eternas, à oficina do artífice do mundo, para então fixar o olho sob as
imaculadas e indestrutíveis formas primordiais, mas rumo à rígida
tranquilidade mortal do mais gélido conceito, daquele que nada diz, a
saber, do conceito de ser. Tencionamos, por certo, guardar"-nos de
interpretar um fato tão curioso como este a partir de falsas analogias.
Aquela fuga não foi uma fuga do mundo no sentido dos filósofos hindus,
para sua consumação não concorreu a profunda e religiosa convicção
sobre a corruptibilidade, a efemeridade e o infortúnio da existência,
sendo que seu derradeiro alvo, o descanso no ser, não era buscado como
a imersão mística \textit{numa} representação extática universalmente
suficiente, o que, para o homem comum, é um enigma e um aborrecimento.
O pensamento de Parmênides não traz consigo, nem de longe, a fragrância
escura e inebriante dos hindus, que talvez não passasse totalmente
despercebida em Pitágoras e Empédocles; o admirável naquele
acontecimento, por volta de tal época, é antes justamente a falta de
fragrância, de cor, alma e forma, a inteira escassez de sangue,
religiosidade e calor ético, o abstrato"-esquemático acima de tudo -- num
grego! --, mas, sobretudo, a assustadora energia do esforço despendido
na busca por \textit{certeza} numa época que pensa miticamente e que se
move, em suma medida, de modo fantástico. Concedei"-me, ó deuses,
somente uma certeza, eis a prece de Parmênides, e que ela seja uma
tábua suficientemente grande para flutuar sobre o mar do incerto!
Guardai para vós mesmos todo vir"-a"-ser, entusiasmo e florescer, bem
como o que é ilusório, instigante e vivo: e dai"-me somente a única e
pobre certeza vazia!

O tema da ontologia prenuncia"-se na filosofia de Parmênides. Em nenhuma \label{temadaontologia}
parte a experiência lhe ofereceu um ser tal como havia ideado, mas,
porque foi capaz de pensá"-lo, ele então concluiu que tal ser tinha de
existir; uma conclusão baseada no pressuposto de que temos um órgão do
conhecimento que se dirige à essência das coisas e independe da
experiência. Segundo Parmênides, a matéria de nosso pensar não se acha
presente, em absoluto, na intuição, senão que é trazida de um outro
lugar, a saber, de um mundo supra"-sensível ao qual obtemos um acesso
direto por meio do pensar. Ora, a contrapelo de todas as inferências
semelhantes a essa, Aristóteles tornou patente que a existência jamais
pertence à essência, que o existir nunca faz parte da essência das
coisas. Justamente por isso, não se pode inferir uma
\textit{existentia} do ser a partir do conceito ``ser'' -- cuja
\textit{essentia} nada é senão o próprio ser. A verdade lógica daquela
oposição entre ``ser'' e ``não"-ser'' é inteiramente vazia, caso o objeto
que se acha à sua base não puder ser dado, caso não puder ser dada a
intuição a partir da qual tal oposição deriva"-se por abstração,
sendo que, sem essa remissão à intuição, tal verdade é apenas um jogo
com representações por meio do qual, de fato, nada é conhecido. Pois,
tal como ensina Kant, o mero critério lógico da verdade, a saber, 
o acordo de um conhecimento com as leis universais e formais do
entendimento e da razão, é a \textit{conditio sine qua non} e,
por conseguinte, a condição negativa de toda verdade: a lógica não
pode, porém, ir mais além e, para descobrir o erro naquilo que tange,
não à forma, mas ao conteúdo, não pode dispor de nenhuma pedra de
toque. Assim que se procura todavia o conteúdo para a verdade lógica
pertinente à oposição ``aquilo que é, é; aquilo que não é, é nada'', então
já não se encontra, com efeito, nenhuma efetividade que fosse
estritamente formada de acordo com tal oposição; acerca de uma árvore
posso dizer tanto que ``ela é'', em comparação com todas as outras coisas
restantes, como também que ``ela vem a ser'', em comparação consigo mesma
num outro momento do tempo, a ponto de poder igualmente dizer, por fim,
que ``ela não é'' quando, por exemplo, ao observar um arbusto, digo
``ainda não é uma árvore''. As palavras são apenas símbolos das relações
das coisas umas com as outras e conosco, não tocam a verdade absoluta
em lugar algum; e mesmo a palavra ``ser'' designa somente a relação mais
universal capaz de unir todas as coisas, bem como a palavra ``não"-ser''.
Mas, se a existência das próprias coisas não pode ser provada, então a
relação das coisas entre si, os assim chamados ``ser'' e ``não"-ser'',
tampouco poderão aproximar"-nos, um passo sequer, da terra da verdade.
Jamais nos será dado, mediante palavras e conceitos, colocar"-se atrás
do muro das relações, como que em algum fabuloso fundamento primordial
das coisas, sendo que mesmo nas formas puras da sensibilidade e do
entendimento, no tempo, espaço e na causalidade, não adquirimos nada
que se compare a uma \textit{veritas aeterna}. É absolutamente
impossível ao sujeito ver algo para além de si e querer conhecê"-lo, tão
impossível que conhecer e ser constituem as mais contraditórias de
todas as esferas. E se Parmênides, na incipiente ingenuidade da então
crítica do intelecto, permitiu"-se imaginar que havia alcançado um
ser"-em"-si para além do conceito eternamente subjetivo, hoje, depois de
Kant, é uma acintosa ignorância estipular a tarefa da filosofia, como
se faz aqui e acolá e, em especial, entre teólogos mal instruídos que
querem brincar de filósofos, ``compreender o absoluto com a
consciência'' ou mesmo na forma com que se expressou Hegel: ``o
absoluto está, já, presente, senão como poderia ser procurado?'', ou,
então, para utilizar a formulação de Beneke, ``o ser deve, de algum
modo, estar dado, sendo"-nos igualmente alcançável, pois, do contrário,
em nenhum momento poderíamos ter o conceito de ser''. O conceito de ser!
Como se ele não apontasse, já, para a mais rasa origem empírica na
própria etimologia da palavra! Pois, no fundo, \textit{esse} significa
apenas ``respirar'': se o homem a utiliza para todas as outras coisas,
então transpõe a convicção de que ele mesmo respira e vive por meio de
uma metáfora sobre as demais coisas, isto é, por meio de algo ilógico,
compreendendo sua existência como um respirar de acordo com a analogia
humana. Oblitera"-se, de imediato, o significado original da palavra;
mas sempre restam muitos indícios de que o homem representa a
existência das outras coisas por analogia com sua própria existência,
por conseguinte, de maneira antropomórfica, e, em todo caso, por meio
de uma transposição ilógica. Mesmo para o homem, deixando de lado
tal transposição, a sentença ``respiro, logo há um ser'' é totalmente
insuficiente: de modo que a ela deve ser feita a mesma objeção que se
faz ao \textit{ambulo, ergo sum} ou \textit{ergo est}.

\sectionitem

O outro conceito, de maior conteúdo que a noção de existente e que
também já fora descoberto por Parmênides, embora este não o tivesse
empregado de modo tão hábil quanto seu aluno Zenão, é o conceito de
infinito. Não pode existir nada de infinito, pois de tal suposição
decorreria o conceito contraditório de uma infinitude acabada. Mas, na
medida em que nossa efetividade, nosso mundo existente, traz em todas
as partes o caráter de tal infinitude acabada, então, conforme sua
essência, significa uma contradição com o lógico e, portanto,
igualmente com o real, sendo, de resto, ilusão, mentira e fantasma.
Zenão valeu"-se, sobretudo, de um método de demonstração indireto: \label{zenaovaleuse}
disse, por exemplo, que ``não pode haver nenhum movimento de um lugar
para outro, pois, se tal movimento existisse, então haveria uma
infinitude acabada; isto é, porém, uma impossibilidade''. Aquiles não
pode, na corrida, alcançar a tartaruga que detém uma pequena vantagem à
sua frente: pois, para atingir o ponto do qual partiu a tartaruga, ele
teria que ter percorrido, já, inúmeros e infinitos espaços, a saber,
primeiramente a metade de tal espaço, daí então um quarto dele, depois
um oitavo, dezesseis avos e assim por diante \textit{ad infinitum}. Se
ele factualmente alcança a tartaruga, trata"-se então de um fenômeno
ilógico, portanto, em todo caso, não há aqui nenhuma verdade, realidade
ou ser verdadeiro, mas apenas uma ilusão. Pois nunca é possível pôr um
termo ao infinito. Um outro meio popular para expressar essa doutrina é
dado pela flecha que voa e, no entanto, permanece em repouso. Em cada
instante de seu voo, ela assume um lugar no espaço: em tal lugar, ela
repousa. Então a soma das infinitas posições de repouso seria algo
idêntico ao movimento? Ou seria o estado de repouso, infinitamente
repetido, igual ao movimento, quer dizer, igual ao seu próprio oposto?
O infinito é, aqui, utilizado como água"-forte da efetividade, de sorte
que esta se dissolve naquele. Mas, se os conceitos são fixos, eternos e
existentes -- e, para Parmênides, ser e pensar coadunam"-se --, se,
portanto, o infinito nunca pode ser completo e o repouso jamais pode
tornar"-se movimento, então, em verdade, a flecha sequer chegou a voar:
nunca saiu da posição e do repouso, nenhum momento do tempo
transcorreu. Ou, dito de outro modo: nesta suposta e assim chamada
efetividade, não há nem tempo nem espaço, e tampouco movimento. A
própria flecha é, ao fim e ao cabo, uma ilusão; pois advém da
multiplicidade, da fantasmagoria do não"-um [\textit{des Nicht"-Einen}]
engendrada pelos sentidos.\footnote{ Para Zenão, a declaração de que o
vir"-a"-ser possui existência efetiva deveria ser desacreditada em
virtude das inconsequências a que essa mesma afirmação conduziria. A
seu ver, a própria transitoriedade retiraria de si, \textit{per
absurdum}, as provas que a denegam. Situando a força do raciocínio
lógico contra os dados da percepção, o pensador eleata procura
enfraquecer a hipótese adversária manipulando a ideia de série
infinita, oferecendo um paradoxo cuja irresolução teria, ao que parece,
escapado à pouca cautela dos defensores do vir"-a"-ser. Tal como
mostra"-nos o texto de Nietzsche, o célebre argumento se desdobraria da
seguinte maneira: no trajeto entre dois pontos situados ao longo de uma
dada reta -- \textsc{a} e \textsc{b}, por exemplo --, ter"-se"-ia de percorrer, num certo
tempo, a metade de seu comprimento a fim de levar a cabo o deslocamento
prenunciado, mas, a essa metade, impor"-se"-ia ainda mais uma divisão e a
esta, por sua vez, outras infinitas e inafugentáveis subdivisões. Se,
por um lado, a reta permanece sempre igual, \textsc{ab} $=$ 1, a série que nela se
desdobra cresceria, em contrapartida, ao infinito: ½ $+$ ¼ $+$ \ldots $=$ 1.
Zenão irá questionar, pois, a possibilidade de percorrer, num tempo
limitado, a extensão de uma reta ilimitadamente fracionada.
Inconcebível ao pensamento, tal situação levaria à fatídica conclusão
de que nem o movimento nem a transitoriedade seriam logicamente
concebíveis. Donde sua lapidar asserção: ``O móvel nem no espaço em que
está se move, nem naquele em que não está'' (\textit{Os pré"-socráticos}.
Tradução de Ísis L.~Borges. In: ``Os Pensadores''. São Paulo, Abril
Cultural, 1973, p.~203).} Admitindo"-se que a flecha tivesse
um ser, então ela seria imóvel, atemporal, nunca vindo a ser, rígida e
eterna -- uma representação impossível! Admitindo"-se que o movimento
fosse verdadeiramente real, então não haveria repouso, e, portanto,
qualquer posição para a flecha, logo nenhum espaço -- uma representação
impossível! Admitindo"-se que o tempo fosse real, então não poderia ser
infinitamente divisível; o tempo utilizado pela flecha teria de
consistir numa quantidade limitada de momentos temporais, sendo que
cada um destes momentos deveria equivaler a um átomo -- uma
representação impossível! Todas as nossas representações conduzem a
contradições logo que seu conteúdo empiricamente dado, criado a partir
desse mundo intuitivo, é tomado por \textit{veritas aeterna}. Havendo
movimento absoluto, então não há espaço; havendo espaço absoluto, então
não há movimento; havendo um ser absoluto, então não há multiplicidade.
Havendo uma multiplicidade absoluta, então não há unidade. Dever"-se"-ia,
com isso, ficar claro quão pouco tocamos o coração das coisas ou
desfazemos o nó da realidade mediante tais conceitos. 
Parmênides e Zenão aferram"-se, ao contrário, à verdade e à validade
universal dos conceitos, desvencilhando"-se do mundo intuitivo como a
antítese dos conceitos verdadeiros e universalmente válidos, como uma
objetivação do ilógico e contraditório. Em todas as suas demonstrações,
eles partem do pressuposto totalmente indemonstrável, inclusive
improvável, de que detemos, naquela faculdade de conceitos, o mais
elevado critério decisório sobre o ser e não"-ser, quer dizer, sobre a
realidade objetiva e seu contrário. Tais conceitos não devem colocar"-se
à prova e corrigir a si mesmos a partir da efetividade, ainda que dela
sejam factualmente derivados, senão que devem, ao contrário, medi"-la e
dirigi"-la, e, no caso de uma contradição com aquilo que há de lógico,
devem até mesmo amaldiçoá"-la. A fim de poder conceder aos conceitos
essas proficiências judicantes, Parmênides teve de atribuir"-lhes o
mesmo ser que havia deixado vigorar, em geral, como o único ser: o
pensar e aquela única e perfeita esfera do existente, que nunca veio a
ser, já não deviam ser apreendidos como dois tipos distintos de ser, já
que não podia haver qualquer dualidade do ser. Então se tornou
necessária a ideia muitíssimo ousada de explicar, como coisas
idênticas, pensar e ser; aqui, nenhuma forma de intuição, nenhum
símbolo e nenhuma alegoria poderiam prestar algum socorro; a ideia era
inteiramente irrepresentável, porém necessária, a ponto de celebrar, na
falta de qualquer possibilidade de sensibilização intuitiva, o mais
elevado triunfo sobre o mundo e as exigências dos sentidos. Segundo o
imperativo parmenidiano, o pensar e aquele ser esférico"-bulboso,
funesto, massivo, rígido e inerte de ponta a ponta, deveriam, para o
espanto de toda fantasia, coincidir numa unidade, sendo a mesma e uma
só coisa. Esta identidade bem que pode contradizer os sentidos!
Justamente isto serve para endossar o fato de que não foi emprestada
dos sentidos. 

\sectionitem

Além disso, poder"-se"-ia igualmente mobilizar contra Parmênides um
poderoso par de argumentos \textit{ad hominem} ou \textit{ex concessis}
por meio dos quais, com efeito, a verdade mesma não poderia ser trazida
à luz, mas, talvez, a inverdade pertinente àquela separação absoluta
entre o mundo dos sentidos e o mundo dos conceitos, bem como a da
identidade entre ser e pensar. Em primeiro lugar, se o pensar da razão
mediante conceitos é real, então a multiplicidade e o movimento também
devem ter realidade, pois o pensar racional move"-se, consistindo, então,
num movimento que vai de um conceito a outro conceito, e, portanto, no
interior de uma pluralidade de realidades. Contra isso não há qualquer
evasiva, sendo totalmente impossível descrever o pensar como um rígido
perdurar, como um pensar"-a"-si"-mesmo da unidade eternamente imóvel. Em
segundo lugar: se dos sentidos só advêm engodo e aparência, havendo, em
verdade, somente a identidade real entre ser e pensar, o que são então
os próprios sentidos? Em todo caso, também uma aparência apenas: haja
vista que não se coadunam com o pensar e o seu produto, o mundo dos
sentidos, tampouco se coaduna com o ser. Mas, se os sentidos mesmos
constituem uma aparência, então para quem eles aparecem como tal? Como
poderiam, sendo irreais, ainda assim iludir? O não"-existente não pode
enganar sequer uma vez. Assim, a questão ``de onde vem?'' relativamente à
ilusão e à aparência permanece um enigma, inclusive uma contradição.
Denominamos estes argumentos \textit{ad hominem} uma objeção face à
mobilidade da razão e à origem da aparência. Da primeira objeção
decorreria a realidade do movimento e da multiplicidade, e, da segunda,
a impossibilidade da aparência parmenidiana; pressupondo que a
principal doutrina de Parmênides, aquela sobre o ser, esteja admitida,
já, como uma doutrina fundamentada.

Mas, esta principal doutrina prega apenas que só o existente possui um ser,
o não"-existente nada é. Se, porém, o movimento constitui tal ser, então
vale para ele o que vale para o existente em geral e em todos os casos, que o que
não veio a ser é eterno e indestrutível, sem aumento e decréscimo.
Mas, se a aparência é posta para fora deste mundo com o auxílio da
pergunta ``de onde ela vem?'', se o palco do assim chamado vir"-a"-ser, da
mudança, da nossa rica, colorida, multiforme e ininterrupta existência
está como que protegido contra a rejeição parmenidiana, então é preciso
caracterizar este mundo da transformação e da mudança como uma
\textit{soma} de tais essencialidades verdadeiramente existentes,
presentes a um só tempo por toda a eternidade. Naturalmente, com tal
suposição, também não se pode falar, nem de longe, de um vir"-a"-ser, de
uma mudança no sentido estrito do termo. Mas, assim como todas as
qualidades, agora a multiplicidade possui um ser verdadeiro, e em igual
medida o movimento; e dever"-se"-ia poder dizer a respeito de cada
momento deste mundo, ainda que tais momentos arbitrariamente
selecionados estivessem separados por milênios entre si: todas as
verdadeiras essencialidades contidas no mundo existem num todo único e
de uma só vez, imutáveis, irredutíveis, sem acréscimo nem decréscimo.
Depois de um milênio, são precisamente as mesmas e nada se transformou.
Apesar disso, se de tempos em tempos o mundo parece totalmente
diferente, então não se trata de nenhuma ilusão e tampouco de algo
meramente aparente, senão, da consequência do eterno movimento. O
verdadeiro existente move"-se ora de um jeito, ora de outro, algumas
vezes junto e outras separado, quer para cima quer para baixo, consigo
e através de si.

\sectionitem

Já demos, com esta representação, um passo em direção ao âmbito da doutrina
de Anaxágoras. As duas objeções contra Parmênides são por ele
levantadas com força total, a que diz respeito à mobilidade do pensar e
a que se refere à questão ``de onde vem a aparência?''; mas, na
proposição principal, Parmênides subjugou"-o tanto quanto subjugou todos
os filósofos e investigadores da natureza que eram mais jovens.
Todos denegam a possibilidade do vir"-a"-ser e perecer no sentido em que
o povo os imagina e tal como Anaximandro e Heráclito haviam admitido
com uma ponderação mais profunda, mas, ainda assim, temerária. Tal
surgimento mitológico a partir do nada, tal desaparecer no nada, tal mudança
arbitrária do nada em alguma outra coisa, tal intercâmbio casual, tal
extrair e aduzir das qualidades passou a valer, daí em diante, como
algo sem sentido; mas, do mesmo modo e pelas mesmas razões, um surgimento da
multiplicidade a partir do uno, das qualidades múltiplas a partir da
única qualidade primordial, em suma, a dedução do mundo a partir de uma
matéria primordial à maneira de Tales ou Heráclito. Agora, de fato, o
legítimo problema foi colocado, a saber, como transpor a doutrina do
ser imperecível e sem vir"-a"-ser para este mundo existente sem recorrer,
como abrigo, à teoria da aparência e da ilusão por meio dos sentidos.
Mas se o mundo empírico não deve ser aparência, se as coisas não devem
ser deduzidas do nada e menos ainda de algo único, então essas coisas
mesmas devem encerrar um ser verdadeiro, sua matéria e seu conteúdo têm
de ser necessariamente reais, sendo que toda mudança pode referir"-se
apenas à forma, quer dizer, ao posicionamento, à ordenação, ao
agrupamento, à mistura e à decomposição dessas essencialidades que
existem eternamente juntas. Tudo se passa, então, como num jogo de
dados: são sempre os mesmos dados, mas, ao caírem quer de um jeito,
quer de outro, passam a significar coisas distintas para nós. Todas as
teorias mais antigas remetiam a um elemento primordial, como ventre e
causa do vir"-a"-ser, seja ele água, ar, fogo ou o indeterminado de
Anaximandro. Em contraposição, Anaxágoras afirma que o desigual jamais
poderia emanar do igual e que a mudança nunca poderia ser explicada a
partir do único ser existente. Pode"-se imaginar esta única suposta
matéria como algo condensado ou diluído, mas, com tal condensação ou
diluição, jamais se alcançará o que se deseja esclarecer: a
multiplicidade das qualidades. Mas se o mundo é, de fato, prenhe das
mais diversas qualidades, então estas últimas, caso não sejam
aparência, têm de possuir um ser, isto é, devem existir eternamente
sem que tivessem vindo a ser, de modo imperecível e a um só tempo. Não
podem, contudo, ser aparência, haja vista que, com isso, a questão ``de
onde vem a aparência?'' permanece sem resposta, sim, e responde a si 
mesma com um ``não!''. Ao estabelecerem apenas uma substância
que traria, em seu ventre, as possibilidades de todo vir"-a"-ser, os
investigadores mais antigos haviam desejado simplificar o problema do
vir"-a"-ser; agora, ao contrário, diz"-se: há inúmeras substâncias, mas
nunca mais, nunca menos, jamais substâncias novas. Apenas o movimento
as faz girar, como num lance de dados, umas em relação às outras; que,
no entanto, o movimento é uma verdade e não uma aparência, eis o que
Anaxágoras demonstrou, contra Parmênides, a partir da indiscutível
sucessão de nossas representações no próprio pensar. Temos, pois, da
maneira mais imediata, uma ideia acerca da verdade do movimento e da
sucessão na medida em que pensamos e possuímos representações. Assim,
de qualquer modo, o ser único de Parmênides, fixo, imóvel e morto,
acha"-se fora do caminho, pois é tão certo haver diversos
existentes quanto é certo que todos eles (existências, substâncias
etc.) estão em movimento. Mudança é movimento -- mas de onde se origina
o movimento? Acaso tal movimento não deixa totalmente intacta a
verdadeira essência daquelas diversas substâncias isoladas e autônomas,
e não \textit{tem} ele de ser, em si, conforme o mais rigoroso conceito
de existente, estranho a tais substâncias? Ou pertence ele, apesar de
tudo, às coisas mesmas? Deparamo"-nos com uma relevante decisão: a
depender da direção para qual viramos, iremos adentrar pelo terreno de
Anaxágoras, Empédocles ou Demócrito. A inquietante pergunta
deve ser colocada: se há diversas substâncias e estas se movimentam, o
que então as movimenta? Movimentam"-se umas contra as outras? É apenas a
força da gravidade que as move? Ou há, nas próprias coisas, forças
mágicas de atração ou repulsão? Ou o ensejo do movimento acha"-se para
além dessas diversas substâncias reais? Ou, para indagar com mais
rigor: quando duas coisas indicam uma sucessão, uma mudança recíproca
de posição, isso provém delas mesmas? E cumpre explicá"-lo mecânica ou
magicamente? Ou, se esse não fosse o caso, haveria então um terceiro
elemento que as movimentasse? É um problema terrível: pois, mesmo
concedendo que existissem muitas substâncias, Parmênides também
poderia, ainda assim, ter demonstrado a impossibilidade do movimento
contra Anaxágoras. Poderia, por exemplo, dizer: tomai duas essências
existentes em si, cada qual com um ser autonomamente incondicionado e
completamente diverso -- e as substâncias de Anaxágoras são desse tipo
--, elas jamais poderiam, pois, ir de encontro uma à outra, nunca se
movendo ou atraindo"-se mutuamente, sem qualquer causalidade ou ponte
entre ambas, de sorte que, não se tocando nem se perturbando, uma nada
tem a ver com a outra. Desse modo, o contato é tão ininteligível quanto \label{ocontatoe}
a atração mágica; coisas que são necessariamente estranhas entre si não \label{coisasquesao}
podem exercer qualquer tipo de efeito uma em relação à outra, e,
portanto, tampouco podem mover ou ser movidas. Parmênides teria até
mesmo acrescentado: a única saída que vos resta é imputar o movimento
às coisas mesmas; mas tudo aquilo que vós conheceis e vedes como
movimento não passa, afinal, de uma ilusão e está longe de ser o
verdadeiro movimento, pois o único tipo de movimento que poderia
coadunar"-se com aquelas substâncias incondicionalmente peculiares seria
somente um movimento idiossincrático sem aquele mencionado efeito. Ora,
vós aceitais precisamente o movimento a fim de explicar tais efeitos da
transformação, do deslocamento no espaço, da mudança, em suma, as
causalidades e as relações das coisas entre si. Mas justamente tais
efeitos não seriam explicados e permaneceriam, de resto, tão
problemáticos quanto antes; por isso, não se pode vislumbrar por que
motivo seria necessário admitir um movimento que não pode, em absoluto,
fornecer aquilo que vós desejais. O movimento não é inerente à essência
das coisas, é, quanto ao mais, eternamente estranho a elas.

A fim de sobreporem"-se a tal argumentação, aqueles adversários da
unidade imóvel eleata se deixaram levar por um preconceito advindo da
sensibilidade. Parece ser incontestável o fato de que qualquer coisa
verdadeiramente existente é um corpo que preenche o espaço, uma porção
de matéria, grande ou pequena, mas que, de qualquer modo, estende"-se no
espaço; de sorte que duas ou mais porções de matéria não poderiam ocupar um
único espaço. Sob a égide de tal pressuposto, Anaxágoras admitiu, tal
como Demócrito o fará mais tarde, que tais porções teriam de
esbarrar"-se, caso fossem ao encontro uma da outra em seus movimentos,
que disputariam o mesmo espaço entre si e que justamente essa luta
seria a causa de toda mudança. Noutras palavras: aquelas substâncias
inteiramente isoladas, eternamente imutáveis e completamente diferentes, 
não foram, contudo, pensadas como sendo coisas absolutamente
diferentes, mas todas possuíam, além de uma qualidade específica
e completamente particular, um substrato totalmente semelhante, a
saber, um pedaço de matéria apta a preencher o espaço. Na participação
com a matéria, todas elas se equiparavam e podiam, por isso, agir umas
sobre as outras, isto é, chocarem"-se. Geralmente, nenhuma mudança
dependia em absoluto da dessemelhança entre tais substâncias, mas de
sua semelhança como matéria. Aqui, à base da suposição de Anaxágoras,
acha"-se um lapso lógico, pois, o que existe verdadeiramente em si tem
de ser totalmente incondicionado e unívoco, não podendo pressupor, por
conseguinte, nada que fosse sua causa -- enquanto todas aquelas
substâncias de Anaxágoras possuem, no entanto, algo condicionado, a
matéria, cuja existência é por elas pressuposta; para Anaxágoras, a
substância ``vermelho'', por exemplo, não era apenas o vermelho em si,
mas, além disso, e de modo implícito, um pedaço de matéria privado de
qualidades. Somente com esta última, a matéria, o ``vermelho em si''
podia atuar sobre outras substâncias, não com o vermelho, mas com
aquilo que não é vermelho nem colorido e que, em geral, não é
qualitativamente determinado. Se o vermelho tivesse sido tomado
rigorosamente como vermelho, como a substância propriamente dita, quer
dizer, sem aquele substrato, então Anaxágoras decerto não teria
arriscado falar de um efeito do vermelho sobre outras substâncias e
tampouco enunciado que o ``vermelho em si'' propala, mediante colisão, o
movimento recebido do ``carnal em si''. Tornar"-se"-ia patente, então, que
tal ser verdadeiramente existente jamais poderia ser movido.

\sectionitem

É preciso dirigir os olhos para os adversários dos eleatas a fim de
apreciar as extraordinárias vantagens contidas na suposição de
Parmênides. Tais dilemas -- dos quais Parmênides havia escapado --
aguardavam Anaxágoras e todos aqueles que acreditavam na multiplicidade
das substâncias, quando da pergunta ``quantas substâncias?''. Anaxágoras
deu o salto, fechou os olhos e disse: ``infinitamente muitas''; ao menos,
ele se safou da tarefa inacreditavelmente exaustiva de demonstrar uma
quantidade determinada de matérias elementares. Como estas últimas
deveriam existir há eternidades, em quantidade infinita, sem acréscimo
e imutáveis, então havia, em tal suposição, a contradição pertinente a
uma infinitude fechada a ser pensada como algo acabado. Em suma, a
multiplicidade, o movimento e a infinitude, que haviam sido afugentados
por Parmênides por meio da admirável sentença sobre o ser único,
voltaram do exílio e dispararam seus tiros contra os adversários de
Parmênides, para, com eles, causar ferimentos aos quais não há cura. É
evidente que tais adversários não possuem uma clara consciência acerca
da assombrosa força daquele pensamento eleata segundo o qual ``não pode
haver nenhum tempo, movimento ou espaço, pois todos só poderiam ser
concebidos por nós como infinitos, quer dizer, uma vez como
infinitamente grandes e, aí então, como infinitamente divisíveis; mas
tudo o que é infinito não tem ser, não existe''; pensamento do qual não
pode duvidar aquele que se prende ao sentido rigoroso da palavra ser e
que toma por impossível a existência de algo contraditório, como, por
exemplo, de uma infinitude completa. Mas se justamente a efetividade
se nos mostra, a todos, apenas sob a forma da infinitude acabada, então
salta aos olhos que ela se contradiz a si mesma e que, portanto, não
possui verdadeira realidade. Se, porém, aqueles adversários desejassem
objetar: ``mas há, com efeito, sucessão em vosso próprio pensar;
portanto, também vosso pensar poderia não ser real e, consequentemente,
tampouco poderia provar coisa alguma''; então Parmênides, talvez de modo
semelhante ao de Kant num caso similar e diante de uma repreensão
análoga, bem que poderia ter respondido: ``decerto posso afirmar que
minhas representações se sucedem umas às outras, mas significa somente
que delas tomamos consciência tal como numa sucessão temporal, isto é,
de acordo com a forma do sentido interno. Por esse motivo, o tempo não
é algo em si e tampouco uma determinação objetivamente inerente às
coisas''.\footnote{ O trecho citado é, em verdade, uma nota de rodapé
contida no § 7 da Primeira Parte da Doutrina Transcendental dos
Elementos (``Estética Transcendental'') da \textit{Crítica da razão pura}
(Cf.~Kant, Immanuel. \textit{Kritik der reinen Vernunft}. Frankfurt am
Main, Suhrkamp, 1974, p.~84 [\textsc{b}55, 56/\textsc{a}38, 39].} Seria então
preciso operar uma distinção entre o puro pensar, que seria tão
atemporal quanto o ser único parmenidiano, e a consciência deste
pensar, sendo que esse último traduziria, já, o pensar sobre a forma da
aparência, quer dizer, da sucessão, da multiplicidade e do movimento. É
provável que Parmênides tivesse lançado mão desta saída; seria preciso,
aliás, objetar contra ele o mesmo que A.~Spir (\textit{Pensamento e
efetividade}, p.~264)\footnote{ Neocriticista pouco conhecido entre
nós, Afrikan Spir nasceu em 1837 na cidade de Yelisawetgrad (atual 
Kirovohrad, Ucrânia). Sua obra mais importante, \textit{Pensamento e efetividade:
tentativa de uma renovação da filosofia crítica}, foi publicada em
1873, tendo sido emprestada três vezes por Nietzsche da biblioteca da
Universidade da Basileia entre os anos de 1873 e 1875, até que, em
1877, o filósofo termina por adquirir para si a segunda edição do
livro. No que diz respeito aos textos preparados para edição, ele cita
uma outra passagem da mencionada obra em \textit{Humano, demasiado
humano} -- sem, contudo, acusar nominalmente o autor da citação: ``Se
algum dia for escrita a história da origem do pensar, então dela também
há de constar, iluminada por uma nova luz, a seguinte sentença de um
lógico excepcional: `A lei original e universal do sujeito do
conhecimento consiste na necessidade interna de conhecer cada objeto em
si, em sua essência própria, com algo idêntico a si mesmo, quer dizer,
existente por si e, no fundo, constantemente igual e inalterável, em
suma, tal como uma substância''' (Nietzsche, Friedrich.
\textit{Menschliches, Allzumenschlich} \textsc{i.} In:
\textit{Sämtliche Werke}. \textit{Kritische Studienausgabe}. Edição
organizada por Giorgio Colli e Mazzino Montinari. Berlim/Nova York,
Walter de Gruyter, 1999, § 18, vol.~\textsc{ii}, p.~38).} alega
contra Kant: ``Ora, em primeiro lugar, está claro que nada posso saber
sobre uma sucessão como tal, caso não possua, ao mesmo tempo, as
concatenações sequenciais de tal sucessão em minha consciência. A
representação de uma sucessão não é, pois, sucessiva em si mesma e, por
conseguinte, também é totalmente distinta da sucessão de nossas
representações. Em segundo lugar, a suposição de Kant implica
absurdidades tão evidentes que alguém toma por milagre o fato de ele 
tê"-las ignorado. De acordo com tal suposição, César e Sócrates
não estão efetivamente mortos, mas vivem tão bem quanto há dois mil anos
e parecem estar simplesmente mortos em virtude de um arranjo de meu
`sentido interno'. Homens do porvir já estão vivos, sendo que, se eles
ainda não surgiram como seres viventes, então isso se deve igualmente
àquele arranjo do `sentido interno'. Aqui, cumpre indagar acima de
tudo: como pode o início e o fim da própria vida consciente, incluindo
todos os seus sentidos internos e externos, existirem simplesmente à
maneira do sentido interno? O fato é que não se pode denegar, em
absoluto, a realidade da mudança. Uma vez jogada pela janela, ela então
ressurge uma vez mais pela buraco da fechadura. Dir"-se"-ia: `A mim me
parece apenas que as situações e as representações mudaram' -- mas então
tal aparência é, ela mesma, algo objetivamente dado e a sucessão
possui, nela, uma indubitável realidade objetiva, de sorte que, aqui,
as coisas realmente se sucedem umas às outras. Além disso, cumpre frisar
que a inteira crítica da razão só pode ter sentido pleno a partir do
pressuposto de que nossas próprias representações se nos apresentam tais
como elas são. Pois, se as representações também se nos apresentassem
diferentes do que efetivamente são, então não se poderia estabelecer
nenhuma afirmação válida a seu respeito e, desse modo, tampouco se
poderia levar a cabo uma teoria do conhecimento e qualquer investigação
`transcendental' de validade objetiva. Que nossas próprias
representações se nos apresentam como coisas sucessivas, eis, no entanto,
algo que está fora de dúvida''.

O exame dessa sucessão indubitavelmente certa, bem como de tal
mobilidade, exortou Anaxágoras a uma hipótese digna de ponderação.
Saltava aos olhos que as representações se moviam por si mesmas, não
eram movidas e tampouco possuíam, para além de si, alguma causa do
movimento. Então -- disse ele a si mesmo -- há algo que traz consigo a
origem e o início do movimento; mas, em segundo lugar, ele ainda \label{aorigemeoinicio}
constata que tal representação não move apenas a si própria, mas \label{asipropria}
também algo completamente diferente, a saber, o corpo. Descobre, assim,
na experiência mais imediata, um efeito das representações sobre a
matéria extensa e que se dá a conhecer como movimento desta última.
Para ele, isto era um fato; mas, apenas depois, num momento secundário,
ocorreu"-lhe esclarecê"-lo. Bastava"-lhe possuir um esquema regulativo
para o movimento no mundo, que, agora, concebia ou como um movimento
das verdadeiras essencialidades isoladas por meio daquilo que é
representativo, do \textit{nous},\footnote{ No dialeto ático, quer como
nome, quer como verbo, o termo comporta diversas definições, sendo, de resto, 
utilizado de maneiras distintas de acordo com os autores que dele se valem -- equivalendo, em
geral, às palavras ``mente'', ``entendimento'', ``pensamento'', ``razão'',
``sentido'', ``propósito'', bem como à ação de ``perceber'' ou ``apreender''
pelo intelecto. Para aquilo que nos importa, no contexto anaxagórico,
pode"-se dizer que se refere mais propriamente
à ``\textit{mente} como princípio ativo do universo'' (Cf.~Liddell, Henry
George; Scott, Robert. \textit{A Greek"-English Lexicon}. Oxford,
Clarendon Press, 1996, p.~1181).} ou, então, como movimento
através daquilo que já se move. Que este último tipo de mobilidade, a
transferência mecânica de movimentos e colisões, também traz consigo um
problema inerente à sua suposição básica, eis algo que provavelmente
lhe escapou: a banalidade e rotinização do efeito mediante colisão
decerto ofuscaram seu olhar diante de seu caráter enigmático. Em
contrapartida, ele percebeu nitidamente a natureza problemática,
inclusive contraditória de um efeito de representações sobre
substâncias existentes em si mesmas, e, por isso, procurou remeter
igualmente este efeito a eventos mecânicos de impulsão e colisão que
reputava inteligíveis. Em todo caso, o \textit{nous} decerto era tal
substância existente em si mesma e foi por ele caracterizada como uma
matéria completamente fina e delicada, com a qualidade específica do
pensar. Tendo em vista a admissão de tal caráter, o efeito de tal
matéria sobre outra deveria ser, sem dúvida, do mesmo tipo que aquele
exercido por uma substância distinta sobre uma terceira, isto é, que se
move mecanicamente por meio de empuxo e colisão. De qualquer maneira,
ele agora tinha uma substância que se move a si mesma e as demais, cujo
movimento não vem de fora nem depende de quem quer que seja; ao passo
que parecia ser praticamente indiferente a questão de como pensar esse
auto"-movimento, que se assemelhava, em certa medida, ao vaivém de
bolinhas de mercúrio muitíssimo finas e delicadas. De todas as
perguntas que dizem respeito ao movimento, nenhuma é mais incômoda do
que a pergunta pelo início do movimento; se cabe imaginar todos os
movimentos restantes como consequências e efeitos de algo, então seria
ainda sempre necessário explicar o primeiríssimo movimento inicial;
mas, para os movimentos mecânicos, o primeiro elemento da série não
pode, em todo caso, residir num movimento mecânico, já que isto
significaria o mesmo que recorrer ao conceito absurdo de \textit{causa \label{causasui}
sui}. Não se trata, apesar disso, de conferir movimento próprio às
coisas eternas e incondicionadas como se fosse, por assim dizer, um
dote inicial de sua existência. Pois não há como representar o
movimento sem uma direção de lá para cá, quer dizer, somente como
relação e condição; uma dada coisa, porém, já não mais existe em si
mesma e incondicionalmente se tiver de relacionar"-se, conforme sua
natureza, com algo que existe para além dela. Nesse dilema, Anaxágoras \label{nessedilema}
julgou encontrar uma salvação e um socorro extraordinário naquele
\textit{nous} que se move a si mesmo e que de nada depende; cuja
essência é suficientemente obscura e velada para poder iludir quanto ao
fato de que sua suposição também envolve aquela \textit{causa sui}
proibida. O ponto de vista empírico toma por certo, inclusive, que o
representar não é uma \textit{causa sui}, senão o efeito do
cérebro; sim, para tal ponto de vista, deve parecer um admirável
deboche separar o ``espírito'', isto é, o produto do cérebro de sua \textit{causa}
e, depois desta dissociação, ainda concebê"-lo como algo existente. É o
que fez Anaxágoras; esqueceu"-se do cérebro e de sua espantosa
inventividade, da delicadeza e do caráter labiríntico de suas
convoluções e decursos, e decretou o ``espírito em si''. Este ``espírito
em si'' possuía arbítrio, era a única, dentre todas as substâncias,
provida de arbítrio -- um reconhecimento soberbo! Podia, a qualquer
momento, dar início ao movimento das coisas que lhe eram exteriores,
ou, ao contrário, ocupar"-se de si mesmo durante enormes períodos de
tempo -- em suma, Anaxágoras precisou admitir, num tempo primordial, um
\textit{primeiro} momento do movimento como ponto germinal de todo
assim chamado vir"-a"-ser, isto é, de toda mudança, a saber, de
todo deslocamento e reposicionamento das substâncias eternas e suas
ínfimas partes. Ainda que o próprio espírito exista eternamente, ele de
modo algum se acha obrigado a sofrer por todas as eternidades com o
movimento circundante dos grãos de matéria; e, em todo caso, houve um
tempo e um estado destes grãos em que o \textit{nous} -- pouco importa
se isso durou muito ou pouco -- ainda não havia agido sobre eles e onde
tais grãos ainda eram imóveis. Este é o período pertinente ao caos anaxagórico.

\sectionitem

O caos anaxagórico não é uma concepção que se dá a conhecer de imediato:
para apreendê"-la, é preciso ter compreendido, antes, a representação
que o nosso filósofo criou para si do assim chamado vir"-a"-ser. Pois, o
estado de todas as diferentes existências elementares jamais
produziria, em si e antes de todo movimento, necessariamente uma
mistura absoluta de todas as ``sementes das coisas'', conforme a
expressão de Anaxágoras, uma mistura que ele imaginou como um completo \label{mistura}
penetrar"-se em si mesmo até a mais ínfima parte, depois que todas
aquelas existências elementares estivessem trituradas como numa
argamassa e dissolvidas em átomos de pó, de sorte que, em meio a tal
caos, elas pudessem ser remexidas entre si tal como numa batedeira.
Poder"-se"-ia dizer que essa concepção de caos não possui nada de
necessário; ao contrário, seria preciso admitir apenas uma posição
fortuita e acidental de todas aquelas existências, mas não uma infinita
divisibilidade das mesmas; já bastaria uma contiguidade irregular, não
carecendo de nenhuma mistura, e menos ainda de uma mistura tão global.
Como Anaxágoras chegou, pois, a esta representação árdua e complicada?
Como foi dito, por meio de sua concepção do vir"-a"-ser empiricamente
dado. A partir de sua própria experiência, ele criou, de saída, uma
sentença extremamente notável sobre o vir"-a"-ser, e tal sentença
acarretou, como sua consequência, aquela doutrina do caos.

Foi a observação dos processos da gênese na natureza, e não uma
consideração acerca de um sistema anterior, que forneceu a Anaxágoras a
doutrina segunda a qual \textit{tudo se origina de tudo}; esta era a
convicção do investigador da natureza, baseada numa indução multifária
e, no fundo, ilimitadamente exígua. Ele a demonstrou do seguinte modo:
se até mesmo o contrário pode surgir do contrário, a cor negra, por
exemplo, da cor branca, então tudo é possível; isto ocorre, pois,
quando da dissolução da neve branca em água negra. Ele elucidou a
nutrição do corpo a partir do fato de que deveriam existir, nos
alimentos, componentes invisivelmente pequenos de carne, sangue ou
osso, que, por ocasião da nutrição, se separavam e reuniam"-se àquilo
que se lhes assemelhava no corpo. Mas se tudo pode vir a ser a partir
de tudo, o sólido do líquido, o duro do macio, a cor branca da cor
negra, o que é de carne do pão, então tudo também tem de estar contido
em tudo. Os nomes das coisas expressam, pois, apenas a primazia de uma
substância sobre as outras, que possuem massas inferiores e, em geral,
são imperceptíveis. No ouro, quer dizer, naquilo que se designa
\textit{a potiore} com o nome ``ouro'', também devem estar contidos prata,
neve, pão e carne, mas em ínfimas porções, sendo que o todo é
denominado conforme aquilo que prepondera, conforme a substância ouro.

Como é possível, no entanto, que uma substância prepondere e preencha
uma coisa com massa maior do que as outras? A experiência mostra que
essa preponderância é engendrada, aos poucos, apenas por meio do
movimento, que a preponderância é o resultado de um processo que
ordinariamente chamamos de vir"-a"-ser; o fato de que tudo está presente
em tudo não é, em contrapartida, o resultado de um processo, mas, ao
contrário, o pressuposto de todo vir"-a"-ser e de todo estar"-em"-movimento
[\textit{Bewegtsein}], e, por conseguinte, antecede todo vir"-a"-ser.
Noutros termos, a empiria ensina que o semelhante é continuamente
conduzido ao semelhante, como, por exemplo, por meio da nutrição, e
que, portanto, o semelhante não se achava, de início, unido e
misturado, mas separado. De modo bem diferente, nos processos
empíricos existentes diante de nossos olhos, o semelhante é sempre
arrancado e empurrado pelo dessemelhante (como, por exemplo, na
nutrição, as partículas de carne são arrancadas do pão etc.), e, dessa
maneira, a interpenetração das diferentes substâncias constitui a forma
mais antiga de constituição das coisas e, no que concerne ao tempo, é
anterior a todo vir"-a"-ser e movimentar. Se então todo assim chamado
vir"-a"-ser é uma decomposição e pressupõe, de resto, uma mistura, cabe
indagar, pois, em que grau esta mistura, esta decomposição deve
originariamente ter ocorrido. Embora o processo seja um movimento do
semelhante rumo ao semelhante e o vir"-a"-ser perdure, já, por um período
colossal de tempo, reconhece"-se, no entanto, como ainda se acham
contidos em todas as coisas restos e grãos germinais de todas outras
coisas, à espera de sua decomposição, e como uma dada preponderância é
trazida à tona apenas aqui e acolá; a mistura primordial deve ter sido
integral, isto é, deve ter atingido o que há de infinitamente ínfimo,
já que se faz necessário um espaço infinito de tempo para desfazê"-la.
Aqui, trata"-se de aferrar"-se firmemente ao pensamento de que tudo o que
possui um ser essencial é divisível ao infinito, sem que termine por
perder, com isso, aquilo que lhe é específico.

Anaxágoras representa a existência primordial do mundo de acordo com
tais pressupostos, como se tratasse, por assim dizer, de uma massa em
pó repleta de pontos infinitamente pequenos, em que cada qual é
especificamente simples e possuidor apenas de uma qualidade, mas de tal
modo que cada qualidade específica se acha representada nos pontos
particulares infinitamente pequenos. Aristóteles chamou tais pontos de
\textit{homoiomerien}, levando em conta que são partes semelhantes num
todo que, por sua vez, também se assemelha às suas partes.		\label{suaspartes}
Enganar"-se"-ia redondamente, porém, quem igualasse esta interpenetração
primordial de todos esses pontos, de tais ``grãos germinais das coisas'',
à única matéria primordial de Anaximandro; pois, esta última,
denominada o ``ilimitado'', é uma massa uniforme e unívoca de ponta a
ponta, ao passo que a primeira é um agregado de matérias. Pode"-se, com
efeito, dizer a respeito deste agregado de matérias o mesmo que se diz
a propósito do ``ilimitado'' de Anaximandro; tal como faz Aristóteles:
não podia ser nem branco nem cinza ou de cor negra, e tampouco de
alguma outra cor qualquer, era insípido e inodoro; ele, como
totalidade, não podia, em geral, ser determinado quantitativa ou
qualitativamente; tão longe vai a semelhança entre o ilimitado de
Anaximandro e a mistura primordial de Anaxágoras. Com exceção desta
semelhança negativa, ambos se diferenciam positivamente pelo fato de a
última ser composta e o primeiro constituir uma unidade.\footnote{ Cf.,
a esse respeito, a seguinte passagem da \textit{Física}: ``Há dois modos
pelos quais os estudiosos da natureza se pronunciam. Pois uns, fazendo
um só o corpo subjacente [\ldots] geram as outras coisas, fazendo"-as
muitas, por densidade e rareza [\ldots] Outros, por sua vez, [geram as
outras coisas] por discriminar, a partir de uma só coisa, as
contrariedades lá inerentes, tal como Anaximandro afirma e também todos
aqueles que afirmam haver um e muitos, como Empédocles e Anaxágoras:
pois também eles discriminam as outras coisas a partir da mistura. E
diferenciam"-se entre si porque um deles faz um ciclo dessas coisas, ao
passo que o outro as faz uma só vez, e também porque um faz serem
discriminadas coisas ilimitadas -- tanto as homeômeras como os
contrários --, enquanto o outro discrimina apenas os chamados elementos''
(Aristóteles. \textit{Física}. Tradução de Lucas Angioni. In: ``Clássicos
da Filosofia: Cadernos de Tradução nº 1''. Campinas, \textsc{ifch}/Unicamp, 2002,
livro \textsc{i}, 187a20--23, p.~27).} Ao menos, com a suposição de seu
caos, Anaxágoras estava tão à frente de Anaximandro, que não se achava
obrigado a derivar o múltiplo do uno, bem como aquilo que vem a ser do que já existe.

Ele precisou, por certo, fazer uma exceção à sua mistura global de
germens: até então o \textit{nous} não existia e, em geral, mesmo
depois, tampouco se deixa misturar a outra coisa qualquer. Pois, caso
estivesse misturado com algo existente, então deveria residir em todas
as coisas de modo infinitamente fragmentado. Em termos lógicos, tal
exceção é assaz temerária, a começar pela natureza material do
\textit{nous}, que, tal como foi descrita anteriormente, possui algo de
mitológico e parece um tanto arbitrária, mas que, segundo as premissas
de Anaxágoras, consistia numa rígida necessidade. O espírito, que, a
propósito, como toda outra matéria, é divisível ao infinito, mas não
por meio de outras matérias, senão através de si mesmo, quando ele
se divide, decompondo"-se e amontoando"-se ora em grandes ora em pequenas
proporções, mantém sua mesma massa e qualidade por toda a eternidade;
e aquilo que, neste instante e no mundo todo, é espírito nos animais,
plantas e homens, eis algo que já existia há um milênio sem aumento ou
diminuição, ainda que dividido de maneira diferente. Mas sempre que
estabelecia uma relação com alguma outra substância, ele jamais se lhe
misturava, porém a tomava como bem quisesse, movendo"-a e
empurrando"-a a seu bel"-prazer, em suma, dominando"-a. Ele, o único a ter
movimento em si, também detém exclusivamente o domínio no mundo, o que
ele torna patente ao movimentar os grãos substanciais. Mas em que
direção ele os movimenta? Ou porventura é possível conceber um
movimento sem direção, sem curso? É o espírito igualmente arbitrário em
suas impulsões, quando impulsiona e quando não impulsiona? Em suma, é o
acaso que vigora no interior do movimento, isto é, a mais cega
arbitrariedade? Adentramos, em tais limites, naquilo que há de mais
sagrado na circunscrição das representações de Anaxágoras. 

\sectionitem

O que tinha de ser feito com aquela interpenetração caótica própria ao
estado primordial anterior a todo movimento, para que, a partir dela,
sem qualquer acréscimo de novas substâncias e forças, viesse a ser o
mundo existente com o curso regular dos astros, com as formas bem
proporcionadas entre anos e dias, com a multifária beleza e ordenação,
em suma, para que um cosmos surgisse do caos? Ele só pode ser
consequência do movimento, mas de um movimento determinado e
engenhosamente direcionado. Este movimento é, ele próprio, o instrumento
do \textit{nous}, sua meta seria a completa segregação do
semelhante, uma meta até então inalcançada, já que, no início, a
desordem e a mistura eram infinitas. Tal meta deve ser lograda somente
por meio de um processo colossal, não podendo ser produzida, de uma só
vez, através de um passe de mágica mitológico; se ela for uma vez
alcançada, num ponto do tempo infinitamente distante, de maneira que tudo
o que é semelhante tenha se agrupado e as existências primordiais,
indivisas, agora estejam acomodadas umas ao lado das outras numa bela
ordenação, se cada partícula encontrou sua pátria e seus companheiros,
quando, enfim, a grande paz tiver lugar após a grande decomposição e
cisão das substâncias, não havendo mais nada de cindido ou decomposto,
então o \textit{nous} retornará uma vez mais ao seu auto"-movimento e
não vagará mais dividido pelo mundo, ora em grandes, ora em pequenas
massas, como espírito vegetal ou animal, e tampouco residirá noutra
matéria. Entrementes, porém, a tarefa ainda não foi levada a bom termo:
o tipo de movimento de que o \textit{nous} se valeu para levá"-la a cabo
dá provas de uma fantástica conformidade a fins
[\textit{Zweckmässigkeit}], pois, por meio dele, a tarefa é, a cada 
novo momento, mais e mais solucionada. Ele tem, pois, o caráter de um
movimento circular concentricamente contínuo: sob a forma de uma
pequena rotação, teve início em algum ponto da mistura caótica e, em
cursos cada vez maiores, tal movimento circular esquadrinha todo ser
existente, expelindo, rapidamente e em toda parte, o semelhante rumo ao
semelhante. De saída, esta reviravolta giratória aproxima tudo o que é
espesso do espesso, tudo o que é fino do fino e, de igual modo, tudo o
que é escuro, claro, úmido e seco daquilo que lhe é respectivamente
semelhante; para além destas rubricas gerais, há ainda outras duas mais
abrangentes, a saber, o éter, isto é, tudo aquilo que é quente, luzente
e fino, e o ar, ou seja, tudo o que denota escuridão, frieza, gravidade
e rigidez. Por meio da separação das massas etéreas das massas aéreas
forma"-se, como primeiríssimo efeito daquela roda que gira em círculos
cada vez maiores, algo semelhante a um redemoinho que alguém provoca
numa água parada: as partes mais pesadas são levadas para o centro e se
amontoam. Do mesmo modo, aquele crescente turbilhão no caos forma"-se,
para fora, a partir das componentes etéreas, finas e luzentes, e, para
dentro, a partir das componentes nebulosas, graves e úmidas. Depois, no
curso posterior deste processo, a água separa"-se daquela massa aérea
que se mescla no interior e o elemento telúrico, por sua vez, separa"-se
novamente da água, sendo que, sob o efeito de um frio assustador, são
as pedras que, aí então, terminam por se separar do elemento telúrico.
De novo, em virtude do impacto da rotação, algumas massas rochosas
acabam sendo dilaceradas sobre a terra e lançadas em direção ao âmbito
quente e leve do éter; aí situadas, nesta instância cujo elemento ígneo
as leva à incandescência e as impele ao movimento circular etéreo, tais
massas passam a emanar luz e, como o sol e os astros, iluminam e
aquecem a terra, que, em si, é fria e escura. A concepção como um todo é de
uma deslumbrante audácia e simplicidade, e não tem em si nada daquela
teleologia canhestra e antropomórfica que foi comumente ligada ao nome
de Anaxágoras. Tal concepção adquire sua grandeza e sua altivez
justamente por derivar o cosmos inteiro do vir"-a"-ser de um círculo
móvel, ao passo que Parmênides contemplou o ser verdadeiramente
existente como uma esfera imóvel e morta. Assim que tal círculo começa
a mover"-se e que o \textit{nous} o faz girar, então toda ordenação,
regularidade e beleza do mundo tornam"-se a consequência natural daquela
primeira impulsão. Que injustiça é feita contra Anaxágoras quando se
busca recriminá"-lo pelo seu sábio afastamento da teleologia, tal como
fica nítido nesta concepção, e quando, com desdém, procura"-se falar de
seu \textit{nous} como de um \textit{deus ex machina}. Justamente por \label{exmachina}
causa da rejeição de miraculosas intervenções mitológicas e teístas,
bem como de propósitos antropomórficos e utilidades, Anaxágoras poderia
servir"-se, antes do mais, de palavras tão altivas quanto aquelas de que
se utilizou Kant em sua \textit{História natural do céu}. Com efeito,
trata"-se de um pensamento sublime a remissão total daquele esplendor do
cosmos, assim como daquele arranjo deslumbrante dos cursos das
estrelas, a um movimento pura e simplesmente mecânico e como que a uma
figura matemática móvel; portanto, uma remissão, não aos desígnios e
mãos interventoras de um \textit{deus ex machina}, mas apenas a um tipo de
oscilação que, tendo uma vez começado, prossegue de modo necessário e
determinado em seu curso, visando a efeitos comparáveis ao mais sábio
cálculo da agudeza de espírito e à conformidade a fins mais elaborada,
sem que, no entanto, assim o sejam. ``Sem o auxílio de fabulações
arbitrárias'', diz Kant, ``fruo o prazer de ver, sob o ensejo de leis
estipuladas do movimento, o engendrar"-se de uma totalidade bem ordenada
e tão semelhante ao nosso sistema de mundo, que não posso furtar"-me de
tomá"-la pelo mesmo. Num certo sentido e sem pretensão, a mim me parece
que, aqui, poder"-se"-ia dizer: `dai"-me matéria, dela pretendo erigir um
mundo!'\,''\footnote{ A passagem pertence ao Prefácio da \textit{História
universal da natureza e teoria do céu}, escrito elaborado e publicado
anonimamente por Kant em 1755.}

\sectionitem

Mesmo pressupondo que tal mistura primordial tenha sido corretamente
inferida, algumas dificuldades advindas da mecânica parecem, no
entanto, ir imediatamente de encontro ao grande delineamento da
estrutura do mundo. Se, a saber, também o espírito em algum lugar dá ensejo
a um movimento circular, ainda assim é muito difícil representar sua
continuidade, em especial, porque deve ser infinito e embalar pouco a
pouco todas as massas existentes. Poder"-se"-ia presumir, de antemão, que
a pressão de toda a matéria restante deveria comprimir este pequeno e
recém"-iniciado movimento circular; como isto não ocorre, supõe"-se que,
de repente, o \textit{nous} movente irrompa com força assustadora e tão
rapidamente que somos obrigados a chamar tal movimento de redemoinho;
da mesma maneira que Demócrito imaginou tal redemoinho. E, já que
este último tem de ser infinitamente intenso a fim de não ser
obstaculizado pelo mundo infinito inteiro que pesa sobre ele, então
será infinitamente rápido, pois, primordialmente, a intensidade só pode
revelar"-se na rapidez. Em contrapartida, quanto mais amplos são os
anéis concêntricos, mais lento será tal movimento; se o movimento
conseguisse uma vez atingir o fim do mundo infinitamente distendido,
então ele deveria possuir, já, a velocidade infinitamente pequena da
rotação. Inversamente, se concebemos o movimento como infinitamente
grande, isto é, como infinitamente veloz quando do seu primeiríssimo
irrompimento, então o círculo inicial também deve ter sido
infinitamente pequeno; desse modo, logramos como começo um ponto que
gira sobre si mesmo, com um conteúdo material infinitamente reduzido.
Este, porém, de modo algum explicaria o movimento seguinte; poder"-se"-ia
imaginar todos os pontos da massa primordial como que girando em torno
de si mesmos, mas, ainda assim, a inteira massa permaneceria imóvel e
indivisa. Se, em compensação, aquele ponto material de infinita
pequenez, tomado e empurrado pelo \textit{nous}, não fosse levado a
girar em torno de si mesmo, mas sim que descrevesse uma circunferência
acidentalmente maior, então isto já bastaria para impulsionar, mover
adiante, centrifugar e fazer ricochetear outros pontos materiais, de
maneira a provocar, aos poucos, um tumulto agitado e intensificador, no
qual o resultado mais imediato teve de ser aquela separação entre as
massas aéreas e etéreas. Assim como a inserção do movimento é, ela
mesma, um ato arbitrário do \textit{nous}, assim também é arbitrário o
tipo desta inserção, na medida em que o primeiro movimento descreve um
círculo cujo raio é fortuitamente selecionado como sendo maior que um ponto.

\sectionitem

Poder"-se"-ia decerto perguntar, aqui, o que teria acometido tão
subitamente o \textit{nous} para que, no início, viesse a impulsionar
um ínfimo e fortuito ponto material em meio a uma infinidade de outros
pontos, revirando"-o numa dança rodopiante, e por que motivo isto não
lhe havia acometido antes. A tal pergunta responderia Anaxágoras: ele
possui o privilégio do arbítrio e, dependendo apenas de si mesmo, pode
começar quando bem entender, ao passo que todo o restante é determinado
a partir do exterior. Não possui nenhuma obrigação e, portanto,		\label{naopossuinenhuma}
tampouco algum propósito que estivesse forçado a cumprir; se uma vez
deu início àquele movimento e estabeleceu para si uma meta -- a resposta
é difícil, sendo que o próprio Heráclito poderia respondê"-la --, tudo
isto não passou de um ``jogo''.

 Esta parece ter sido, sempre, a derradeira solução ou elucidação que os
gregos tinham na ponta da língua. O espírito anaxagórico é um artista,
ou, mais precisamente, o mais pujante gênio da mecânica e da arte de
construção, apto a criar, através dos meios mais simples, as mais
portentosas formas e vias e, por assim dizer, uma arquitetura
móvel, mas, seja como for, a partir daquele arbítrio irracional que se
encontra na profundidade do artista. É como se Anaxágoras apontasse
para Fídias\footnote{ Considerado o maior escultor grego do período
clássico, coube a Fídias (490--430 a.C.) elaborar e supervisionar -- 
por nomeação de Péricles -- a construção do Pártenon e do templo da
deusa Atena, na acrópole ateniense.} e, em face da descomunal
obra do artista, isto é, face a face com o cosmos, como que diante do
Pártenon, dissesse"-nos em tom exclamativo: ``o vir"-a"-ser não é um
fenômeno moral, mas artístico''. Aristóteles conta que, chamado a
explicar por que razão a existência era"-lhe, em geral, digna de apreço,
Anaxágoras respondera: ``para contemplar o céu e a ordenação total
do cosmos''.\footnote{ Cf.~\textit{Ética a Eudemo}, 1216a11--14.} 
Lidava com as coisas físicas com a mesma devoção e a reverência
sigilosa de que nos investimos quando estamos diante de um templo
antigo; sua doutrina tornou"-se um tipo de exercício religioso
libertário, protegendo"-se através do \textit{odi profanum vulgus et
arceo}\footnote{ ``Odeio a massa profana e a denego.'' Horácio,
\textit{Carminas}, \textsc{iii} 1, 1.} e elegendo cuidadosamente seus
adeptos em meio a mais elevada e nobre sociedade de Atenas. Na
confraria privativa de anaxagóricos atenienses, a mitologia do povo era
permitida apenas a título de uma linguagem simbólica; todos os mitos,
deuses e heróis valiam, aqui, apenas como hieróglifos da interpretação
da natureza, sendo que o próprio \textit{epos} homérico deveria ser como que o
canto canônico consagrado ao império do \textit{nous}, bem como às
lutas e leis da física. Aqui e acolá penetrava, no povo, um som haurido
desta sociedade de sublimes espíritos livres; e, em especial, o grande
Eurípedes, sempre arrojado e inovador, que ousou manifestar alto e bom
som, por meio das máscaras trágicas, aquilo que penetrava a massa como
uma flecha através dos sentidos e da qual conseguia libertar"-se somente
por meio de caricaturas grotescas e reinterpretações cômicas.

Mas, dentre os anaxagóricos, o maior de todos é Péricles,\footnote{ Mais
importante político de Atenas, Péricles (c.~495--429 a.C.) tomou
diversas vezes sobre os ombros, como estratego"-chefe, o destino civil e
militar da cidade. Discípulo de Zenão e Anaxágoras, acumula ainda o
mérito de ter levado a retórica ao ápice de sua consistência durante a
assim chamada ``Era de ouro'' de Atenas.} o homem mais poderoso e
digno do mundo; e é justamente referindo"-se a ele que Platão presta
testemunho de que somente a filosofia da Anaxágoras proporcionou ao seu
gênio o voo sublime.\footnote{ Cf.~\textit{Fedro}, 269a--270a.} 
Quando se colocava diante do povo como orador público, na bela
rigidez e imobilidade de um marmóreo olímpico, e, aí então, tranquilo,
enrolado em seu manto impecavelmente drapejado, sem aquela
transformação da expressão facial e sem sorrir, com aquele timbre de
voz forte e inalterado, falando, portanto, justamente à maneira de
Péricles, e de modo algum no estilo de Demóstenes, começava então a
trovejar e relampejar, aniquilava e redimia -- pois ele era a
abreviatura do cosmos anaxagórico, a imagem do \textit{nous} que erigiu
para si a mais bela e digna morada, e, por assim dizer, a nítida
encarnação humana da força engendradora, movente, divisora, ordenadora,
abrangente e artisticamente indeterminada do espírito. O próprio
Anaxágoras afirmou que o homem é, por isso mesmo, a criatura mais
racional dentre todas ou que, pelo mesmo motivo, tem de albergar o
\textit{nous} numa amplitude maior do que todas as demais criaturas, já
que possui órgãos tão dignos de admiração como as mãos; disto se concluiu,
então, que aquele \textit{nous}, a depender da grandeza e da massa com
que se apodera de um corpo material, sempre constrói para si, a partir
de tal matéria, as ferramentas condizentes com seu grau quantitativo, e
que, portanto, quando surge na mais vasta amplitude, ele termina por
engendrar as mais belas e propositadas ferramentas. E, assim como o
feito mais fantástico e propositado do \textit{nous} foi
necessariamente aquele movimento circular primordial, já que, até
então, o espírito ainda se achava indivisamente junto a si mesmo, assim
também o efeito do discurso de Péricles frequentemente assemelhava"-se,
para o atento Anaxágoras, a uma imagem alegórica daquele movimento circular
primordial; pois, também aqui ele farejou, desde logo, um ativo
redemoinho de pensamentos, detentor de uma força assustadora, mas
ordenada, que, em círculos concêntricos, apreendia e dilacerava o mais
próximo e o mais distante e que, quando chegava a seu fim, havia
redimensionado o povo inteiro, ordenando"-o e dividindo"-o.

Para os filósofos posteriores da Antiguidade, o modo pelo qual
Anaxágoras fez uso de seu \textit{nous} para explicar o mundo era algo
um tanto quanto miraculoso, sim, e quase imperdoável; parecia"-lhes que
ele havia descoberto uma ferramenta magnífica, mas não soube
compreendê"-la corretamente, de sorte que procuraram, então, recuperar
aquilo que havia escapado ao descobridor. Não reconheceram, portanto,
qual o sentido que possuía a renúncia de Anaxágoras, que, outorgada
pelo mais puro espírito do método das ciências naturais, não coloca a
pergunta pela razão de algo existir (\textit{causa finalis}), mas, em \label{causafinalis}
todos os casos e antes de tudo, a pergunta pelo meio através do qual
algo existe (\textit{causa efficiens}). O \textit{nous} só foi aventado
por Anaxágoras com vistas à resposta da pergunta especifica: ``por meio
de quê há movimento e através de quê há movimentos regulares?''; Platão
recrimina"-o, porém, por ele não ter indicado, como deveria ter feito, o
fato de que cada coisa se encontra, à sua maneira e no seu devido
lugar, na melhor, mais bela e propositada disposição.\footnote{ Trata"-se 
da seguinte passagem do \textit{Fédon}: ``Ora, certo dia ouvi
alguém que lia um livro de Anaxágoras. Dizia este que `o espírito é o
ordenador e a causa de todas as coisas'. Isso me causou alegria [\ldots]
exultei acreditando haver encontrado em Anaxágoras o explicador da
causa, inteligível para mim, de tudo que existe. Esperava que ele iria
dizer"-me, primeiro, se a terra é plana ou redonda, e, depois de o ter
dito, que à explicação acrescentasse a causa e a necessidade desse
fato, mostrando"-me ainda assim como é ela a melhor [\ldots] Mas, meu
grande amigo, bem depressa essa maravilhosa esperança se afastava de
mim! À medida que avançava e ia estudando mais e mais, notava que esse
homem não fazia nenhum uso do espírito nem lhe atribuía papel algum
como causa na ordem do universo, indo procurar tal causalidade no éter,
no ar, na água e em muitas outras coisas absurdas!'' (Platão,
\textit{Fédon}. In: ``Os pensadores''. Tradução de Jorge Paleikat e João
Cruz Costa . São Paulo, Abril Cultural, 1983, 97b--98c, p. 104).} 
Isto, porém, Anaxágoras não teria ousado afirmar em nenhum caso
particular, haja vista que, para ele, o mundo existente não era o mundo
mais perfeito que se pode pensar, pois via cada coisa surgir a partir
de cada coisa e achava que a separação das substâncias por meio do
\textit{nous} não era levada a cabo e dissolvida nem no fim do pleno
espaço no mundo nem nos seres particulares. Para o seu conhecimento,
era perfeitamente suficiente ter descoberto um movimento que, pelo
simples efeito de continuar agindo, fosse capaz de criar a ordenação
visível a partir de um caos completamente mesclado, guardando"-se de
fazer a pergunta pelo porquê do movimento, pelo seu propósito racional.
Se o \textit{nous} tivesse, pois, segundo sua essência, um propósito
necessário a cumprir por meio do movimento, então já não caberia mais
ao seu arbítrio dar início, em algum momento, ao movimento; na medida
em que existe eternamente, também já deveria estar eternamente
determinado por tal propósito e, desse modo, não poderia haver nenhum
ponto do tempo no qual ainda faltasse movimento, inclusive seria
logicamente proibido supor um ponto de partida para o movimento; em
virtude disso, a representação do caos original, o fundamento da
inteira interpretação de mundo anaxagórica, teria sido igualmente
impossível em termos lógicos. A fim de escapar de tais dificuldades
criadas pela teleologia, Anaxágoras sempre teve de enfatizar e
assegurar, em máxima medida, que o espírito é arbitrário; todos os seus
atos, até mesmo o que diz respeito àquele movimento primordial, são
atos da ``vontade livre'', ao passo que, em compensação, todo o mundo
restante forma"-se de modo rigorosamente determinado ou, melhor ainda,
de modo mecanicamente determinado após tal momento primordial. Aquela
vontade absolutamente livre só pode ser pensada, todavia, como algo sem
objetivo, mais ou menos à semelhança do jogo da criança ou do
impulso artístico e lúdico. É um equívoco quando se espera de Anaxágoras a habitual
confusão do teleologista, que, pasmado com a extraordinária
regularidade, com a harmonia das partes com o todo, em especial, no
âmbito orgânico, pressupõe que aquilo que existe para o intelecto
também veio à baila por meio do intelecto e que aquilo que ele só
consegue lograr mediante a tutela do conceito de finalidade também tem
de ser logrado pela natureza mediante reflexão e conceitos finalistas. (Cf.~Schopenhauer, 
\textit{O mundo como vontade e representação}, vol.~\textsc{ii},
p.~373.) Mas, concebida à maneira de Anaxágoras, a ordenação e a
conformidade a fins das coisas são, ao contrário, apenas o resultado
direto de um cego movimento mecânico; e apenas a fim de dar ocasião a
este movimento, para, em algum momento, sair da mórbida imobilidade do
caos, Anaxágoras supôs o \textit{nous}, arbitrário e dependente somente
de si. Conferiu a este último, como valor, a propriedade de ser
fortuito, portanto, de ser capaz de agir incondicional e
indeterminadamente, não sendo conduzido nem por causas nem por fins.


